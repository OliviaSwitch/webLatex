\chapter{Support Vector Machine: SVM}

该方法的机理可以简单描述为:寻找一个满足分类要求的最优分类超平面(\ref{sec:optimal-classification-hyperplane}),使得该超平面在保证分类精度的同时,能够使超平面两侧的空白区域最大化;从理论上来说,支持向量机能够实现对线性可分数据的最优分类。为了进一步解决非线性问题,Vapnik等人通过引入核映射方法转化为高维空间的线性可分问题来解决。

\section{最优分类超平面}\label{sec:optimal-classification-hyperplane}

对于两类线性可分的情形,可以直接构造最优超平面,使得样本集中的所有样本满足如下条件(结构风险最小化(SRM)的原则):
\begin{enumerate}
    \item 能被某一超平面正确划分
    \item 距该超平面最近的异类向量与超平面之间的距离最大,即分类间隔(margin)最大
\end{enumerate}

我们可以训练样本输入为
$$
    \mathbf{x}_i,\quad (1, 2, \cdots, l), \quad \mathbf{x}_i \in \mathbf{R}^d
$$

对应的期望输出为
$$
    y_i \in \{-1, +1\}
$$

如果训练集中的所有向量均能被某超平面正确划分,并且距离平面最近的异类向量之间的距离最大(即边缘margin最大化),则该超平面为最优超平面(Optimal Hyperplane)

% 最优超平面示意图
\begin{figure}[ht]
    \centering
    \begin{tikzpicture}[line width=1pt]
    % 绘制数据点
    \draw[fill=blue] (1,4) circle (2pt);
    \draw[fill=blue] (3,6) circle (2pt);
    \draw[fill=blue] (2,3.7) circle (2pt);
    \draw[fill=blue] (2.4,4.5) circle (2pt);
    \draw[fill=blue] (2.5,5) circle (2pt);
    \draw[fill=red] (3,1) circle (2pt);
    \draw[fill=red] (4,2) circle (2pt);
    \draw[fill=red] (3.3,2.75) circle (2pt);
    \draw[fill=red] (2,1.68) circle (2pt);
    \draw[fill=red] (3.1,2) circle (2pt);
    \draw[fill=red] (4.2,2.5) circle (2pt);

    % 绘制分类面
    \draw (0,0) node[left] {$H_2$} -- (6,5);
    \draw (0,1) node[left] {$H$} -- (6,6);
    \draw (0,2) node[left] {$H_1$} -- (6,7);

    % 绘制分类面的法向量
    \draw[<->] (5.7,4.75) -- ($(0,2)!(5.7,4.75)!(6,7)$);

    % 添加标签
    \node[right] at (5.5,5.4) {margin};
    \draw[->] (1.5,0) node[below, align=center]{支持向量\\(Support Vect)} -- (1.95,1.57);
\end{tikzpicture}


    \caption{最优超平面示意图}
    \label{fig:optimal-classification-hyperplane}
\end{figure}