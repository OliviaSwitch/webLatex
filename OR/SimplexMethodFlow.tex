% texlive2015, pdflatex



%\begin{document}
\thispagestyle{empty}
% 流程图定义基本形状
\tikzstyle{startstop} = [rectangle, rounded corners, minimum width = 2cm, minimum height=1cm,text centered, draw = black, align=center]
%\tikzstyle{io} = [trapezium, trapezium left angle=70, trapezium right angle=110, minimum width=2cm, minimum height=1cm, text centered, draw=black]
\tikzstyle{process} = [rectangle, minimum width=3cm, minimum height=1cm, text centered, draw=black]
\tikzstyle{decision} = [diamond, aspect = 3, text centered, draw=black]
\tikzstyle{point}=[coordinate,]
% 箭头形式
\tikzstyle{arrow} = [->,>=stealth]

\begin{tikzpicture}[node distance=2cm]
%定义流程图具体形状
\node[startstop](start){确定初始基本可行解(极点)\\ 即单位矩阵$x$};
\node[process, below of = start](cul1){计算
$\boldsymbol{\sigma_N^T=c_N^T-c_B^TB^{-1}N}$};
\node[decision, below of = cul1](dec1)
{所有$\boldsymbol{\sigma_N\le 0}?$};
\node[process, below of = dec1](cul2){计算
$\boldsymbol{\sigma_k}=max\{\sigma_j|\sigma_j>0\}$};
\node[decision, below of = cul2](dec2){$
\boldsymbol{B^{-1}p_k\le0}?$};
\node[process, below of = dec2](cul3){计算
$\theta=min\left\{\frac{(\boldsymbol{B^{-1}b)_{i}}}{\boldsymbol{B^{-1}p_k)}_{i}}|\boldsymbol{(B^{-1}p_k)}_{i}>0\right\}=\frac{(\boldsymbol{B^{-1}b)_{r}}}{\boldsymbol{B^{-1}p_k)}_{r}}$};
\node[process, below of = cul3](end){使$x_k$进基,$x_r$出基进行迭代};
\node[startstop, right of = dec1, xshift=3cm](stop1){停,x-opt};
\node[startstop, right of = dec2, xshift=3.5cm](stop2){停,无有限最优解};
\node[point, left of = end, node distance=5cm](point1){};
\node[point, above of = cul1, yshift=-1cm](point2){};

%连接具体形状
\draw [arrow] (start) -- (cul1);
\draw [arrow] (cul1) -- (dec1);
\draw [arrow] (dec1) -- node[right]{N} (cul2);
\draw [arrow] (cul2) -- (dec2);
\draw [arrow] (dec2) -- node[right]{N} (cul3);
\draw [arrow] (cul3) -- (end);
\draw [arrow] (dec1) -- node[above]{Y} (stop1);
\draw [arrow] (dec2) -- node[above]{Y} (stop2);
\draw [-](end) -- (point1);
\draw [arrow](point1) |- node{}(point2);

%\draw (dec1) -- node [above] {Y} (point1);
%\draw [arrow] (point1) |- (pro1);
%\draw [arrow] (dec1) -- (dec1|-pro1) -> (pro1);
%\draw [arrow] (dec1) -- node [right] {N} (pro2);
\end{tikzpicture}

%\end{document}