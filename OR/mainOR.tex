%!TEX program = lualatex
\documentclass{book}

\title{OR}
\author{G.Olivia}
\date{September 2023}

\usepackage{graphicx} % Required for inserting images
\usepackage{geometry}
\usepackage{ctex}
\usepackage{amsmath,amssymb,amsfonts}
\usepackage{hyperref}
\usepackage{palatino}
\usepackage{tikz}
\usepackage{pgfplots}
\usepackage{multirow}

\geometry{a4paper,scale=0.8}

\usetikzlibrary{shapes.geometric, arrows}

\newcommand{\subsubsubsection}[1]{\paragraph{#1}\mbox{}\\}
\setcounter{secnumdepth}{4} % how many sectioning levels to assign numbers to
\setcounter{tocdepth}{4} % how many sectioning levels to show in ToC

\begin{document}

\maketitle

\tableofcontents

\chapter{Introduction}
\section{数学基础}
\subsection{梯度与黑塞矩阵}
简便记法

\subsubsection{梯度}

\begin{equation}
\nabla f(x_1, x_2, x_3) = (\frac{\partial f}{\partial x_1}, \frac{\partial f}{\partial x_2}, \frac{\partial f}{\partial x_3})^T
\label{con:grad}
\end{equation}

\subsubsection{黑塞矩阵}

\begin{equation}
\nabla^2 f(x_1, x_2, x_3)= 
\begin{pmatrix}
\cfrac{\partial^2 f}{\partial x_1^2} & \cfrac{\partial^2 f}{\partial x_2\partial x_1} & \cfrac{\partial^2 f}{\partial x_3\partial x_1} \\
\cfrac{\partial^2 f}{\partial x_1\partial x_2} & \cfrac{\partial^2 f}{\partial x_2^2} & \cfrac{\partial^2 f}{\partial x_3\partial x_2} \\
\cfrac{\partial^2 f}{\partial x_1\partial x_3} & \cfrac{\partial^2 f}{\partial x_2 \partial x_3} & \cfrac{\partial^2 f}
{\partial x_3^2}
\label{con:Hesse matrix}
\end{pmatrix}
\end{equation}
简而言之,式(\ref{con:Hesse matrix})即为式(\ref{con:grad})逐行关于参数(即$x_1, x_2, x_3, \cdots$)求偏导。

\subsection{正定矩阵}

对一个 $n$ 阶方阵 $\boldsymbol{M}$,如果其对任意非零向量 $\boldsymbol{Z}$ 都有 $\boldsymbol{Z^TMZ}>0$,其中$\boldsymbol{Z^T}$ 为 $\boldsymbol{Z}$ 的转置,则 $\boldsymbol{M}$为正定矩阵。\\
定理:对于 $n$ 阶实对称矩阵$\boldsymbol{M}$,下列条件是等价的:
\begin{enumerate}
    \item $\boldsymbol{M}$ 是正定矩阵。
    \item $\boldsymbol{M}$ 的特征值均为正。
    \item $\boldsymbol{M}$ 的一切顺序主子式均为正。
\end{enumerate}

\section{基本理论}
\subsection{数学规划模型的一般形式}

\begin{equation}
	(fS) = \begin{cases}
	      min & f(x) \\
	      s.t. & x \in S	
		   \end{cases}
    \label{con:general form}
\end{equation}

\subsection{凸集、凸函数和凸规划}
\subsubsection{凸集}

定义:设$S \subset \boldsymbol{R}^n$,如果$\boldsymbol{x}^{(1)}, \boldsymbol{x}^{(2)} \in S, \lambda \in [0, 1]$,均有
\begin{equation}
\boldsymbol{\lambda x}^{(1)}+(1-\boldsymbol{\lambda})\boldsymbol{x}^{(2)} \in S
\end{equation}
则称$S$为凸集

集合中任意两点连成的线段必属于该集合;规定空集$\varnothing$为凸集,单点集$\{x\}$为凸集。

性质:
\begin{enumerate}
    \item 凸集的交集是凸集
    \item 凸集的内点集是凸集
    \item 凸集的闭包是凸集
    \item 分离和支撑:凸集边界上任意点存在支撑超平面;两个相互不交的凸集之间存在分离超平面
\end{enumerate}

有一特殊的凸集:凸锥

定义:设非空集合$C\subset\boldsymbol{R}^n$,如果$\forall\boldsymbol{x}\in C$对$\forall\lambda>0$有$\boldsymbol{\lambda x \in C}$,则称$C$为以0为顶点的锥(不一定含$0$点)。若$C$又是凸集,则称$C$为凸锥。

\subsubsection{凸函数}
定义2-7:设$S\subset \boldsymbol{R}^n$,非空,凸集,函数$f:S\rightarrow \boldsymbol{R}$,如果对$\forall x^{(1)}, x^{(2)}\in S, \forall\lambda\in (0,1)$恒有
\begin{equation}
f(\boldsymbol{\lambda x^{(1)}+(1-\boldsymbol{\lambda})\boldsymbol{x^{(2)}})\le\boldsymbol{\lambda}f(\boldsymbol{x^{(1)}})+(1-\boldsymbol{\lambda})f(\boldsymbol{x^{(2)}})}
\label{con:convex function}
\end{equation}
则称$f$为$S$上的凸函数。如果式(\ref{con:convex function})恒以严格不等式成立,则称$f$为$S$上的严格凸函数。

几何意义:任意两个点连线在函数曲线的上方。

\subsubsubsection{水平集}
定义2-8:设$S\subset\boldsymbol{R}^n$,非空,$f:S\rightarrow\boldsymbol{R}, a\in \boldsymbol{R},$则称
\begin{equation}
    S_{\alpha}=\{\boldsymbol{x}|f(\boldsymbol{x)}\le\alpha,\boldsymbol{x}\in S\}
    \label{con:level set}
\end{equation}
为$f$的水平集。

水平集的概念相当于在地形图中,海拔高度不高于某一数值的区域。

注意:容易证明,当$f$为凸函数时,$\forall\alpha\in\boldsymbol{R}, S_{\alpha}$是凸集。但是它的逆不成立。

\subsubsubsection{凸函数的性质}

\begin{enumerate}
    \item 定理 $f(x)$为凸集$S$上的凸函数$\Leftrightarrow S$上任意有限点的凸组合的函数值不大于各点函数值的凸组合
    \item 设$f_1,f_2$为凸函数,$\lambda_1,\lambda_2>0$,则:
        \begin{itemize}
            \item $\lambda_1f_1+\lambda_2f_2$是凸函数
            \item $f(x)=max\{f_1(x),f_2(x)\}$是凸函数
            \item $g(x)=min\{f_1(x),f_2(x)\}$不一定是凸函数
        \end{itemize}
    \item 若$f$在$S$上凸,那么$f$在$S$的内点集($intS$)上连续(注:$f$在$S$上不一定连续)
    \item 若$f$在非空凸集$S$上凸,则对任意方向的方向导数存在

    \textbf{凸函数常用判定条件:}
    \item 设$S$非空,凸集,开集,$f$在$S$上可微,则:
        \begin{itemize}
            \item $f$在$S$上凸$\rightleftharpoons \forall \bar{x}\in S$,有$f(x)\ge f(\bar{x})+\nabla f^T(\bar{x})(x-\bar{x}), \forall x\in S$。
            \item $f$在$S$上严格凸$\rightleftharpoons \forall \bar{x}\in S$,有$f(x)> f(\bar{x})+\nabla f^T(\bar{x})(x-\bar{x}), \forall x\in S,x\ne\bar{x}$。
        \end{itemize}
    \item 设$S$非空,凸集,开集,$f$在$S$上可微,则:
        \begin{itemize}
            \item $f$在$S$上凸$\rightleftharpoons \forall x^{(1)},x^{(2)}\in S,(\nabla f(x^{(1)})-\nabla f(x^{(2)}))^T(x^{(1)}-x^{(2)})\ge 0$
            \item $f$在$S$上严格凸$\rightleftharpoons \forall x^{(1)},x^{(2)}\in S, x^{(1)}\ne x^{(2)},(\nabla f(x^{(1)})-\nabla f(x^{(2)}))^T(x^{(1)}-x^{(2)})> 0$
        \end{itemize}
    \item 设$S$是开集,$f$在$S$上二次可微,则:
        \begin{itemize}
            \item $f$在$S$上凸$\rightleftharpoons \forall x\in S, \nabla^2f(x)$半正定。
            \item 如果$\forall x \in S, \nabla^2f(x)$正定,则$f$在$S$上严格凸。
        \end{itemize}
\end{enumerate}

但是,上课讲的时候,一般是使用\textbf{黑塞矩阵是否正定}判断
\begin{enumerate}
    \item 当$H$为半正定时,$f$为凸函数;
    若$H$是正半定的,当且仅当$H$的每一个主子式都大于等于0
    \item 当$H$为正定时,$f$为严格凸函数;
    若$H$是正定矩阵,当且仅当$H$的$n$个顺序主子式(严格)为正。
    \item 当$H$为半负定时,$f$为凹函数;
    若$H$是负半定的,当且仅当$H$的每一个奇阶的主子式小于等于0,每一个偶阶的主子式大于等于0
    \item 当$H$为负定时,$f$为严格凹函数;
    若$H$是负定矩阵,当且仅当$H$的$N$个顺序主子式以如下方式交替出现:
    $|A_1|<0, |A_2|>0, |A_3|<0, \cdots$
    \item 当$H$为不定时,$f$既非凸也非凹函数。
\end{enumerate}
(!注意:只能一个方向判断,不能由是否凸函数来判断矩阵是否正定)

具体可看:
\href{https://zhuanlan.zhihu.com/p/594455595}{数学选读01:矩阵的主子式与顺序主子式}

\subsubsection{凸规划}

定义:
    \begin{itemize}
    \item 当$(fS)$中,$S$为凸集,$f$是$S$上的凸函数(求$min$时),称$(fS)$为凸规划。
    \item 对于$(fgh)$,当$f, g_i$为凸函数,$h_i$为线性函数时,$(fgh)$为凸规划。
    \end{itemize}

定理2-4:设$S\subset \boldsymbol{R}^n$,非空,凸,$f:S\rightarrow\boldsymbol{R}$是凸函数。$x^*$为问题$(fS)$的l.opt.,则$x^*$为g.opt.;又如果$f$是严格凸函数,那么$x^*$是问题$(fS)$的唯一g.opt.。

\subsection{多面体、极点、极方向}
(看书上p29开始的图更好理解)

多面体:有限个半闭空间的交为多面体

极点:$x\in S$,不存在$S$中另外两个点$x^{(1)}$和$x^{(2)}$,及$\lambda\in(0,1)$,使$x=\lambda x^{(1)}+(1-\lambda)x^{(2)}$\\
根据定义,闭球体的表面上每一点都是极点;一般的闭凸锥有唯一极点,即顶点;平面没有极点;

极方向:
\begin{itemize}
    \item 方向:$d\in \boldsymbol{R}^n, d\ne 0$及对于任意$x\in S,\lambda>0$,总有$x+\lambda d\in S$(可行方向)。其中,当$d^{(1)}=\lambda d^{(2)}(\lambda>0)$时,称$d^{(1)}$和$d^{(2)}$同方向。
    \item 极方向:方向$d$不能表示为两个不同方向的组合$(d=d^{(1)}+d^{(2)})$。
\end{itemize}

\subsubsection{极点特征定理}

设$A$行满秩,$x$是$S$极点的充要条件是:

存在分解$A=(B,N)$,其中$B$为$m$阶非奇异矩阵,使$x^T=(x_B^T,x_N^T)$,这里$x_B=B^{-1}b\ge0,x_N=0$

$S$中必定存在有限多个极点$(\le C_n^m)$

\chapter{Linear Programming Problem}

\section{Graphical Method}
图解法

具体可看ppt或:
\href{https://blog.csdn.net/shulianghan/article/details/102671536}{【运筹学】线性规划 图解法 ( 唯一最优解 | 无穷最优解 | 无界解 | 无可行解 )}

基本解:各个等式约束直线的交点,外加与坐标轴的交点

基本可行解:基本解里面在可行域范围的那些基本解,可行域的顶点

最优解:基本可行解里面使目标函数最大(最小)的基本可行解


\section{Simplex Method}

一些必要概念:
(建议先了解解题步骤,再回过头来看概念)

\textbf{线性规划的标准形式}
\begin{equation}
(LP)
    \begin{cases}
        max\ &\boldsymbol{c^Tx}\\
        s.t. &\boldsymbol{Ax}=\boldsymbol{b}\\
            &\boldsymbol{x\ge0}
    \end{cases}
    \label{con:standard form of LP}
\end{equation}

其中,$\boldsymbol{A}$是$m\times n$的矩阵$(m<n)$
$$\boldsymbol{A}=
\begin{pmatrix}
    a_{11} & \cdots & a_{1m} & a_{1(m+1)} & \cdots & a_{1n} \\
    \vdots & & \vdots & \vdots & & \vdots\\
    a_{m1} & \cdots & a_{mm} & a_{m(m+1)} & \cdots & a_{mn}
\end{pmatrix}
$$
\textbf{可行解}:
在公式(\ref{con:standard form of LP})中,$\boldsymbol{x}=(x_1, \cdots, x_n)^T$即为\textbf{可行解},可以理解为多面体的极点(顶点?)\\
\textbf{基本可行解}:可以理解为满足非负约束条件的可行解?(非负约束条件大概就是公式(\ref{con:standard form of LP})中$\boldsymbol{x\ge0}$)\\
\textbf{基}:$\boldsymbol{B}$是线性规划问题的一个基
$$\boldsymbol{B}=
\begin{pmatrix}
    a_{11} & \cdots & a_{1m}\\
    \vdots & \ddots & \vdots\\
    a_{m1} & \cdots & a_{mm}
\end{pmatrix}$$
$\boldsymbol{B}$中的每个列向量$\boldsymbol{p}_j=(a_{1j}, a_{2j}, \cdots, a_{mj})^T$为\textbf{基向量},基向量中的$a_{mj}$们即为\textbf{基变量}。
一般的,我们讲基变量为$m$个线性无关的变量。(一般是标准化后的松弛变量所对应的列们的系数,也是\textbf{单位矩阵})

% texlive2015, pdflatex



%\begin{document}
\thispagestyle{empty}
% 流程图定义基本形状
\tikzstyle{startstop} = [rectangle, rounded corners, minimum width = 2cm, minimum height=1cm,text centered, draw = black, align=center]
%\tikzstyle{io} = [trapezium, trapezium left angle=70, trapezium right angle=110, minimum width=2cm, minimum height=1cm, text centered, draw=black]
\tikzstyle{process} = [rectangle, minimum width=3cm, minimum height=1cm, text centered, draw=black]
\tikzstyle{decision} = [diamond, aspect = 3, text centered, draw=black]
\tikzstyle{point}=[coordinate,]
% 箭头形式
\tikzstyle{arrow} = [->,>=stealth]

\begin{tikzpicture}[node distance=2cm]
%定义流程图具体形状
\node[startstop](start){确定初始基本可行解(极点)\\ 即单位矩阵$x$};
\node[process, below of = start](cul1){计算
$\boldsymbol{\sigma_N^T=c_N^T-c_B^TB^{-1}N}$};
\node[decision, below of = cul1](dec1)
{所有$\boldsymbol{\sigma_N\le 0}?$};
\node[process, below of = dec1](cul2){计算
$\boldsymbol{\sigma_k}=max\{\sigma_j|\sigma_j>0\}$};
\node[decision, below of = cul2](dec2){$
\boldsymbol{B^{-1}p_k\le0}?$};
\node[process, below of = dec2](cul3){计算
$\theta=min\left\{\frac{(\boldsymbol{B^{-1}b)_{i}}}{\boldsymbol{B^{-1}p_k)}_{i}}|\boldsymbol{(B^{-1}p_k)}_{i}>0\right\}=\frac{(\boldsymbol{B^{-1}b)_{r}}}{\boldsymbol{B^{-1}p_k)}_{r}}$};
\node[process, below of = cul3](end){使$x_k$进基,$x_r$出基进行迭代};
\node[startstop, right of = dec1, xshift=3cm](stop1){停,x-opt};
\node[startstop, right of = dec2, xshift=3.5cm](stop2){停,无有限最优解};
\node[point, left of = end, node distance=5cm](point1){};
\node[point, above of = cul1, yshift=-1cm](point2){};

%连接具体形状
\draw [arrow] (start) -- (cul1);
\draw [arrow] (cul1) -- (dec1);
\draw [arrow] (dec1) -- node[right]{N} (cul2);
\draw [arrow] (cul2) -- (dec2);
\draw [arrow] (dec2) -- node[right]{N} (cul3);
\draw [arrow] (cul3) -- (end);
\draw [arrow] (dec1) -- node[above]{Y} (stop1);
\draw [arrow] (dec2) -- node[above]{Y} (stop2);
\draw [-](end) -- (point1);
\draw [arrow](point1) |- node{}(point2);

%\draw (dec1) -- node [above] {Y} (point1);
%\draw [arrow] (point1) |- (pro1);
%\draw [arrow] (dec1) -- (dec1|-pro1) -> (pro1);
%\draw [arrow] (dec1) -- node [right] {N} (pro2);
\end{tikzpicture}

%\end{document}

需要注意的是,$\sigma$在表中的计算没有那么复杂,可以先略过。

\subsection{单纯形法的表格计算}

考虑规范形式的线性规划问题:$b_i>0,i=1,\cdots,m$
$$
\begin{aligned}
    &max\quad z=c_1x_1+c_2x_2+\cdots+c_nx_n&\\
    &s.t.\quad
    \begin{cases}
        a_{11}x_1+a_{12}x_2+\cdots+a_{1n}x_n&\le b_1\\
        a_{21}x_1+a_{22}x_2+\cdots+a_{2n}x_n&\le b_2\\
        \quad\vdots\\
        a_{m1}x_1+a_{m2}x_2+\cdots+a_{mn}x_n&\le b_m\\
        x_1,x_2,\cdots,x_n\ge0
    \end{cases}
\end{aligned}
$$
加入松弛变量,化为标准形:
\begin{equation}
    \begin{cases}
        max\quad &z=c_1x_1+c_2x2+\cdots+c_nx_n \\
        s.t.\quad &a_{11}x_1+a_{12}x_2+\cdots+a_{1n}x_n +x_{n+1}=b_1\\
        &a_{21}x_1+a_{22}x_2+\cdots+a_{2n}x_n \quad+x_{n+2}=b_2\\
        &\quad\vdots\\
        &a_{m1}x_1+a_{m2}x_2+\cdots+a_{mn}x_n \quad\quad+x_{n+m}=b_m\\
        &x_1,x_2,\cdots,x_n,x_{n+1},\cdots,x_{n+m}\ge0
    \end{cases}
    \label{con: eq2.2}
\end{equation}

STEPS:
\begin{enumerate}
    \item 根据公式(\ref{con: eq2.2})构造初始单纯形表:\\
    
    \begin{table}[h]
        \centering
        \begin{tabular}{c|c|c|cccccccc|c}
            \hline
             \multirow{2}{*}{$\boldsymbol{c_B}$}& \multirow{2}{*}{$\boldsymbol{x_B}$} & \multirow{2}{*}{$\boldsymbol{b}$} & $c_1$ & $c_2$ & $\cdots$ & $c_n$ & $c_{n+1}$ & $c_{n+2}$ & $\cdots$ & $c_{n+m}$ & \multirow{2}{*}{$\boldsymbol{\theta}$} \\
             \cline{4-11}
             &&& $x_1$ & $x_2$ & $\cdots$ & $x_n$ & $x_{n+1}$ & $x_{n+2}$ & $\cdots$ & $x_{n+m}$ &\\
             \hline
             $c_{n+1}$ & $x_{n+1}$ & $b_1$ & $a_{11}$ & $a_{12}$ & $\cdots$ & $a_{1n}$ & 1 & 0 & $\cdots$ & 0 & $\theta_1$\\
             $c_{n+2}$ & $x_{n+2}$ & $b_2$ & $a_{21}$ & $a_{22}$ & $\cdots$ & $a_{2n}$ & 0 & 1 & $\cdots$ & 0 & $\theta_2$\\
             $\vdots$ & $\vdots$ & $\vdots$ & $\vdots$ & $\vdots$ &  & $\vdots$ & $\vdots$ & $\vdots$ &  & $\vdots$ & $\vdots$\\
             $c_{n+m}$ & $x_{n+m}$ & $b_m$ & $a_{m1}$ & $a_{m2}$ & $\cdots$ & $a_{mn}$ & 0 & 0 & $\cdots$ & 1 & $\theta_m$\\
             \hline
             \multicolumn{2}{c|}{$-z$}  & $-z^{\prime}$ & $\sigma_1$ & $\sigma_2$ & $\cdots$ & $\sigma_n$ & 0 & 0 & $\cdots$ & 0 &\\
             \hline
        \end{tabular}
        \caption{单纯形表}
        \label{tab:label de SM}
    \end{table}
    表(\ref{tab:label de SM})中;
    \begin{itemize}
        \item 列$\boldsymbol{c_B}$填入目标函数中基变量$x$的系数
        \item 列$\boldsymbol{x_B}$填入基变量
        \item 列$\boldsymbol{b}$填入约束方程右端的常数
        \item 中间4-11 列填入约束方程中$x$的系数,其中从$x_{n+1}$开始的列组成单位矩阵
    \end{itemize}
    \item 求出$-z^{\prime}=-\sum\limits_{i=1}^mc_{n+i}b_i$(初始状态下一般为0)
    \item 求出\textbf{检验数}$\boldsymbol{\sigma}_j=c_j-\sum\limits_{i=1}^mc_{n+i}a_{ij}$,可以通俗理解为$\sigma_1=$第一行中的$c_1$减去($c_B$的每一项与$c_1$列对应行的$a_{\_1}$相乘的求和)。\\
    需要注意的是,这里的检验数不需要考虑基(即单位矩阵)对应的列,否则会影响下一步。
    \item 判断是否所有检验数$\sigma_j\le0$,如果全部小于等于0,则当前的基本可行解是最优解;如果有一检验数大于0,那么进行下一步的计算。
    \item 求入基变量$x_k$:
    \begin{itemize}
        \item 先求出大于0的检验数中最大的\textbf{检验数}$\boldsymbol{\sigma_k}$
        \item 得到的下标$k$,$\boldsymbol{\sigma_k}$所在的列就是主元列
        \item 那么所对应的$x_k$就是入基变量
    \end{itemize}
    \item 求出基变量$x_r$:
    \begin{itemize}
        \item 找出主元列($x_k$列)对应元素$a_{ik}>0$的
        \item 使每个大于0的$a_{ik}$被对应行中的$b_i$除,得到$\theta_i=\frac{b_i}{a_{ik}}$\\
        (若$a_{ik}\le0$,则$\boldsymbol{\theta_i}=\infty$)
        \item 找出$min\{\boldsymbol{\theta_i}\}=\boldsymbol{\theta_r}$
        \item 此时$\boldsymbol{\theta_r}$所在的行$r$即为主元行,$x_r$为出基变量
    \end{itemize}
    \item 求出入基变量$x_k$和出基变量$x_r$后,就可以构建下一张表,这张表中:
    \begin{itemize}
        \item $\boldsymbol{x_B}$上的出基变量被入基变量所取代,同时需要更改列$\boldsymbol{c_B}$中与之相对应的行的元素
        \item 在表右侧中间部分,使用高斯消元法将$x_k$对应的$a$与其他基变量对应的$a$组成一个单位矩阵,同时列$\boldsymbol{b}$也跟着改变
        \item 表2基本构建完成,重复第二步,直至出现流程表中stop的情况
    \end{itemize}
\end{enumerate}

\subsection{一般线性规划问题的处理}

在某些情况下,如基本初始可行解不明显,即很难在标准形的问题下找到单位矩阵时,可以考虑使用\textbf{大M法}与\textbf{二阶段法}

\subsubsection{大M法}
STEPS:
\begin{enumerate}
    \item 将线性规划问题转化为标准型
    \item 观察变量,若初始基本可行解明显,直接进行单纯形法;否则引入人工变量 $x_{n+i} \ge 0 (i = 1 , … ,m)$及\textbf{充分大正数}$M$,改写原目标函数,进行单纯形法
    \item 若得到的最优解满足: $$
    x_{n+i} = 0,(i = 1 , … , m)
    $$则是原问题的最优解;否则,原问题无可行解
\end{enumerate}

例:使用大M法求解下面的问题
$$
\begin{aligned}
    &max\quad z=5x_1+2x_2+3x_3-x_4&\\
    &s.t.\quad
    \begin{cases}
        x_1+2x_2+3x_3&= 15\\
        2x_1+x_2+5x_3&= 20\\
        x_1+2x_2+4x_3+x_4&= 26\\
        x_1,x_2,x_3,x_4\ge0
    \end{cases}
\end{aligned}
$$
标准化并引入人工变量:
$$
\begin{aligned}
    &max\quad z=5x_1+2x_2+3x_3-x_4-Mx_5-Mx_6&\\
    &s.t.\quad
    \begin{cases}
        x_1+2x_2+3x_3\qquad+x_5&= 15\\
        2x_1+x_2+5x_3\qquad\qquad+x_6&= 20\\
        x_1+2x_2+4x_3+x_4&= 26\\
        x_1,x_2,x_3,x_4,x_5,x_6\ge0
    \end{cases}
\end{aligned}
$$
最后使用单纯形法计算
    \begin{table}[h]
        \centering
        \begin{tabular}{c|c|c|cccccc|c}
            \hline
             \multirow{2}{*}{$\boldsymbol{c_B}$}& \multirow{2}{*}{$\boldsymbol{x_B}$} & \multirow{2}{*}{$\boldsymbol{b}$} & 5 & 2 & 3 & -1 & $-M$ & $-M$ & \multirow{2}{*}{$\boldsymbol{\theta}$} \\
             \cline{4-9}
             &&& $x_1$ & $x_2$ & $x_3$ & $x_4$ & $x_5$ & $x_6$ &\\
             \hline
             $-M$ & $x_5$ & 15 & 1 & 2 & 3 & 0 & 1 & 0 & 5\\
             $-M$ & $x_6$ & 20 & 2 & 1 & [5] & 0 & 0 & 1 & 4\\
             -1 & $x_4$ & 26 & 1 & 2 & 4 & 1 & 0 & 0 & 6.5\\
             \hline
             \multicolumn{2}{c|}{$-z$}  & $35M+26$ & $3M+6$ & $3M+4$ & $8M+7$ & 0 & 0 & 0 &\\
             \hline
             $-M$ & $x_5$ & 3 & -1/5 & [7/5] & 0 & 0 & 1 & -3/5 & 15/7\\
             3 & $x_3$ & 4 & 2/5 & 1/5 & 1 & 0 & 0 & 1/5 & 20\\
             -1 & $x_4$ & 10 & -3/5 & 6/5 & 0 & 1 & 0 & -4/5 & 25/3\\
             \hline
             \multicolumn{2}{c|}{$-z$}  & $3M-2$ & $-M/5+16/5$ & $7/5M+13/5$ & 0 & 0 & 0 & -8/5M-7/5 &\\
             \hline
             2 & $x_2$ & 15/7 & -1/7 & 1 & 0 & 0 & 5/7 & -3/7 & \\
             3 & $x_3$ & 25/7 & [3/7] & 0 & 1 & 0 & -1/7 & 2/7 & 25/3\\
             -1 & $x_4$ & 52/7 & -3/7 & 0 & 0 & 1 & -6/7 & -2/7 & \\
             \hline
             \multicolumn{2}{c|}{$-z$}  & -53/7 & 25/7 & 0 & 0 & 0 & -M-13/7 & -M-2/7 &\\
             \hline
             2 & $x_2$ & 10/3 & 0 & 1 & 1/3 & 0 & 2/3 & -1/3 & \\
             5 & $x_1$ & 25/3 & 1 & 0 & 7/3 & 0 & -1/3 & 2/3 & \\
             -1 & $x_4$ & 11 & 0 & 0 & 1 & 1 & -1 & -0 & \\
             \hline
             \multicolumn{2}{c|}{$-z$}  & -112/3 & 0 & 0 & -25/3 & 0 & -M-2/3 & -M+8/3 &\\
             \hline
        \end{tabular}
        \caption{大M法例题}
        \label{tab:example de big M method}
    \end{table}

    得到基本可行解:$(25/3,10/3,0,11)^T$为最优解
    
    得到最优值$z=112/3$

\subsubsection{二阶段法}
STEPS:
\begin{enumerate}
    \item 将线性规划问题转化为标准型
    \item 观察变量,若初始基本可行解明显,直接进行单纯形法;
    否则引入人工变量 $x_{n+i} \ge 0 (i= 1 , … , m)$构造辅助问题(LP - 1)
    \item 第一阶段,求解辅助问题(LP - 1),若得到的最优解满足 $x_{n+i} = 0 (i= 1 , … , m)$,则是原问题的基本可行解;\textbf{否则,原问题无可行解}。
    \item 第二阶段,得到原问题的基本可行解后,直接删除人工变量,求解原问题
\end{enumerate}

例:使用二阶段法求解下面的问题
$$
\begin{aligned}
    &max\quad z=5x_1+2x_2+3x_3-x_4&\\
    &s.t.\quad
    \begin{cases}
        x_1+2x_2+3x_3&= 15\\
        2x_1+x_2+5x_3&= 20\\
        x_1+2x_2+4x_3+x_4&= 26\\
        x_1,x_2,x_3,x_4\ge0
    \end{cases}
\end{aligned}
$$
标准化、引入人工变量并构造第一阶段问题(LP - 1):
$$
\begin{aligned}
    &max\quad z^{\prime}=-x_5-x_6&\\
    &s.t.\quad
    \begin{cases}
        x_1+2x_2+3x_3\qquad+x_5&= 15\\
        2x_1+x_2+5x_3\qquad\qquad+x_6&= 20\\
        x_1+2x_2+4x_3+x_4&= 26\\
        x_1,x_2,x_3,x_4,x_5,x_6\ge0
    \end{cases}
\end{aligned}
$$

建立第一阶段的单纯形表:
\begin{table}[h]
    \centering
        \begin{tabular}{c|c|c|cccccc|c}
            \hline
             \multirow{2}{*}{$\boldsymbol{c_B}$}& \multirow{2}{*}{$\boldsymbol{x_B}$} & \multirow{2}{*}{$\boldsymbol{b}$} & 0 & 0 & 0 & 0 & -1 & -1 & \multirow{2}{*}{$\boldsymbol{\theta}$} \\
             \cline{4-9}
             &&& $x_1$ & $x_2$ & $x_3$ & $x_4$ & $x_5$ & $x_6$ &\\
             \hline
             -1 & $x_5$ & 15 & 1 & 2 & 3 & 0 & 1 & 0 & 5\\
             -1 & $x_6$ & 20 & 2 & 1 & [5] & 0 & 0 & 1 & 4\\
             0 & $x_4$ & 26 & 1 & 2 & 4 & 1 & 0 & 0 & 6.5\\
             \hline
             \multicolumn{2}{c|}{$-z$}  & 35 & 3 & 3 & 8 & 0 & 0 & 0 &\\
             \hline
             -1 & $x_5$ & 3 & -1/5 & [7/5] & 0 & 0 & 1 & -3/5 & 15/7\\
             0 & $x_3$ & 4 & 2/5 & 1/5 & 1 & 0 & 0 & 1/5 & 20\\
             0 & $x_4$ & 10 & -3/5 & 6/5 & 0 & 1 & 0 & -4/5 & 25/3\\
             \hline
             \multicolumn{2}{c|}{$-z$}  & 3 & -1/5 & 7/5 & 0 & 0 & 0 & -8/5 &\\
             \hline
             2 & $x_2$ & 15/7 & -1/7 & 1 & 0 & 0 & 5/7 & -3/7 & \\
             3 & $x_3$ & 25/7 & [3/7] & 0 & 1 & 0 & -1/7 & 2/7 & 25/3\\
             -1 & $x_4$ & 52/7 & -3/7 & 0 & 0 & 1 & -6/7 & -2/7 & \\
             \hline
             \multicolumn{2}{c|}{$-z$}  & 0 & 0 & 0 & 0 & 0 & -1 & -1 &\\
             \hline
        \end{tabular}
        \caption{二阶段法例题1}
        \label{tab:example de two-phase method1}
\end{table}
\\
    得到原问题的基本可行解$(0,15/7,25/7,52/7)^T$

第二阶段 删除人工变量,并把基本可行解填入表中

\begin{table}[h]
    \centering
        \begin{tabular}{c|c|c|cccc|c}
            \hline
             \multirow{2}{*}{$\boldsymbol{c_B}$}& \multirow{2}{*}{$\boldsymbol{x_B}$} & \multirow{2}{*}{$\boldsymbol{b}$} & 0 & 0 & 0 & 0 & \multirow{2}{*}{$\boldsymbol{\theta}$} \\
             \cline{4-7}
             &&& $x_1$ & $x_2$ & $x_3$ & $x_4$ &\\
             \hline
             2 & $x_2$ & 15/7 & -1/7 & 1 & 0 & 0 & \\
             3 & $x_3$ & 25/7 & [3/7] & 0 & 1 & 0 & 25/3\\
             -1 & $x_4$ & 52/7 & -3/7 & 0 & 0 & 1 & \\
             \hline
             \multicolumn{2}{c|}{$-z$}  & -53/7 & 25/7 & 0 & 0 & 0 &\\
             \hline
             2 & $x_2$ & 10/3 & 0 & 1 & 1/3 & 0 & \\
             5 & $x_1$ & 25/3 & 1 & 0 & 7/3 & 0 & \\
             -1 & $x_4$ & 11 & 0 & 0 & 1 & 1 & \\
             \hline
             \multicolumn{2}{c|}{$-z$}  & -112/3 & 0 & 0 & -25/3 & 0 &\\
             \hline
        \end{tabular}
        \caption{二阶段法例题2}
        \label{tab:example de two-phase method2}
\end{table}
    得到基本可行解:$(25/3,10/3,0,11)^T$为最优解
    
    得到最优值$z=112/3$

    (表见下页)
\\
\\
\\

\subsection{线性规划的对偶问题}

给定一个优化问题,我们去理解它的时候,或者设计算法的时候,可以研究它的对偶。

有时原问题不好解,但它的对偶相对容易。这个时候,可以从对偶问题出发,进而寻求原问题的解。

\subsubsection{对偶问题的形式}

\begin{enumerate}
    \item 对称形式的对偶问题
    \begin{flalign*}
    & (P)
        \begin{cases}
            MAX\quad &z=\boldsymbol{c^T x}\\
            s.t.\quad &Ax \le b\\
            &x\ge 0
        \end{cases}
    & (D)
        \begin{cases}
            MIN\quad &f=\boldsymbol{b^T y}\\
            s.t.\quad &A^Ty \ge c\\
            &y\ge 0
        \end{cases}
        &&
    \end{flalign*}

    一对对称形式的对偶规划之间具有下面的对应关系
    \begin{itemize}
        \item “$Max,\le$”和“$Min,\ge$”相对应
        \item 从约束系数矩阵看:一个模型中为$A$,则另一个模型中为$A^T$;一个模型是m个约束、n个变量,则它的对偶模型为n个约束、m个变量
        \item 从数据b、c的位置看:在两个规划模型中,b和c的位置对换
        \item 两个规划模型中的变量皆非负
    \end{itemize}
    
    \item 非对称形式的对偶问题

    (下式并不囊括所有情况)

    \begin{flalign*}
    & (P)
        \begin{cases}
            MAX\quad &z=\boldsymbol{c^T x}\\
            s.t.\quad &Ax = b\\
            &x\ge 0
        \end{cases}
    & (D)
        \begin{cases}
            MIN\quad &f=\boldsymbol{b^T y}\\
            s.t.\quad &A^Ty \ge c
        \end{cases}
        &&
    \end{flalign*}

    一般称不具有对称形式的一对线性规划为非对称形式的对偶规划。
    对于非对称形式的规划,可以按照下面的对应关系直接给出其对偶规划。
    \begin{itemize}
        \item 将模型统一为“$Max,\le$”或“$Min,\ge$” 的形式
        \item 若原规划的某个约束条件为等式约束,则在对偶规划中与此约束对应的那个变量取值没有非负限制
        \item 若原规划的某个变量的值没有非负限制,则在对偶问题中与此变量对应的那个约束为等式
    \end{itemize}
\end{enumerate}

原问题与对偶问题的对应关系

\begin{table}[h]
    \centering
        \begin{tabular}{cc|cc}
            \hline
            \multicolumn{2}{c|}{原问题(对偶问题)}& \multicolumn{2}{c}{对偶问题(原问题)}\\
            \hline
            \multicolumn{2}{c|}{min} & \multicolumn{2}{c}{max}\\
            \hline
            \multirow{4}{*}{变量} & n个变量 & \multirow{4}{*}{约束} & n个约束\\
            & 变量$\ge$0 && 约束$\le$ \\
            & 变量$\le$0 && 约束$\ge$\\
            & 无正负限制 && 约束 = \\
            \hline
            \multirow{4}{*}{约束} & m个约束 & \multirow{4}{*}{变量} & m个变量\\
            & 约束$\le$ && 变量$\le$0 \\
            & 约束$\ge$ && 变量$\ge$0\\
            & 约束 = && 无正负限制 \\
            \hline
            \multicolumn{2}{c|}{约束条件右端项} & \multicolumn{2}{c}{目标函数中的变量系数}\\
            \multicolumn{2}{c|}{目标函数中的变量系数} & \multicolumn{2}{c}{约束条件右端项}\\
            \hline
        \end{tabular}
        \caption{原问题与对偶问题的对应关系}
        \label{tab:corresponding relationship between problem
        P and the problem D}
\end{table}

\subsubsection{对偶定理}

设有一对互为对偶的线性规划

\begin{flalign*}
    & (P)
        \begin{cases}
            MAX\quad &z=\boldsymbol{c^T x}\\
            s.t.\quad &Ax \le b\\
            &x\ge 0
        \end{cases}
    & (D)
        \begin{cases}
            MIN\quad &f=\boldsymbol{b^T y}\\
            s.t.\quad &A^Ty \ge c\\
            &y\ge 0
        \end{cases}
        &&
\end{flalign*}

定理3-3:若$\boldsymbol{x}$和$\boldsymbol{y}$分别为原规划$(P)$和$(D)$对偶规划,则
$$
\boldsymbol{c}^T\boldsymbol{x}\le \boldsymbol{b}^T\boldsymbol{y}
$$

推论3-1:设$\boldsymbol{x}$和$\boldsymbol{y}$分别为原规划$(P)$和$(D)$的可行解,当$\boldsymbol{c}^T\boldsymbol{x}= \boldsymbol{b}^T\boldsymbol{y}$时,$\boldsymbol{x},\boldsymbol{y}$分别是两个问题的最优解

推论3-2:若规划$(P)$有可行解,则规划$(P)$有最优解的充分必要条件是规划$(D)$有可行解

推论3-3:若规划$(D)$有可行解,则规划$(D)$有最优解的充分必要条件是规划$(P)$有可行解

定理3-4:若原规划$(P)$有最优解,则对偶规划$(D)$也有最优解,反之亦然,且两者的目标函数值相等。

\subsubsection{对偶单纯形法}

对偶单纯形法是求解原规划的一种方法。

\textbf{原理}:

作为原规划的一个解,会有两个性质等待满足:可行性和最优性。而原问题的可行性和最优性恰好对应对偶问题的最优性和可行性。

单纯形法的思路是,先满足可行性,再逐渐逼近最优性;而对偶单纯形法的思路是,先找到最优性,再逐渐逼近可行性。

也就是说,先找到对偶问题的可行解,再找到原问题的可行解(即对偶问题的最优解)。

最优性:看检验数$\sigma_j$
可行性:看右端项$b_i$

从原规划的一个基本解出发,此基本解不一定可行,但它对应着一个对偶可行解;就是说可以从一个对偶可行解出发,然后检验原规划的基本解是否可行,即是否有负的分量。如果有负的分量,则进行迭代,求另一个基本解,此基本解对应着另一个对偶可行解(检验数非正);而得到的基本解的分量皆非负,则该基本解为最优解。

也就是说,对偶单纯形法在迭代过程中始终保持对偶解的可行性(检验数非正),使得原规划的基本解由不可行变为可行,当同时得到对偶规划与原规划的可行解时,得到原规划的最优解。

STEPS:
\begin{enumerate}
    \item 根据线性规划典式形式,建立初始单纯形表(就是还按照单纯形法填初始单纯形表)。此表对应原规划的一个基本解。表要求:检验数数行各元素一定非正,原规划的基本解可以有小于零的分量。
    \item 若基本解的所有分量皆非负,则得到原规划的最优解,停止计算;若基本解中有小于零的分量$b_i$,并且$b_i$所在行各系数$a_{ij}\ge0$,则原规划无可行解,停止计算;若$b_i<0$,并且存在$a_{ir}<0$,则确定$x_r$为出基变量,并计算$$\theta=min\left\{\frac{\sigma_j}{a_{rj}}|a_{rj}<0\right\}=\frac{\sigma_k}{a_{rk}}$$确定$x_k$为进基向量。若有多个$b_i<0$,则选择最小的进行分析计算。
    \item 以$b_{rk}$为中心元素,按照与单纯形法类似的方法,在表中进行迭代计算,返回第2步。
\end{enumerate}

%\begin{figure}[h]
%    \centering
%    \includegraphics[scale=.5]{DualSMFlow.pdf}
%    \caption{单纯形法与对偶单纯形法流程图}
%    \label{con:DualSMFlow}
%\end{figure}


\tikzstyle{startstop} = [rectangle, rounded corners, minimum width=2cm, minimum height=1cm,text centered, draw=black]
\tikzstyle{io} = [minimum width=3cm, minimum height=1cm, text centered]
\tikzstyle{process} = [rectangle, minimum width=3cm, minimum height=1cm, text width=4.7cm, text centered, draw=black]
\tikzstyle{decision} = [diamond, aspect=2, minimum width=3cm, minimum height=1cm, text centered, draw=black]
\tikzstyle{point}=[coordinate,]
%\tikzstyle{arrow} = [thick,-Stealth]
\tikzstyle{arrow} = [->,>=stealth]

\begin{tikzpicture}[node distance=2.4cm]

    % Define nodes for the primal simplex method
    \node[io](start1){单纯形法};
    \node[process, below of = start1, yshift=1.5cm](cul1){典式对应原规划的\\ 基本解是可行的};
    \node[decision, below of = cul1](dec1){所有$\sigma_i\le0?$};
    \node[process, below of = dec1](cul2){计算
        $\sigma_k=\max\{\sigma_j|\sigma_j>0\}$};
    \node[decision, below of = cul2](dec2){所有$a_{ik}\le0$?};
    \node[process, below of = dec2](cul3){计算\\ $\theta=\min\left\{\frac{b_i}{a_{ik}}|a_{ik}>0\right\}=\frac{b_e}{a_{ek}}$};
    \node[process, below of = cul3](end1){以$a_{ek}$为中心元素进行迭代};

    % Define nodes for the dual simplex method
    \node[io, right of = start1, xshift=7.5cm](start2){对偶单纯形法};
    \node[process, below of = start2, yshift=1.5cm](pro1){ 典式对应原规划的\\ 基本解的检验数$\sigma_j\le0$ };
    \node[decision, below of = pro1](deci1){所有$b_i\ge0?$};
    \node[process, below of = deci1](pro2){计算
        $b_e=\min\{b_i|b_i<0\}$};
    \node[decision, below of = pro2](deci2){所有$a_{lj}\ge0$?};
    \node[process, below of = deci2](pro3){计算\\ $\theta=\min\left\{\frac{\sigma_i}{a_{ej}}|a_{ej}<0\right\}=\frac{\sigma_k}{a_{ek}}$};
    \node[process, below of = pro3](end2){以$a_{ek}$为中心元素进行迭代};


    % center
    \node[process, right of=dec1, xshift=2.6cm, text width=1.5cm](cpro1){得到\\最优解};
    \node[startstop, below of=cpro1](cstop){结束};
    \node[process, right of=dec2, xshift=.7cm, minimum width=1cm, text width=.5cm](cpro2){没有有限最优解};
    \node[process, left of=deci2, xshift=-.7cm, minimum width=1cm, text width=.5cm](cpro3){没有有限最优解};

    % return
    \node[point, left of = end1, node distance=3cm](point1){};
    \node[point, above of = dec1, yshift=-1cm](point2){};
    \node[point, right of = end2, node distance=3cm](point3){};
    \node[point, above of = deci1, yshift=-1cm](point4){};

    % arrow
    \draw [arrow] (cul1) -- (dec1);
    \draw [arrow] (dec1) -- node[right]{N} (cul2);
    \draw [arrow] (cul2) -- (dec2);
    \draw [arrow] (dec2) -- node[right]{N} (cul3);
    \draw [arrow] (cul3) -- (end1);

    \draw [arrow] (pro1) -- (deci1);
    \draw [arrow] (deci1) -- node[right]{N} (pro2);
    \draw [arrow] (pro2) -- (deci2);
    \draw [arrow] (deci2) -- node[right]{N} (pro3);
    \draw [arrow] (pro3) -- (end2);

    \draw [arrow] (dec1) -- node[above]{Y} (cpro1);
    \draw [arrow] (deci1) -- node[above]{Y} (cpro1);
    \draw [arrow] (cpro1) -- (cstop);
    \draw [arrow] (dec2) -- node[above]{Y} (cpro2);
    \draw [arrow] (cpro2) -- (cstop);
    \draw [arrow] (deci2) -- node[above]{Y} (cpro3);
    \draw [arrow] (cpro3) -- (cstop);

    \draw [-](end1) -- (point1);
    \draw [arrow](point1) |- node{}(point2);
    \draw [-](end2) -- (point3);
    \draw [arrow](point3) |- node{}(point4);

\end{tikzpicture}



\subsection{灵敏度分析}

前提:在求灵敏度分析和影子价格中,常常会遇到要使用对偶问题的最优解(或最优基$\boldsymbol{B}$)求解问题,这里解释一下怎么使用最优单纯形表求得。

\begin{enumerate}
    \item 首先利用单纯形法得到最优单纯形表
    \item 得到$b$列的$n$个数值,最优单纯形表从后往前数$n$列(就是松弛变量对应的列),有矩阵:$$\begin{pmatrix}
        a_{1(m-n)} & a_{1(m-n+1)} & \cdots & a_{1m}\\
        a_{2(m-n)} & a_{2(m-n+1)} & \cdots & a_{2m}\\
        \vdots &&& \vdots\\
        a_{n(m-n)} & a_{n(m-n+1)} & \cdots & a_{nm}\\
        \hline
        \sigma_{m-n} & \sigma_{m-n} & \cdots & \sigma_{m-n}
    \end{pmatrix}$$
    其中:
    \begin{itemize}
        \item 最后一行的检验数$\sigma^T =- \boldsymbol{c_B^TB}^{-1}$,各检验数取相反数(即$\boldsymbol{c_B^TB}^{-1}$)即为对偶问题的最优解。
        \item 除最后一行外(即线上方的)矩阵即为$\boldsymbol{B}^{-1}$
    \end{itemize}
\end{enumerate}

\subsubsection{影子价格}

影子价格是一个向量,它的分量表示最优目标值随相应资源数量变化的变化率。

若$x^*, y^*$分别为$(LP)$和$(DP)$的最优解,那么有$$\boldsymbol{c^Tx^*}=\boldsymbol{b^Ty^*}$$根据$f=\boldsymbol{b^Ty^*}=b_1y^*_1, b_2y^*_2, \cdots, b_my^*_m$可知$$\frac{\partial f}{\partial b_i} = y_i^*$$

$y_i^*$表示$b_i$变化一个单位对目标$f$产生的影响,称$y_i^*$为$b_i$的影子价格

需要指出,影子价格不是固定不变的,当约束条件、产品利润等发生变化时,有可能使影子价格发生变化。另外,影子价格的经济含义,是指资源在一定范围内增加时的情况,当某种资源的增加超过了这个“一定的范围”时,总利润的增加量则不是按照影子价格给出的数值线性地增加。

如何求解影子价格:
\begin{enumerate}
    \item 求出对偶问题的最优解(可以利用单纯形表,在求出最优解的情况下,将松弛变量对应的各检验数取负$-\sigma_i$就能得到对偶问题的最优解)
    \item 对偶问题最优解中的数字依次对应的就是原问题的各资源影子价格
\end{enumerate}

\textbf{经济意义}:每增加一单位的某资源,最终收益增加多少单位

判断资源是否有剩余:$\begin{cases}
    y_i^*=0\quad \mbox{有剩余}\\
    y_i^*>0\quad \mbox{无剩余}
\end{cases}$

\subsubsection{目标函数系数c变化$\star$}
若只有一个系数$c_j$变化,其他系数不变。$c_j$的变化只影响检验数$\sigma_j$,而不影响解的非负性。
$$\sigma_j=c_j-\boldsymbol{c_B}^T\boldsymbol{B}^{-1}\boldsymbol{p}_j, j=1, 2, \cdots, n $$
\begin{enumerate}
    \item $c_k$是非基变量的系数\\
    非基系数变化只影响与$c_k$有关的一个检验数$\sigma_k$的变化,对其他无影响,故只需要考虑$\sigma_k$。\\
    设$c_k\rightarrow \bar{c}_k=c_k + \Delta c_k$,有$\sigma_k$的变化:$$\bar{\sigma}_k=c_k + \Delta c_k-\boldsymbol{c_B}^T\boldsymbol{B}^{-1}\boldsymbol{p}_j= \sigma_k+ \Delta c_k$$为了保持最优解不变,$\sigma_k$必须满足$\bar{\sigma}_k=\sigma_k+\Delta c_k\le 0$。也就是说:
    \begin{equation}
        \Delta c_k\le -\sigma_k,\bar{c}_k=c_k+\Delta c_k\le c_k-\sigma_k
        \label{con:2.3}
    \end{equation}
    $c_k-\sigma_k$是$c_k$变化的上限,若$c_k$不超出上限,最优解不变;否则,将最优单纯形表中的检验数$\sigma_k$用$\bar{\sigma}_k$取代,取$x_k$为进基变量,继续单纯形的表格计算。

    \item $c_l$是基变量的系数\\
    设$c_l\rightarrow c_l+\Delta c_l$,引入$\Delta c=(0, \cdots, 0, \Delta c_l, 0, \cdots, 0)$,有
    \begin{align*}
        \sigma_j\rightarrow\bar{\sigma}_j&=c_j-[\boldsymbol{c_B}^T+(\Delta\boldsymbol{c})^T]\boldsymbol{B}^{-1}\boldsymbol{p}_j,j\ne l\\
        &=\sigma_j-\Delta c_la_{lj}^{\prime}
    \end{align*}
    (注意:上式中a的l与左边基变量的下标对应)
    为保证最优解不变,$\Delta c_l$要满足$$\max\left\{ \frac{\sigma_j}{a_{ij}^{\prime}}|a_{ij}^{\prime}>0 \right\}\le\Delta c_l\le\min\left\{ \frac{\sigma_j}{a_{ij}^{\prime}}|a_{ij}^{\prime}<0 \right\}$$若$\Delta c_l$超出此范围,应求出新的检验数$\bar{\sigma}_j$,选择其中大于零的检验数对应的变量$x_j$为进基变量,继续迭代。
\end{enumerate}

\subsubsection{右端常数b变化$\star$}

$b_r$的变化影响解的可行性,但不影响检验数的符号变化。由$\boldsymbol{x_B}=\boldsymbol{B}^{-1}\boldsymbol{b}$可知$b_r$的变化必会引起最优解数值变化。

最优解的变化分为以下两类:
\begin{enumerate}
    \item 保持$\boldsymbol{B}^{-1}\boldsymbol{b}\ge\boldsymbol{0}$,即最优基$\boldsymbol{B}$不变(影子价格不变,也就是对偶问题的最优解不变)\\
    只需要将变化后的$b_r$带入$\boldsymbol{B}^{-1}\boldsymbol{b}$的表达式重新计算即可
    \item $\boldsymbol{B}^{-1}\boldsymbol{b}$出现负分量,这使最优基$\boldsymbol{B}$变化\\
    需要通过迭代求解新的最优基和最优解
\end{enumerate}
综合一下,可以利用下面的步骤计算:\\
设$b_r\rightarrow\bar{b}_r=b_r+\Delta b_r$,$\Delta b_r$,此时有$$
\boldsymbol{x_B}\rightarrow\boldsymbol{\bar{x}_B}=\boldsymbol{B}^{-1}\begin{pmatrix}
    b_1\\ \vdots\\ b_r+\Delta b_r\\ \vdots\\ b_m
\end{pmatrix}=\boldsymbol{B}^{-1}\boldsymbol{b}+\boldsymbol{B}^{-1}\begin{pmatrix}
    0\\ \vdots\\ \Delta b_r\\ \vdots\\ 0
\end{pmatrix}=\boldsymbol{x_B}+\Delta b_r\begin{pmatrix}
    \beta_{1r}\\ \vdots\\ \beta_{mr}
\end{pmatrix}=\begin{pmatrix}
    b_1^{\prime}\\ \vdots\\ b_m^{\prime}
\end{pmatrix}+\Delta b_r\begin{pmatrix}
    \beta_{1r}\\ \vdots\\ \beta_{mr}
\end{pmatrix}\ge\begin{pmatrix}
    0\\ \vdots\\ 0
\end{pmatrix}
$$
(上式只需要随便算其中的一个,然后看是否大于等于0就行,不需要特别算下面的$\Delta b_r$)

其中,为$\boldsymbol{x_B}^{-1}\boldsymbol{b}$原最优解,$b_i^{\prime}$为$\boldsymbol{x_B}=\boldsymbol{B}^{-1}\boldsymbol{b}$的第$i$个分量,$\beta_{ir}$为$\boldsymbol{B}^{-1}$的第$i$行第$r$列元素

要满足$$\max\left\{ \frac{-b_i^{\prime}}{\beta_{ir}}|\beta_{ir}>0 \right\}\le\Delta b_r\le\min\left\{ \frac{-b_i^{\prime}}{\beta_{ir}}|\beta_{ir}<0 \right\}$$当$\Delta b_r$超过此范围时,将使最优解中某个分量小于零,使最优基发生变化。此时可用对偶单纯形法继续迭代新的最优解。

就是说,如果没超范围,直接把$\boldsymbol{\bar{x_B}}$当最优解得出就行;如果超范围的话,就把$\boldsymbol{\bar{x_B}}$放到最优单纯形表的$b$列,然后求对偶单纯形。

\subsubsection{约束条件系数a变化}
假设只有一个$a_{ij}$变化,其他数据不变,且只讨论$a_{ij}$为非基变量$x_j$的系数的情况。那么此时$a_{ij}$的变化只影响一个检验数$\sigma_j$。

设$a_{ij}\rightarrow a_{ij}+\Delta a_{ij}$,由检验数的另一种表示形式$$\sigma_j\rightarrow\bar{\sigma}_j=c_j-\boldsymbol{y^T}
\begin{pmatrix}a_{1j}\\ \vdots\\ a_{ij}+\Delta a_{ij}\\ \vdots\\ a_{mj}\end{pmatrix}=c_j-\boldsymbol{y^Tp}_j-\boldsymbol{y^T}\begin{pmatrix}0\\ \vdots\\ \Delta a_{ij}\\ \vdots\\ 0 \end{pmatrix}=\sigma_j-y_i^*\Delta a_{ij}$$其中,$\boldsymbol{y}$为对偶最优解,$y_i^*$为$\boldsymbol{y}$的第$i$个变量

为使最优解不变,要使$\sigma_j\le0$,即\begin{align*}
    \sigma_j\le y_i^*\Delta a_{ij}\\
    \Delta a_{ij}\ge\frac{\sigma_j}{y_i^*},y^*_i>0\\
    \Delta a_{ij}\le\frac{\sigma_j}{y^*_i},y^*_i<0
\end{align*}

\subsubsection{新增变量x分析}

增加变量$x_{n+1}$,则有相应的约束条件$\boldsymbol{p}_{n+1}$,目标函数系数$c_{n+1}$,那么,计算出$$\sigma_{n+1}=c_{n+1}-\boldsymbol{c_B}^T\boldsymbol{B}^{-1}\boldsymbol{p}_{n+1}$$
填入最优单纯形表,若 $\sigma_{n+1} \le 0$,则最优解不变;否则,进一步用单纯形法求解

\subsubsection{新增约束条件}

增加一个约束之后,应把最优解代入新的约束,若满足,则最优解不变;否则,填入最优单纯形表作为新的一行,引入一个新的非负变量(原约束若是小于等于形
式,可引入非负松弛变量;否则,引入非负人工变量),并通过矩阵行变换把对应基
变量的元素变为0,进一步用单纯形法或对偶单纯形法求解。

\end{document}
