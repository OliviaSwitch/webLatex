
\documentclass{book}

\title{OR}
\author{S.Olivia}
\date{September 2023}

\usepackage{graphicx} % Required for inserting images
\usepackage{geometry}
%\usepackage{ctex}
\usepackage[UTF8]{ctex}
\usepackage{amsmath,amssymb,amsfonts}
\newtheorem{theorem}{定理}[chapter]
\newtheorem{definition}{\indent 定义}[chapter]

\usepackage{palatino}
\usepackage{tikz}
\usetikzlibrary{patterns}
\usepackage{pgfplots}
\usepackage{multirow}
\usepackage{newtxmath}
\usepackage{hyperref}

\geometry{a4paper,scale=0.8}

\usetikzlibrary{shapes.geometric, arrows}

\newcommand{\subsubsubsection}[1]{\paragraph{#1}\mbox{}\\}
\setcounter{secnumdepth}{4} % how many sectioning levels to assign numbers to
\setcounter{tocdepth}{4} % how many sectioning levels to show in ToC

\hypersetup{
colorlinks=true,
linkcolor=blue,
}

\begin{document}

\maketitle

\tableofcontents

\chapter{Introduction}
\section{数学基础}
\subsection{梯度与黑塞矩阵}
简便记法

\subsubsection{梯度}

\begin{equation}
    \nabla f(x_1, x_2, x_3) = (\frac{\partial f}{\partial x_1}, \frac{\partial f}{\partial x_2}, \frac{\partial f}{\partial x_3})^T
    \label{con:grad}
\end{equation}

\subsubsection{黑塞矩阵}

\begin{equation}
    \nabla^2 f(x_1, x_2, x_3)=
    \begin{pmatrix}
        \cfrac{\partial^2 f}{\partial x_1^2}           & \cfrac{\partial^2 f}{\partial x_2\partial x_1}  & \cfrac{\partial^2 f}{\partial x_3\partial x_1} \\
        \cfrac{\partial^2 f}{\partial x_1\partial x_2} & \cfrac{\partial^2 f}{\partial x_2^2}            & \cfrac{\partial^2 f}{\partial x_3\partial x_2} \\
        \cfrac{\partial^2 f}{\partial x_1\partial x_3} & \cfrac{\partial^2 f}{\partial x_2 \partial x_3} & \cfrac{\partial^2 f}
        {\partial x_3^2}
        \label{con:Hesse matrix}
    \end{pmatrix}
\end{equation}
简而言之,式(\ref{con:Hesse matrix})即为式(\ref{con:grad})逐行关于参数(即$x_1, x_2, x_3, \cdots$)求偏导。

\subsection{正定矩阵}

对一个 $n$ 阶方阵 $\boldsymbol{M}$,如果其对任意非零向量 $\boldsymbol{Z}$ 都有 $\boldsymbol{Z}^T\boldsymbol{MZ}>0$,其中$\boldsymbol{Z}^T$ 为 $\boldsymbol{Z}$ 的转置,则 $\boldsymbol{M}$为正定矩阵。
\begin{theorem}
    对于 $n$ 阶实对称矩阵$\boldsymbol{M}$,下列条件是等价的:
    \begin{enumerate}
        \item $\boldsymbol{M}$ 是正定矩阵。
        \item $\boldsymbol{M}$ 的特征值均为正。
        \item $\boldsymbol{M}$ 的一切顺序主子式均为正。
    \end{enumerate}
\end{theorem}

\section{基本理论}
\subsection{数学规划模型的一般形式}

\begin{equation}
    (fS) = \begin{cases}
        min  & f(x)    \\
        s.t. & x \in S
    \end{cases}
    \label{con:general form}
\end{equation}

\subsection{凸集、凸函数和凸规划}
\subsubsection{凸集}

\begin{definition}
    设$S \subset \mathbf{R}^n$,如果$\boldsymbol{x}^{(1)}, \boldsymbol{x}^{(2)} \in S, \lambda \in [0, 1]$,均有
    \begin{equation}
        \boldsymbol{\lambda x}^{(1)}+(1-\boldsymbol{\lambda})\boldsymbol{x}^{(2)} \in S
    \end{equation}
    则称$S$为凸集
\end{definition}

集合中任意两点连成的线段必属于该集合;规定空集$\varnothing$为凸集,单点集$\{x\}$为凸集。

性质:
\begin{enumerate}
    \item 凸集的交集是凸集
    \item 凸集的内点集是凸集
    \item 凸集的闭包是凸集
    \item 分离和支撑:凸集边界上任意点存在支撑超平面;两个相互不交的凸集之间存在分离超平面
\end{enumerate}

有一特殊的凸集:凸锥

\begin{definition}
    设非空集合$C\subset\mathbf{R}^n$,如果$\forall\boldsymbol{x}\in C$对$\forall\lambda>0$有$\boldsymbol{\lambda x} \in C$,则称$C$为以0为顶点的锥(不一定含$0$点)。若$C$又是凸集,则称$C$为凸锥。
\end{definition}

\subsubsection{凸函数}
\begin{definition}
    设$S\subset \mathbf{R}^n$,非空,凸集,函数$f:S\rightarrow \mathbf{R}$,如果对$\forall x^{(1)}, x^{(2)}\in S, \forall\lambda\in (0,1)$恒有
    \begin{equation}
        f(\boldsymbol{\lambda x}^{(1)}+(1-\boldsymbol{\lambda})\boldsymbol{x}^{(2)})\le\boldsymbol{\lambda}f(\boldsymbol{x}^{(1)})+(1-\boldsymbol{\lambda})f(\boldsymbol{x}^{(2)})
        \label{con:convex function}
    \end{equation}
    则称$f$为$S$上的凸函数。如果式(\ref{con:convex function})恒以严格不等式成立,则称$f$为$S$上的严格凸函数。
\end{definition}

几何意义:任意两个点连线在函数曲线的上方。

\subsubsubsection{水平集}
\begin{definition}
    设$S\subset\mathbf{R}^n$,非空,$f:S\rightarrow\mathbf{R}, a\in \mathbf{R},$则称
    \begin{equation}
        S_{\alpha}=\{\boldsymbol{x}|f(\boldsymbol{x)}\le\alpha,\boldsymbol{x}\in S\}
        \label{con:level set}
    \end{equation}
    为$f$的水平集。
\end{definition}

水平集的概念相当于在地形图中,海拔高度不高于某一数值的区域。

注意:容易证明,当$f$为凸函数时,$\forall\alpha\in\mathbf{R}, S_{\alpha}$是凸集。但是它的逆不成立。

\subsubsubsection{凸函数的性质}

\begin{enumerate}
    \item $f(x)$为凸集$S$上的凸函数$\Leftrightarrow S$上任意有限点的凸组合的函数值不大于各点函数值的凸组合
    \item 设$f_1,f_2$为凸函数,$\lambda_1,\lambda_2>0$,则:
          \begin{itemize}
              \item $\lambda_1f_1+\lambda_2f_2$是凸函数
              \item $f(x)=\max\{f_1(x),f_2(x)\}$是凸函数
              \item $g(x)=\min\{f_1(x),f_2(x)\}$不一定是凸函数
          \end{itemize}
    \item 若$f$在$S$上凸,那么$f$在$S$的内点集($intS$)上连续(注:$f$在$S$上不一定连续)
    \item 若$f$在非空凸集$S$上凸,则对任意方向的方向导数存在

          \textbf{凸函数常用判定条件:}
    \item 设$S$非空,凸集,开集,$f$在$S$上可微,则:
          \begin{itemize}
              \item $f$在$S$上凸$\rightleftharpoons \forall \bar{x}\in S$,有$f(x)\ge f(\bar{x})+\nabla f^T(\bar{x})(x-\bar{x}), \forall x\in S$。
              \item $f$在$S$上严格凸$\rightleftharpoons \forall \bar{x}\in S$,有$f(x)> f(\bar{x})+\nabla f^T(\bar{x})(x-\bar{x}), \forall x\in S,x\ne\bar{x}$。
          \end{itemize}
    \item 设$S$非空,凸集,开集,$f$在$S$上可微,则:
          \begin{itemize}
              \item $f$在$S$上凸$\rightleftharpoons \forall x^{(1)},x^{(2)}\in S,(\nabla f(x^{(1)})-\nabla f(x^{(2)}))^T(x^{(1)}-x^{(2)})\ge 0$
              \item $f$在$S$上严格凸$\rightleftharpoons \forall x^{(1)},x^{(2)}\in S, x^{(1)}\ne x^{(2)},(\nabla f(x^{(1)})-\nabla f(x^{(2)}))^T(x^{(1)}-x^{(2)})> 0$
          \end{itemize}
    \item 设$S$是开集,$f$在$S$上二次可微,则:
          \begin{itemize}
              \item $f$在$S$上凸$\rightleftharpoons \forall x\in S, \nabla^2f(x)$半正定。
              \item 如果$\forall x \in S, \nabla^2f(x)$正定,则$f$在$S$上严格凸。
          \end{itemize}
\end{enumerate}

但是,上课讲的时候,一般是使用\textbf{黑塞矩阵是否正定}判断
\begin{enumerate}
    \item 当$H$为半正定时,$f$为凸函数;
          若$H$是正半定的,当且仅当$H$的每一个主子式都大于等于0
    \item 当$H$为正定时,$f$为严格凸函数;
          若$H$是正定矩阵,当且仅当$H$的$n$个顺序主子式(严格)为正。
    \item 当$H$为半负定时,$f$为凹函数;
          若$H$是负半定的,当且仅当$H$的每一个奇阶的主子式小于等于0,每一个偶阶的主子式大于等于0
    \item 当$H$为负定时,$f$为严格凹函数;
          若$H$是负定矩阵,当且仅当$H$的$N$个顺序主子式以如下方式交替出现:
          $|A_1|<0, |A_2|>0, |A_3|<0, \cdots$
    \item 当$H$为不定时,$f$既非凸也非凹函数。
\end{enumerate}
(!注意:只能一个方向判断,不能由是否凸函数来判断矩阵是否正定)

具体可看:
\href{https://zhuanlan.zhihu.com/p/594455595}{数学选读01:矩阵的主子式与顺序主子式}

\subsubsection{凸规划}

\begin{definition}
    有两种情况
    \begin{itemize}
        \item 当$(fS)$中,$S$为凸集,$f$是$S$上的凸函数(求$\min$时),称$(fS)$为凸规划。
        \item 对于$(fgh)$,当$f, g_i$为凸函数,$h_i$为线性函数时,$(fgh)$为凸规划。
    \end{itemize}
\end{definition}

\begin{theorem}
    设$S\subset \mathbf{R}^n$,非空,凸,$f:S\rightarrow\mathbf{R}$是凸函数。$x^*$为问题$(fS)$的l.opt.,则$x^*$为g.opt.;又如果$f$是严格凸函数,那么$x^*$是问题$(fS)$的唯一g.opt.。
\end{theorem}

\subsection{多面体、极点、极方向}
(看书上p29开始的图更好理解)

多面体:有限个半闭空间的交为多面体

极点:$x\in S$,不存在$S$中另外两个点$x^{(1)}$和$x^{(2)}$,及$\lambda\in(0,1)$,使$x=\lambda x^{(1)}+(1-\lambda)x^{(2)}$\\
根据定义,闭球体的表面上每一点都是极点;一般的闭凸锥有唯一极点,即顶点;平面没有极点;

极方向:
\begin{itemize}
    \item 方向:$d\in \mathbf{R}^n, d\ne 0$及对于任意$x\in S,\lambda>0$,总有$x+\lambda d\in S$(可行方向)。其中,当$d^{(1)}=\lambda d^{(2)}(\lambda>0)$时,称$d^{(1)}$和$d^{(2)}$同方向。
    \item 极方向:方向$d$不能表示为两个不同方向的组合$(d=d^{(1)}+d^{(2)})$。
\end{itemize}

\begin{theorem}[极点特征定理]
    设$A$行满秩,$\boldsymbol{x}$是$S$极点的充要条件是:\\
    存在分解$\boldsymbol{A}=(\boldsymbol{B},\boldsymbol{N})$,其中$\boldsymbol{B}$为$m$阶非奇异矩阵,使$\boldsymbol{x}^T=(\boldsymbol{x_B}^T,\boldsymbol{x_N}^T)$,这里$\boldsymbol{x_B}=\boldsymbol{B}^{-1}b\ge0,\boldsymbol{x_N}=0$
\end{theorem}

$S$中必定存在有限多个极点$(\le C_n^m)$

\chapter{Linear Programming Problem}

\section{Graphical Method}
图解法

具体可看ppt或:
\href{https://blog.csdn.net/shulianghan/article/details/102671536}{【运筹学】线性规划 图解法 ( 唯一最优解 | 无穷最优解 | 无界解 | 无可行解 )}

基本解:各个等式约束直线的交点,外加与坐标轴的交点

基本可行解:基本解里面在可行域范围的那些基本解,可行域的顶点

最优解:基本可行解里面使目标函数最大(最小)的基本可行解


\section{Simplex Method}

一些必要概念:
(建议先了解解题步骤,再回过头来看概念)

\textbf{线性规划的标准形式}
\begin{equation}
    (LP)
    \begin{cases}
        \max\  & \boldsymbol{c}^T\boldsymbol{x}  \\
        s.t.   & \boldsymbol{Ax}=\boldsymbol{b}  \\
               & \boldsymbol{x}\ge\boldsymbol{0}
    \end{cases}
    \label{con:standard form of LP}
\end{equation}

其中,$\boldsymbol{A}$是$m\times n$的矩阵$(m<n)$
$$\boldsymbol{A}=
    \begin{pmatrix}
        a_{11} & \cdots & a_{1m} & a_{1(m+1)} & \cdots & a_{1n} \\
        \vdots &        & \vdots & \vdots     &        & \vdots \\
        a_{m1} & \cdots & a_{mm} & a_{m(m+1)} & \cdots & a_{mn}
    \end{pmatrix}
$$
\textbf{可行解}:
在公式(\ref{con:standard form of LP})中,$\boldsymbol{x}=(x_1, \cdots, x_n)^T$即为\textbf{可行解},可以理解为多面体的极点(顶点?)\\
\textbf{基本可行解}:可以理解为满足非负约束条件的可行解?(非负约束条件大概就是公式(\ref{con:standard form of LP})中$\boldsymbol{x\ge0}$)\\
\textbf{基}:$\boldsymbol{B}$是线性规划问题的一个基
$$\boldsymbol{B}=
    \begin{pmatrix}
        a_{11} & \cdots & a_{1m} \\
        \vdots & \ddots & \vdots \\
        a_{m1} & \cdots & a_{mm}
    \end{pmatrix}$$
$\boldsymbol{B}$中的每个列向量$\boldsymbol{p}_j=(a_{1j}, a_{2j}, \cdots, a_{mj})^T$为\textbf{基向量},基向量中的$a_{mj}$们即为\textbf{基变量}。
一般的,我们讲基变量为$m$个线性无关的变量。(一般是标准化后的松弛变量所对应的列们的系数,也是\textbf{单位矩阵})

\begin{figure}[ht]
    \centering
    % texlive2015, pdflatex



%\begin{document}
\thispagestyle{empty}
% 流程图定义基本形状
\tikzstyle{startstop} = [rectangle, rounded corners, minimum width = 2cm, minimum height=1cm,text centered, draw = black, align=center]
%\tikzstyle{io} = [trapezium, trapezium left angle=70, trapezium right angle=110, minimum width=2cm, minimum height=1cm, text centered, draw=black]
\tikzstyle{process} = [rectangle, minimum width=3cm, minimum height=1cm, text centered, draw=black]
\tikzstyle{decision} = [diamond, aspect = 3, text centered, draw=black]
\tikzstyle{point}=[coordinate,]
% 箭头形式
\tikzstyle{arrow} = [->,>=stealth]

\begin{tikzpicture}[node distance=2cm]
%定义流程图具体形状
\node[startstop](start){确定初始基本可行解(极点)\\ 即单位矩阵$x$};
\node[process, below of = start](cul1){计算
$\boldsymbol{\sigma_N^T=c_N^T-c_B^TB^{-1}N}$};
\node[decision, below of = cul1](dec1)
{所有$\boldsymbol{\sigma_N\le 0}?$};
\node[process, below of = dec1](cul2){计算
$\boldsymbol{\sigma_k}=max\{\sigma_j|\sigma_j>0\}$};
\node[decision, below of = cul2](dec2){$
\boldsymbol{B^{-1}p_k\le0}?$};
\node[process, below of = dec2](cul3){计算
$\theta=min\left\{\frac{(\boldsymbol{B^{-1}b)_{i}}}{\boldsymbol{B^{-1}p_k)}_{i}}|\boldsymbol{(B^{-1}p_k)}_{i}>0\right\}=\frac{(\boldsymbol{B^{-1}b)_{r}}}{\boldsymbol{B^{-1}p_k)}_{r}}$};
\node[process, below of = cul3](end){使$x_k$进基,$x_r$出基进行迭代};
\node[startstop, right of = dec1, xshift=3cm](stop1){停,x-opt};
\node[startstop, right of = dec2, xshift=3.5cm](stop2){停,无有限最优解};
\node[point, left of = end, node distance=5cm](point1){};
\node[point, above of = cul1, yshift=-1cm](point2){};

%连接具体形状
\draw [arrow] (start) -- (cul1);
\draw [arrow] (cul1) -- (dec1);
\draw [arrow] (dec1) -- node[right]{N} (cul2);
\draw [arrow] (cul2) -- (dec2);
\draw [arrow] (dec2) -- node[right]{N} (cul3);
\draw [arrow] (cul3) -- (end);
\draw [arrow] (dec1) -- node[above]{Y} (stop1);
\draw [arrow] (dec2) -- node[above]{Y} (stop2);
\draw [-](end) -- (point1);
\draw [arrow](point1) |- node{}(point2);

%\draw (dec1) -- node [above] {Y} (point1);
%\draw [arrow] (point1) |- (pro1);
%\draw [arrow] (dec1) -- (dec1|-pro1) -> (pro1);
%\draw [arrow] (dec1) -- node [right] {N} (pro2);
\end{tikzpicture}

%\end{document}
    \caption{单纯形法流程图}
    \label{fig:SimplexMethodFlow}
\end{figure}


需要注意的是,$\sigma$在表中的计算没有那么复杂,可以先略过。

\subsection{单纯形法的表格计算}

考虑规范形式的线性规划问题:$b_i>0,i=1,\cdots,m$
$$
    \begin{aligned}
         & \max\quad z=c_1x_1+c_2x_2+\cdots+c_nx_n & \\
         & s.t.\quad
        \begin{cases}
            a_{11}x_1+a_{12}x_2+\cdots+a_{1n}x_n & \le b_1 \\
            a_{21}x_1+a_{22}x_2+\cdots+a_{2n}x_n & \le b_2 \\
            \quad\vdots                                    \\
            a_{m1}x_1+a_{m2}x_2+\cdots+a_{mn}x_n & \le b_m \\
            x_1,x_2,\cdots,x_n\ge0
        \end{cases}
    \end{aligned}
$$
加入松弛变量,化为标准形:
\begin{equation}
    \begin{cases}
        \max\quad & z=c_1x_1+c_2x2+\cdots+c_nx_n                                \\
        s.t.\quad & a_{11}x_1+a_{12}x_2+\cdots+a_{1n}x_n +x_{n+1}=b_1           \\
                  & a_{21}x_1+a_{22}x_2+\cdots+a_{2n}x_n \quad+x_{n+2}=b_2      \\
                  & \quad\vdots                                                 \\
                  & a_{m1}x_1+a_{m2}x_2+\cdots+a_{mn}x_n \quad\quad+x_{n+m}=b_m \\
                  & x_1,x_2,\cdots,x_n,x_{n+1},\cdots,x_{n+m}\ge0
    \end{cases}
    \label{con: eq2.2}
\end{equation}

STEPS:
\begin{enumerate}
    \item 根据公式(\ref{con: eq2.2})构造初始单纯形表:\\

          \begin{table}[ht]
              \centering
              \begin{tabular}{c|c|c|cccccccc|c}
                  \hline
                  \multirow{2}{*}{$\boldsymbol{c_B}$} & \multirow{2}{*}{$\boldsymbol{x_B}$} & \multirow{2}{*}{$\boldsymbol{b}$} & $c_1$      & $c_2$    & $\cdots$   & $c_n$    & $c_{n+1}$ & $c_{n+2}$ & $\cdots$ & $c_{n+m}$ & \multirow{2}{*}{$\boldsymbol{\theta}$} \\
                  \cline{4-11}
                                                      &                                     &                                   & $x_1$      & $x_2$    & $\cdots$   & $x_n$    & $x_{n+1}$ & $x_{n+2}$ & $\cdots$ & $x_{n+m}$ &                                        \\
                  \hline
                  $c_{n+1}$                           & $x_{n+1}$                           & $b_1$                             & $a_{11}$   & $a_{12}$ & $\cdots$   & $a_{1n}$ & 1         & 0         & $\cdots$ & 0         & $\theta_1$                             \\
                  $c_{n+2}$                           & $x_{n+2}$                           & $b_2$                             & $a_{21}$   & $a_{22}$ & $\cdots$   & $a_{2n}$ & 0         & 1         & $\cdots$ & 0         & $\theta_2$                             \\
                  $\vdots$                            & $\vdots$                            & $\vdots$                          & $\vdots$   & $\vdots$ &            & $\vdots$ & $\vdots$  & $\vdots$  &          & $\vdots$  & $\vdots$                               \\
                  $c_{n+m}$                           & $x_{n+m}$                           & $b_m$                             & $a_{m1}$   & $a_{m2}$ & $\cdots$   & $a_{mn}$ & 0         & 0         & $\cdots$ & 1         & $\theta_m$                             \\
                  \hline
                  \multicolumn{2}{c|}{$-z$}           & $-z^{\prime}$                       & $\sigma_1$                        & $\sigma_2$ & $\cdots$ & $\sigma_n$ & 0        & 0         & $\cdots$  & 0        &                                                    \\
                  \hline
              \end{tabular}
              \caption{单纯形表}
              \label{tab:label de SM}
          \end{table}
          表(\ref{tab:label de SM})中;
          \begin{itemize}
              \item 列$\boldsymbol{c_B}$填入目标函数中基变量$x$的系数
              \item 列$\boldsymbol{x_B}$填入基变量
              \item 列$\boldsymbol{b}$填入约束方程右端的常数
              \item 中间4-11 列填入约束方程中$x$的系数,其中从$x_{n+1}$开始的列组成单位矩阵
          \end{itemize}
    \item 求出$-z^{\prime}=-\sum\limits_{i=1}^mc_{n+i}b_i$(初始状态下一般为0)
    \item 求出\textbf{检验数}$\boldsymbol{\sigma}_j=c_j-\sum\limits_{i=1}^mc_{n+i}a_{ij}$,可以通俗理解为$\sigma_1=$第一行中的$c_1$减去($c_B$的每一项与$c_1$列对应行的$a_{\_1}$相乘的求和)。\\
          需要注意的是,这里的检验数不需要考虑基(即单位矩阵)对应的列,否则会影响下一步。
    \item 判断是否所有检验数$\sigma_j\le0$,如果全部小于等于0,则当前的基本可行解是最优解;如果有一检验数大于0,那么进行下一步的计算。
    \item 求入基变量$x_k$:
          \begin{itemize}
              \item 先求出大于0的检验数中最大的\textbf{检验数}$\boldsymbol{\sigma}_k$
              \item 得到的下标$k$,$\boldsymbol{\sigma}_k$所在的列就是主元列
              \item 那么所对应的$x_k$就是入基变量
          \end{itemize}
    \item 求出基变量$x_r$:
          \begin{itemize}
              \item 找出主元列($x_k$列)对应元素$a_{ik}>0$的
              \item 使每个大于0的$a_{ik}$被对应行中的$b_i$除,得到$\boldsymbol{\theta}_i=\frac{b_i}{a_{ik}}$\\
                    (若$a_{ik}\le0$,则$\boldsymbol{\theta}_i=\infty$)
              \item 找出$\min\{\boldsymbol{\theta}_i\}=\boldsymbol{\theta}_r$
              \item 此时$\boldsymbol{\theta}_r$所在的行$r$即为主元行,$x_r$为出基变量
          \end{itemize}
    \item 求出入基变量$x_k$和出基变量$x_r$后,就可以构建下一张表,这张表中:
          \begin{itemize}
              \item $\boldsymbol{x_B}$上的出基变量被入基变量所取代,同时需要更改列$\boldsymbol{c_B}$中与之相对应的行的元素
              \item 在表右侧中间部分,使用高斯消元法将$x_k$对应的$a$与其他基变量对应的$a$组成一个单位矩阵,同时列$\boldsymbol{b}$也跟着改变
              \item 表2基本构建完成,重复第二步,直至出现流程表中stop的情况
          \end{itemize}
\end{enumerate}

\subsection{一般线性规划问题的处理}

在某些情况下,如基本初始可行解不明显,即很难在标准形的问题下找到单位矩阵时,可以考虑使用\textbf{大M法}与\textbf{二阶段法}

\subsubsection{大M法}
STEPS:
\begin{enumerate}
    \item 将线性规划问题转化为标准型
    \item 观察变量,若初始基本可行解明显,直接进行单纯形法;否则引入人工变量 $x_{n+i} \ge 0 (i = 1 , … ,m)$及\textbf{充分大正数}$M$,改写原目标函数,进行单纯形法
    \item 若得到的最优解满足: $$
              x_{n+i} = 0,(i = 1 , … , m)
          $$则是原问题的最优解;否则,原问题无可行解
\end{enumerate}

例:使用大M法求解下面的问题
$$
    \begin{aligned}
         & max\quad z=5x_1+2x_2+3x_3-x_4 & \\
         & s.t.\quad
        \begin{cases}
            x_1+2x_2+3x_3     & = 15 \\
            2x_1+x_2+5x_3     & = 20 \\
            x_1+2x_2+4x_3+x_4 & = 26 \\
            x_1,x_2,x_3,x_4\ge0
        \end{cases}
    \end{aligned}
$$
标准化并引入人工变量:
$$
    \begin{aligned}
         & max\quad z=5x_1+2x_2+3x_3-x_4-Mx_5-Mx_6 & \\
         & s.t.\quad
        \begin{cases}
            x_1+2x_2+3x_3\qquad+x_5       & = 15 \\
            2x_1+x_2+5x_3\qquad\qquad+x_6 & = 20 \\
            x_1+2x_2+4x_3+x_4             & = 26 \\
            x_1,x_2,x_3,x_4,x_5,x_6\ge0
        \end{cases}
    \end{aligned}
$$

最后使用单纯形法计算
\begin{table}[ht]
    \centering
    \begin{tabular}{c|c|c|cccccc|c}
        \hline
        \multirow{2}{*}{$\boldsymbol{c_B}$} & \multirow{2}{*}{$\boldsymbol{x_B}$} & \multirow{2}{*}{$\boldsymbol{b}$} & 5           & 2      & 3     & -1      & $-M$      & $-M$  & \multirow{2}{*}{$\boldsymbol{\theta}$} \\
        \cline{4-9}
                                            &                                     &                                   & $x_1$       & $x_2$  & $x_3$ & $x_4$   & $x_5$     & $x_6$ &                                        \\
        \hline
        $-M$                                & $x_5$                               & 15                                & 1           & 2      & 3     & 0       & 1         & 0     & 5                                      \\
        $-M$                                & $x_6$                               & 20                                & 2           & 1      & [5]   & 0       & 0         & 1     & 4                                      \\
        -1                                  & $x_4$                               & 26                                & 1           & 2      & 4     & 1       & 0         & 0     & 6.5                                    \\
        \hline
        \multicolumn{2}{c|}{$-z$}           & $35M+26$                            & $3M+6$                            & $3M+4$      & $8M+7$ & 0     & 0       & 0         &                                                \\
        \hline
        $-M$                                & $x_5$                               & 3                                 & -1/5        & [7/5]  & 0     & 0       & 1         & -3/5  & 15/7                                   \\
        3                                   & $x_3$                               & 4                                 & 2/5         & 1/5    & 1     & 0       & 0         & 1/5   & 20                                     \\
        -1                                  & $x_4$                               & 10                                & -3/5        & 6/5    & 0     & 1       & 0         & -4/5  & 25/3                                   \\
        \hline
        \multicolumn{2}{c|}{$-z$}           & $3M-2$                              & $-M/5+16/5$                       & $7/5M+13/5$ & 0      & 0     & 0       & -8/5M-7/5 &                                                \\
        \hline
        2                                   & $x_2$                               & 15/7                              & -1/7        & 1      & 0     & 0       & 5/7       & -3/7  &                                        \\
        3                                   & $x_3$                               & 25/7                              & [3/7]       & 0      & 1     & 0       & -1/7      & 2/7   & 25/3                                   \\
        -1                                  & $x_4$                               & 52/7                              & -3/7        & 0      & 0     & 1       & -6/7      & -2/7  &                                        \\
        \hline
        \multicolumn{2}{c|}{$-z$}           & -53/7                               & 25/7                              & 0           & 0      & 0     & -M-13/7 & -M-2/7    &                                                \\
        \hline
        2                                   & $x_2$                               & 10/3                              & 0           & 1      & 1/3   & 0       & 2/3       & -1/3  &                                        \\
        5                                   & $x_1$                               & 25/3                              & 1           & 0      & 7/3   & 0       & -1/3      & 2/3   &                                        \\
        -1                                  & $x_4$                               & 11                                & 0           & 0      & 1     & 1       & -1        & -0    &                                        \\
        \hline
        \multicolumn{2}{c|}{$-z$}           & -112/3                              & 0                                 & 0           & -25/3  & 0     & -M-2/3  & -M+8/3    &                                                \\
        \hline
    \end{tabular}
    \caption{大M法例题}
    \label{tab:example de big M method}
\end{table}

得到基本可行解:$(25/3,10/3,0,11)^T$为最优解

得到最优值$z=112/3$

\subsubsection{二阶段法}
STEPS:
\begin{enumerate}
    \item 将线性规划问题转化为标准型
    \item 观察变量,若初始基本可行解明显,直接进行单纯形法;
          否则引入人工变量 $x_{n+i} \ge 0 (i= 1 , … , m)$构造辅助问题(LP - 1)
    \item 第一阶段,求解辅助问题(LP - 1),若得到的最优解满足 $x_{n+i} = 0 (i= 1 , … , m)$,则是原问题的基本可行解;\textbf{否则,原问题无可行解}。
    \item 第二阶段,得到原问题的基本可行解后,直接删除人工变量,求解原问题
\end{enumerate}

例:使用二阶段法求解下面的问题
$$
    \begin{aligned}
         & max\quad z=5x_1+2x_2+3x_3-x_4 & \\
         & s.t.\quad
        \begin{cases}
            x_1+2x_2+3x_3     & = 15 \\
            2x_1+x_2+5x_3     & = 20 \\
            x_1+2x_2+4x_3+x_4 & = 26 \\
            x_1,x_2,x_3,x_4\ge0
        \end{cases}
    \end{aligned}
$$
标准化、引入人工变量并构造第一阶段问题(LP - 1):
$$
    \begin{aligned}
         & max\quad z^{\prime}=-x_5-x_6 & \\
         & s.t.\quad
        \begin{cases}
            x_1+2x_2+3x_3\qquad+x_5       & = 15 \\
            2x_1+x_2+5x_3\qquad\qquad+x_6 & = 20 \\
            x_1+2x_2+4x_3+x_4             & = 26 \\
            x_1,x_2,x_3,x_4,x_5,x_6\ge0
        \end{cases}
    \end{aligned}
$$

建立第一阶段的单纯形表:
\begin{table}[ht]
    \centering
    \begin{tabular}{c|c|c|cccccc|c}
        \hline
        \multirow{2}{*}{$\boldsymbol{c_B}$} & \multirow{2}{*}{$\boldsymbol{x_B}$} & \multirow{2}{*}{$\boldsymbol{b}$} & 0     & 0     & 0     & 0     & -1    & -1    & \multirow{2}{*}{$\boldsymbol{\theta}$} \\
        \cline{4-9}
                                            &                                     &                                   & $x_1$ & $x_2$ & $x_3$ & $x_4$ & $x_5$ & $x_6$ &                                        \\
        \hline
        -1                                  & $x_5$                               & 15                                & 1     & 2     & 3     & 0     & 1     & 0     & 5                                      \\
        -1                                  & $x_6$                               & 20                                & 2     & 1     & [5]   & 0     & 0     & 1     & 4                                      \\
        0                                   & $x_4$                               & 26                                & 1     & 2     & 4     & 1     & 0     & 0     & 6.5                                    \\
        \hline
        \multicolumn{2}{c|}{$-z$}           & 35                                  & 3                                 & 3     & 8     & 0     & 0     & 0     &                                                \\
        \hline
        -1                                  & $x_5$                               & 3                                 & -1/5  & [7/5] & 0     & 0     & 1     & -3/5  & 15/7                                   \\
        0                                   & $x_3$                               & 4                                 & 2/5   & 1/5   & 1     & 0     & 0     & 1/5   & 20                                     \\
        0                                   & $x_4$                               & 10                                & -3/5  & 6/5   & 0     & 1     & 0     & -4/5  & 25/3                                   \\
        \hline
        \multicolumn{2}{c|}{$-z$}           & 3                                   & -1/5                              & 7/5   & 0     & 0     & 0     & -8/5  &                                                \\
        \hline
        2                                   & $x_2$                               & 15/7                              & -1/7  & 1     & 0     & 0     & 5/7   & -3/7  &                                        \\
        3                                   & $x_3$                               & 25/7                              & [3/7] & 0     & 1     & 0     & -1/7  & 2/7   & 25/3                                   \\
        -1                                  & $x_4$                               & 52/7                              & -3/7  & 0     & 0     & 1     & -6/7  & -2/7  &                                        \\
        \hline
        \multicolumn{2}{c|}{$-z$}           & 0                                   & 0                                 & 0     & 0     & 0     & -1    & -1    &                                                \\
        \hline
    \end{tabular}
    \caption{二阶段法例题1}
    \label{tab:example de two-phase method1}
\end{table}
\\
得到原问题的基本可行解$(0,15/7,25/7,52/7)^T$

第二阶段 删除人工变量,并把基本可行解填入表中

\begin{table}[ht]
    \centering
    \begin{tabular}{c|c|c|cccc|c}
        \hline
        \multirow{2}{*}{$\boldsymbol{c_B}$} & \multirow{2}{*}{$\boldsymbol{x_B}$} & \multirow{2}{*}{$\boldsymbol{b}$} & 0     & 0     & 0     & 0     & \multirow{2}{*}{$\boldsymbol{\theta}$} \\
        \cline{4-7}
                                            &                                     &                                   & $x_1$ & $x_2$ & $x_3$ & $x_4$ &                                        \\
        \hline
        2                                   & $x_2$                               & 15/7                              & -1/7  & 1     & 0     & 0     &                                        \\
        3                                   & $x_3$                               & 25/7                              & [3/7] & 0     & 1     & 0     & 25/3                                   \\
        -1                                  & $x_4$                               & 52/7                              & -3/7  & 0     & 0     & 1     &                                        \\
        \hline
        \multicolumn{2}{c|}{$-z$}           & -53/7                               & 25/7                              & 0     & 0     & 0     &                                                \\
        \hline
        2                                   & $x_2$                               & 10/3                              & 0     & 1     & 1/3   & 0     &                                        \\
        5                                   & $x_1$                               & 25/3                              & 1     & 0     & 7/3   & 0     &                                        \\
        -1                                  & $x_4$                               & 11                                & 0     & 0     & 1     & 1     &                                        \\
        \hline
        \multicolumn{2}{c|}{$-z$}           & -112/3                              & 0                                 & 0     & -25/3 & 0     &                                                \\
        \hline
    \end{tabular}
    \caption{二阶段法例题2}
    \label{tab:example de two-phase method2}
\end{table}
得到基本可行解:$(25/3,10/3,0,11)^T$为最优解

得到最优值$z=112/3$

(表见下页)
\\
\\
\\

\subsection{线性规划的对偶问题}

给定一个优化问题,我们去理解它的时候,或者设计算法的时候,可以研究它的对偶。

有时原问题不好解,但它的对偶相对容易。这个时候,可以从对偶问题出发,进而寻求原问题的解。

\subsubsection{对偶问题的形式}

\begin{enumerate}
    \item 对称形式的对偶问题
          \begin{flalign*}
               & (P)
              \begin{cases}
                  \max\quad & z=\boldsymbol{c}^T \boldsymbol{x}  \\
                  s.t.\quad & \boldsymbol{Ax} \le \boldsymbol{b} \\
                            & \boldsymbol{x}\ge \boldsymbol{0}
              \end{cases}
               & (D)
              \begin{cases}
                  \min\quad & f=\boldsymbol{b}^T \boldsymbol{y}                 \\
                  s.t.\quad & \boldsymbol{A}^T\boldsymbol{y} \ge \boldsymbol{c} \\
                            & \boldsymbol{y}\ge \boldsymbol{0}
              \end{cases}
               &     &
          \end{flalign*}

          一对对称形式的对偶规划之间具有下面的对应关系
          \begin{itemize}
              \item “$\max,\le$”和“$\min,\ge$”相对应
              \item 从约束系数矩阵看:一个模型中为$\boldsymbol{A}$,则另一个模型中为$\boldsymbol{A}^T$;一个模型是m个约束、n个变量,则它的对偶模型为n个约束、m个变量
              \item 从数据b、c的位置看:在两个规划模型中,b和c的位置对换
              \item 两个规划模型中的变量皆非负
          \end{itemize}

    \item 非对称形式的对偶问题

          (下式并不囊括所有情况)

          \begin{flalign*}
               & (P)
              \begin{cases}
                  \max\quad & z=\boldsymbol{c}^T \boldsymbol{x} \\
                  s.t.\quad & \boldsymbol{Ax} = \boldsymbol{b}  \\
                            & \boldsymbol{x}\ge \boldsymbol{0}
              \end{cases}
               & (D)
              \begin{cases}
                  \min\quad & f=\boldsymbol{b^T y}                              \\
                  s.t.\quad & \boldsymbol{A}^T\boldsymbol{y} \ge \boldsymbol{c}
              \end{cases}
               &     &
          \end{flalign*}

          一般称不具有对称形式的一对线性规划为非对称形式的对偶规划。
          对于非对称形式的规划,可以按照下面的对应关系直接给出其对偶规划。
          \begin{itemize}
              \item 将模型统一为“$\max,\le$”或“$\min,\ge$” 的形式
              \item 若原规划的某个约束条件为等式约束,则在对偶规划中与此约束对应的那个变量取值没有非负限制
              \item 若原规划的某个变量的值没有非负限制,则在对偶问题中与此变量对应的那个约束为等式
          \end{itemize}
\end{enumerate}

原问题与对偶问题的对应关系

\begin{table}[ht]
    \centering
    \begin{tabular}{cc|cc}
        \hline
        \multicolumn{2}{c|}{原问题(对偶问题)}  & \multicolumn{2}{c}{对偶问题(原问题)}                                   \\
        \hline
        \multicolumn{2}{c|}{min}        & \multicolumn{2}{c}{max}                                         \\
        \hline
        \multirow{4}{*}{变量}             & n个变量                           & \multirow{4}{*}{约束} & n个约束     \\
                                        & 变量$\ge$0                       &                     & 约束$\le$  \\
                                        & 变量$\le$0                       &                     & 约束$\ge$  \\
                                        & 无正负限制                          &                     & 约束 =     \\
        \hline
        \multirow{4}{*}{约束}             & m个约束                           & \multirow{4}{*}{变量} & m个变量     \\
                                        & 约束$\le$                        &                     & 变量$\le$0 \\
                                        & 约束$\ge$                        &                     & 变量$\ge$0 \\
                                        & 约束 =                           &                     & 无正负限制    \\
        \hline
        \multicolumn{2}{c|}{约束条件右端项}    & \multicolumn{2}{c}{目标函数中的变量系数}                                  \\
        \multicolumn{2}{c|}{目标函数中的变量系数} & \multicolumn{2}{c}{约束条件右端项}                                     \\
        \hline
    \end{tabular}
    \caption{原问题与对偶问题的对应关系}
    \label{tab:corresponding relationship between problem
        P and the problem D}
\end{table}

\subsubsection{对偶定理}

设有一对互为对偶的线性规划

\begin{flalign*}
     & (P)
    \begin{cases}
        \max\quad & z=\boldsymbol{c}^T \boldsymbol{x}  \\
        s.t.\quad & \boldsymbol{Ax} \le \boldsymbol{b} \\
                  & \boldsymbol{x}\ge \boldsymbol{0}
    \end{cases}
     & (D)
    \begin{cases}
        \min\quad & f=\boldsymbol{b}^T \boldsymbol{y}                 \\
        s.t.\quad & \boldsymbol{A}^T\boldsymbol{y} \ge \boldsymbol{c} \\
                  & \boldsymbol{y}\ge \boldsymbol{0}
    \end{cases}
     &     &
\end{flalign*}

\begin{theorem}
    若$\boldsymbol{x}$和$\boldsymbol{y}$分别为原规划$(P)$和$(D)$对偶规划,则
    $$
        \boldsymbol{c}^T\boldsymbol{x}\le \boldsymbol{b}^T\boldsymbol{y}
    $$
\end{theorem}

\begin{theorem}
    设$\boldsymbol{x}$和$\boldsymbol{y}$分别为原规划$(P)$和$(D)$的可行解,当$\boldsymbol{c}^T\boldsymbol{x}= \boldsymbol{b}^T\boldsymbol{y}$时,$\boldsymbol{x},\boldsymbol{y}$分别是两个问题的最优解
\end{theorem}
\begin{theorem}
    若规划$(P)$有可行解,则规划$(P)$有最优解的充分必要条件是规划$(D)$有可行解
\end{theorem}

\begin{theorem}
    若规划$(D)$有可行解,则规划$(D)$有最优解的充分必要条件是规划$(P)$有可行解
\end{theorem}

\begin{theorem}
    若原规划$(P)$有最优解,则对偶规划$(D)$也有最优解,反之亦然,且两者的目标函数值相等。
\end{theorem}

\subsubsection{对偶单纯形法}

对偶单纯形法是求解原规划的一种方法。

\textbf{原理}:

作为原规划的一个解,会有两个性质等待满足:可行性和最优性。而原问题的可行性和最优性恰好对应对偶问题的最优性和可行性。

单纯形法的思路是,先满足可行性,再逐渐逼近最优性;而对偶单纯形法的思路是,先找到最优性,再逐渐逼近可行性。

也就是说,先找到对偶问题的可行解,再找到原问题的可行解(即对偶问题的最优解)。

最优性:看检验数$\sigma_j$
可行性:看右端项$b_i$

从原规划的一个基本解出发,此基本解不一定可行,但它对应着一个对偶可行解;就是说可以从一个对偶可行解出发,然后检验原规划的基本解是否可行,即是否有负的分量。如果有负的分量,则进行迭代,求另一个基本解,此基本解对应着另一个对偶可行解(检验数非正);而得到的基本解的分量皆非负,则该基本解为最优解。

也就是说,对偶单纯形法在迭代过程中始终保持对偶解的可行性(检验数非正),使得原规划的基本解由不可行变为可行,当同时得到对偶规划与原规划的可行解时,得到原规划的最优解。

STEPS:
\begin{enumerate}
    \item 根据线性规划典式形式,建立初始单纯形表(就是还按照单纯形法填初始单纯形表)。此表对应原规划的一个基本解。表要求:检验数数行各元素一定非正,原规划的基本解可以有小于零的分量。
    \item 若基本解的所有分量皆非负,则得到原规划的最优解,停止计算;若基本解中有小于零的分量$b_i$,并且$b_i$所在行各系数$a_{ij}\ge0$,则原规划无可行解,停止计算;若$b_i<0$,并且存在$a_{ir}<0$,则确定$x_r$为出基变量,并计算$$\theta=\min\left\{\frac{\sigma_j}{a_{rj}}|a_{rj}<0\right\}=\frac{\sigma_k}{a_{rk}}$$确定$x_k$为进基向量。若有多个$b_i<0$,则选择最小的进行分析计算。
    \item 以$b_{rk}$为中心元素,按照与单纯形法类似的方法,在表中进行迭代计算,返回第2步。
\end{enumerate}

\begin{figure}[ht]
    \centering
    
\tikzstyle{startstop} = [rectangle, rounded corners, minimum width=2cm, minimum height=1cm,text centered, draw=black]
\tikzstyle{io} = [minimum width=3cm, minimum height=1cm, text centered]
\tikzstyle{process} = [rectangle, minimum width=3cm, minimum height=1cm, text width=4.7cm, text centered, draw=black]
\tikzstyle{decision} = [diamond, aspect=2, minimum width=3cm, minimum height=1cm, text centered, draw=black]
\tikzstyle{point}=[coordinate,]
%\tikzstyle{arrow} = [thick,-Stealth]
\tikzstyle{arrow} = [->,>=stealth]

\begin{tikzpicture}[node distance=2.4cm]

    % Define nodes for the primal simplex method
    \node[io](start1){单纯形法};
    \node[process, below of = start1, yshift=1.5cm](cul1){典式对应原规划的\\ 基本解是可行的};
    \node[decision, below of = cul1](dec1){所有$\sigma_i\le0?$};
    \node[process, below of = dec1](cul2){计算
        $\sigma_k=\max\{\sigma_j|\sigma_j>0\}$};
    \node[decision, below of = cul2](dec2){所有$a_{ik}\le0$?};
    \node[process, below of = dec2](cul3){计算\\ $\theta=\min\left\{\frac{b_i}{a_{ik}}|a_{ik}>0\right\}=\frac{b_e}{a_{ek}}$};
    \node[process, below of = cul3](end1){以$a_{ek}$为中心元素进行迭代};

    % Define nodes for the dual simplex method
    \node[io, right of = start1, xshift=7.5cm](start2){对偶单纯形法};
    \node[process, below of = start2, yshift=1.5cm](pro1){ 典式对应原规划的\\ 基本解的检验数$\sigma_j\le0$ };
    \node[decision, below of = pro1](deci1){所有$b_i\ge0?$};
    \node[process, below of = deci1](pro2){计算
        $b_e=\min\{b_i|b_i<0\}$};
    \node[decision, below of = pro2](deci2){所有$a_{lj}\ge0$?};
    \node[process, below of = deci2](pro3){计算\\ $\theta=\min\left\{\frac{\sigma_i}{a_{ej}}|a_{ej}<0\right\}=\frac{\sigma_k}{a_{ek}}$};
    \node[process, below of = pro3](end2){以$a_{ek}$为中心元素进行迭代};


    % center
    \node[process, right of=dec1, xshift=2.6cm, text width=1.5cm](cpro1){得到\\最优解};
    \node[startstop, below of=cpro1](cstop){结束};
    \node[process, right of=dec2, xshift=.7cm, minimum width=1cm, text width=.5cm](cpro2){没有有限最优解};
    \node[process, left of=deci2, xshift=-.7cm, minimum width=1cm, text width=.5cm](cpro3){没有有限最优解};

    % return
    \node[point, left of = end1, node distance=3cm](point1){};
    \node[point, above of = dec1, yshift=-1cm](point2){};
    \node[point, right of = end2, node distance=3cm](point3){};
    \node[point, above of = deci1, yshift=-1cm](point4){};

    % arrow
    \draw [arrow] (cul1) -- (dec1);
    \draw [arrow] (dec1) -- node[right]{N} (cul2);
    \draw [arrow] (cul2) -- (dec2);
    \draw [arrow] (dec2) -- node[right]{N} (cul3);
    \draw [arrow] (cul3) -- (end1);

    \draw [arrow] (pro1) -- (deci1);
    \draw [arrow] (deci1) -- node[right]{N} (pro2);
    \draw [arrow] (pro2) -- (deci2);
    \draw [arrow] (deci2) -- node[right]{N} (pro3);
    \draw [arrow] (pro3) -- (end2);

    \draw [arrow] (dec1) -- node[above]{Y} (cpro1);
    \draw [arrow] (deci1) -- node[above]{Y} (cpro1);
    \draw [arrow] (cpro1) -- (cstop);
    \draw [arrow] (dec2) -- node[above]{Y} (cpro2);
    \draw [arrow] (cpro2) -- (cstop);
    \draw [arrow] (deci2) -- node[above]{Y} (cpro3);
    \draw [arrow] (cpro3) -- (cstop);

    \draw [-](end1) -- (point1);
    \draw [arrow](point1) |- node{}(point2);
    \draw [-](end2) -- (point3);
    \draw [arrow](point3) |- node{}(point4);

\end{tikzpicture}


    \caption{单纯形法与对偶单纯形法流程图}
    \label{con:DualSMFlow}
\end{figure}


\subsection{灵敏度分析}

前提:在求灵敏度分析和影子价格中,常常会遇到要使用对偶问题的最优解(或最优基$\boldsymbol{B}$)求解问题,这里解释一下怎么使用最优单纯形表求得。

\begin{enumerate}
    \item 首先利用单纯形法得到最优单纯形表
    \item 得到$b$列的$n$个数值,最优单纯形表从后往前数$n$列(就是松弛变量对应的列),有矩阵:$$\begin{pmatrix}
                  a_{1(m-n)}   & a_{1(m-n+1)} & \cdots & a_{1m}       \\
                  a_{2(m-n)}   & a_{2(m-n+1)} & \cdots & a_{2m}       \\
                  \vdots       &              &        & \vdots       \\
                  a_{n(m-n)}   & a_{n(m-n+1)} & \cdots & a_{nm}       \\
                  \hline
                  \sigma_{m-n} & \sigma_{m-n} & \cdots & \sigma_{m-n}
              \end{pmatrix}$$
          其中:
          \begin{itemize}
              \item 最后一行的检验数$\sigma^T =- \boldsymbol{c_B}^T\boldsymbol{B}^{-1}$,各检验数取相反数(即$\boldsymbol{c_B}^T\boldsymbol{B}^{-1}$)即为对偶问题的最优解。
              \item 除最后一行外(即线上方的)矩阵即为$\boldsymbol{B}^{-1}$
          \end{itemize}
\end{enumerate}

\subsubsection{影子价格}

影子价格是一个向量,它的分量表示最优目标值随相应资源数量变化的变化率。

若$x^*, y^*$分别为$(LP)$和$(DP)$的最优解,那么有$$\boldsymbol{c}^T\boldsymbol{x}^*=\boldsymbol{b}^T\boldsymbol{y}^*$$根据$f=\boldsymbol{b}^T\boldsymbol{y}^*=b_1y^*_1, b_2y^*_2, \cdots, b_my^*_m$可知$$\frac{\partial f}{\partial b_i} = y_i^*$$

$y_i^*$表示$b_i$变化一个单位对目标$f$产生的影响,称$y_i^*$为$b_i$的影子价格

需要指出,影子价格不是固定不变的,当约束条件、产品利润等发生变化时,有可能使影子价格发生变化。另外,影子价格的经济含义,是指资源在一定范围内增加时的情况,当某种资源的增加超过了这个“一定的范围”时,总利润的增加量则不是按照影子价格给出的数值线性地增加。

如何求解影子价格:
\begin{enumerate}
    \item 求出对偶问题的最优解(可以利用单纯形表,在求出最优解的情况下,将松弛变量对应的各检验数取负$-\sigma_i$就能得到对偶问题的最优解)
    \item 对偶问题最优解中的数字依次对应的就是原问题的各资源影子价格
\end{enumerate}

\textbf{经济意义}:每增加一单位的某资源,最终收益增加多少单位

判断资源是否有剩余:$\begin{cases}
        y_i^*=0\quad \mbox{有剩余} \\
        y_i^*>0\quad \mbox{无剩余}
    \end{cases}$

\subsubsection{目标函数系数c变化\textasteriskcentered}
若只有一个系数$c_j$变化,其他系数不变。$c_j$的变化只影响检验数$\sigma_j$,而不影响解的非负性。
$$\sigma_j=c_j-\boldsymbol{c_B}^T\boldsymbol{B}^{-1}\boldsymbol{p}_j, j=1, 2, \cdots, n $$
\begin{enumerate}
    \item $c_k$是非基变量的系数\\
          非基系数变化只影响与$c_k$有关的一个检验数$\sigma_k$的变化,对其他无影响,故只需要考虑$\sigma_k$。\\
          设$c_k\rightarrow \bar{c}_k=c_k + \Delta c_k$,有$\sigma_k$的变化:$$\bar{\sigma}_k=c_k + \Delta c_k-\boldsymbol{c_B}^T\boldsymbol{B}^{-1}\boldsymbol{p}_j= \sigma_k+ \Delta c_k$$为了保持最优解不变,$\sigma_k$必须满足$\bar{\sigma}_k=\sigma_k+\Delta c_k\le 0$。也就是说:
          \begin{equation}
              \Delta c_k\le -\sigma_k,\bar{c}_k=c_k+\Delta c_k\le c_k-\sigma_k
              \label{con:2.3}
          \end{equation}
          $c_k-\sigma_k$是$c_k$变化的上限,若$c_k$不超出上限,最优解不变;否则,将最优单纯形表中的检验数$\sigma_k$用$\bar{\sigma}_k$取代,取$x_k$为进基变量,继续单纯形的表格计算。

    \item $c_l$是基变量的系数\\
          设$c_l\rightarrow c_l+\Delta c_l$,引入$\Delta c=(0, \cdots, 0, \Delta c_l, 0, \cdots, 0)$,有
          \begin{align*}
              \sigma_j\rightarrow\bar{\sigma}_j & =c_j-[\boldsymbol{c_B}^T+(\Delta\boldsymbol{c})^T]\boldsymbol{B}^{-1}\boldsymbol{p}_j,j\ne l \\
                                                & =\sigma_j-\Delta c_la_{lj}^{\prime}
          \end{align*}
          (注意:上式中a的l与左边基变量的下标对应)
          为保证最优解不变,$\Delta c_l$要满足$$\max\left\{ \frac{\sigma_j}{a_{ij}^{\prime}}|a_{ij}^{\prime}>0 \right\}\le\Delta c_l\le\min\left\{ \frac{\sigma_j}{a_{ij}^{\prime}}|a_{ij}^{\prime}<0 \right\}$$若$\Delta c_l$超出此范围,应求出新的检验数$\bar{\sigma}_j$,选择其中大于零的检验数对应的变量$x_j$为进基变量,继续迭代。
\end{enumerate}

\subsubsection{右端常数b变化\textasteriskcentered}

$b_r$的变化影响解的可行性,但不影响检验数的符号变化。由$\boldsymbol{x_B}=\boldsymbol{B}^{-1}\boldsymbol{b}$可知$b_r$的变化必会引起最优解数值变化。

最优解的变化分为以下两类:
\begin{enumerate}
    \item 保持$\boldsymbol{B}^{-1}\boldsymbol{b}\ge\boldsymbol{0}$,即最优基$\boldsymbol{B}$不变(影子价格不变,也就是对偶问题的最优解不变)\\
          只需要将变化后的$b_r$带入$\boldsymbol{B}^{-1}\boldsymbol{b}$的表达式重新计算即可
    \item $\boldsymbol{B}^{-1}\boldsymbol{b}$出现负分量,这使最优基$\boldsymbol{B}$变化\\
          需要通过迭代求解新的最优基和最优解
\end{enumerate}
综合一下,可以利用下面的步骤计算:\\
设$b_r\rightarrow\bar{b}_r=b_r+\Delta b_r$,$\Delta b_r$,此时有$$
    \boldsymbol{x_B}\rightarrow\boldsymbol{\bar{x}_B}=\boldsymbol{B}^{-1}\begin{pmatrix}
        b_1 \\ \vdots\\ b_r+\Delta b_r\\ \vdots\\ b_m
    \end{pmatrix}=\boldsymbol{B}^{-1}\boldsymbol{b}+\boldsymbol{B}^{-1}\begin{pmatrix}
        0 \\ \vdots\\ \Delta b_r\\ \vdots\\ 0
    \end{pmatrix}=\boldsymbol{x_B}+\Delta b_r\begin{pmatrix}
        \beta_{1r} \\ \vdots\\ \beta_{mr}
    \end{pmatrix}=\begin{pmatrix}
        b_1^{\prime} \\ \vdots\\ b_m^{\prime}
    \end{pmatrix}+\Delta b_r\begin{pmatrix}
        \beta_{1r} \\ \vdots\\ \beta_{mr}
    \end{pmatrix}\ge\begin{pmatrix}
        0 \\ \vdots\\ 0
    \end{pmatrix}
$$
(上式只需要随便算其中的一个,然后看是否大于等于0就行,不需要特别算下面的$\Delta b_r$)

其中,为$\boldsymbol{x_B}^{-1}\boldsymbol{b}$原最优解,$b_i^{\prime}$为$\boldsymbol{x_B}=\boldsymbol{B}^{-1}\boldsymbol{b}$的第$i$个分量,$\beta_{ir}$为$\boldsymbol{B}^{-1}$的第$i$行第$r$列元素

要满足$$\max\left\{ \frac{-b_i^{\prime}}{\beta_{ir}}|\beta_{ir}>0 \right\}\le\Delta b_r\le\min\left\{ \frac{-b_i^{\prime}}{\beta_{ir}}|\beta_{ir}<0 \right\}$$当$\Delta b_r$超过此范围时,将使最优解中某个分量小于零,使最优基发生变化。此时可用对偶单纯形法继续迭代新的最优解。

就是说,如果没超范围,直接把$\boldsymbol{\bar{x_B}}$当最优解得出就行;如果超范围的话,就把$\boldsymbol{\bar{x_B}}$放到最优单纯形表的$b$列,然后求对偶单纯形。

\subsubsection{约束条件系数a变化}
假设只有一个$a_{ij}$变化,其他数据不变,且只讨论$a_{ij}$为非基变量$x_j$的系数的情况。那么此时$a_{ij}$的变化只影响一个检验数$\sigma_j$。

设$a_{ij}\rightarrow a_{ij}+\Delta a_{ij}$,由检验数的另一种表示形式$$\sigma_j\rightarrow\bar{\sigma}_j=c_j-\boldsymbol{y^T}
    \begin{pmatrix}a_{1j}\\ \vdots\\ a_{ij}+\Delta a_{ij}\\ \vdots\\ a_{mj}\end{pmatrix}=c_j-\boldsymbol{y^Tp}_j-\boldsymbol{y^T}\begin{pmatrix}0\\ \vdots\\ \Delta a_{ij}\\ \vdots\\ 0 \end{pmatrix}=\sigma_j-y_i^*\Delta a_{ij}$$其中,$\boldsymbol{y}$为对偶最优解,$y_i^*$为$\boldsymbol{y}$的第$i$个变量

为使最优解不变,要使$\sigma_j\le0$,即\begin{align*}
    \sigma_j\le y_i^*\Delta a_{ij}                 \\
    \Delta a_{ij}\ge\frac{\sigma_j}{y_i^*},y^*_i>0 \\
    \Delta a_{ij}\le\frac{\sigma_j}{y^*_i},y^*_i<0
\end{align*}

\subsubsection{新增变量x分析}

增加变量$x_{n+1}$,则有相应的约束条件$\boldsymbol{p}_{n+1}$,目标函数系数$c_{n+1}$,那么,计算出$$\sigma_{n+1}=c_{n+1}-\boldsymbol{c_B}^T\boldsymbol{B}^{-1}\boldsymbol{p}_{n+1}$$
填入最优单纯形表,若 $\sigma_{n+1} \le 0$,则最优解不变;否则,进一步用单纯形法求解

\subsubsection{新增约束条件}

增加一个约束之后,应把最优解代入新的约束,若满足,则最优解不变;否则,填入最优单纯形表作为新的一行,引入一个新的非负变量(原约束若是小于等于形
式,可引入非负松弛变量;否则,引入非负人工变量),并通过矩阵行变换把对应基
变量的元素变为0,进一步用单纯形法或对偶单纯形法求解。

\chapter{最优化搜索算法的结构和一维搜索}

从此章开始就是求非线性规划问题,与线性规划问题可以在有限步数内得到解不同,非线性规划问题不一定可以在有限的迭代步数内收敛。

\section{常用的搜索算法结构}

\subsection{收敛性概念}

在规划问题的的求解过程中,由于迭代算法是以产生一系列迭代点为目的的,因此算法的收敛性表现在产生的点列$ \{ \boldsymbol{x}^{(k)} \} $上。

\begin{enumerate}
    \item 理想的收敛性概念:设 $\boldsymbol{x}^*$ 为问题 $(fS)$ 的全局最优解。当 $\boldsymbol{x}^* \in \{\boldsymbol{x}^{(k)}\}$ 或 $\boldsymbol{x}^{(k)} \ne \boldsymbol{x}^*$ 对于所有 $k$ 成立,并且满足 $\lim\limits_{k \rightarrow \infty} \boldsymbol{x}^{(k)} = \boldsymbol{x}^*$ 时,称算法收敛到最优解 $\boldsymbol{x}^*$。
    \item 然而,由于实际问题较为复杂,很难达到理想状态,因此需要引入一些实用的收敛性概念:

          首先定义解集 $\Omega$ ,是具有某种可接受性质的点集,通常取下面的几种集合:
          \begin{enumerate}
              \item $\Omega = \{\boldsymbol{x}^*|\boldsymbol{x}^*\ is\ g.opt\}$
              \item $\Omega = \{\boldsymbol{x}^*|\boldsymbol{x}^*\ is\ l.opt\}$
              \item $\Omega = \{\boldsymbol{x}^*|\boldsymbol{x}^*\hbox{满足某种最优条件,或者可以说}\nabla f(\boldsymbol{x}^*)=0\}$
              \item $\Omega = \{\boldsymbol{x}^*|\boldsymbol{x}^*\in S,f(\boldsymbol{x}^*)\le B\}$,其中 $B$ 为可接受的目标函数值的一个上界
          \end{enumerate}
          设算法产生的点列为 $\boldsymbol{x}^{(k)}$ ,满足下列任一情况时,称算法收敛:
          \begin{enumerate}
              \item $\{\boldsymbol{x}^{(k)}\} \cap \Omega \ne \varnothing$
              \item $\{\boldsymbol{x}^{(k)}\}$ 的任意收敛子列的极限点属于 $\Omega$
          \end{enumerate}
\end{enumerate}

求解问题时,需要考虑初始点的影响,因此引入下面的概念:
\begin{enumerate}
    \item 全局收敛性:若算法对任意的初始点 $\boldsymbol{x}^{(1)}$ 都收敛,则称算法全局收敛。
    \item 局部收敛性:若算法只有限制初始点 $\boldsymbol{x}^{(1)}$ 到解集 $\Omega$ 附近(当 $\Omega$ 为非连通时,指在 $\Omega$ 的某个连通子集附近)时,才有收敛性,则称算法局部收敛。
\end{enumerate}

\subsection{收敛准则(停机条件)}

\begin{enumerate}
    \item $ ||\boldsymbol{x}^{(k+m)}-\boldsymbol{x}^{(k)}|| < \varepsilon $(最常用的是 $ m=1 $ )
    \item $\frac{||\boldsymbol{x}^{(k+1)}-\boldsymbol{x}^{(k)}||}{||\boldsymbol{x}^{(k)}||}<\varepsilon$
    \item $|f(\boldsymbol{x}^{(k+1)})-f(\boldsymbol{x}^{(k)})|<\varepsilon$
\end{enumerate}

\subsection{收敛速度}

设 $\{\boldsymbol{x}^{(k)}\}$ 为算法产生的点列,收敛于解 $\boldsymbol{x}^*$ ,且 $\boldsymbol{x}^{(k)}\ne\boldsymbol{x}^*$,有 $\forall k$

\begin{enumerate}
    \item 线性收敛:$\exists \alpha\in(0,1)\frac{||\boldsymbol{x}^{(k+1)}-\boldsymbol{x}^*||}{||\boldsymbol{x}^{(k)}-\boldsymbol{x}^*||}\le\alpha, k$充分大时 $\boldsymbol{x}^{(k)}$ 收敛于 $\boldsymbol{x}^*$\\
          表示每迭代一次,比上一次更接近 $\boldsymbol{x}^*$;收敛的阶数 $p_0=1$
    \item 超线性收敛:$\lim\limits_{k \rightarrow \infty}\frac{||\boldsymbol{x}^{(k+1)}-\boldsymbol{x}^*||}{||\boldsymbol{x}^{(k)}-\boldsymbol{x}^*||}=0$\\
          迭代速度比线性收敛更快,收敛的阶数 $p_0\ge1$
    \item 二阶收敛:$\exists\alpha>0$,使得 $k$ 充分大时,有 $\frac{||\boldsymbol{x}^{(k+1)}-\boldsymbol{x}^*||}{||\boldsymbol{x}^{(k)}-\boldsymbol{x}^*||^2}\le\alpha$\\
          收敛的阶数 $p_0=2$,说明二阶收敛是超线性收敛的一个特例
\end{enumerate}

\subsection{二次终结性}
\begin{definition}
    一个算法用于解正定二次函数的无约束极小时,有限步迭代可达最优解,则称该算法具有二次终结性。
\end{definition}

二次终结性=共轭方向+精确一维搜索

\subsubsection{共轭方向}

\begin{definition}
    设 $A_{n\times n}$ 对称正定,$d^{(1)},d^{(2)} \in R^n,d^{(1)} \ne 0,d^{(2)} \ne 0$,满足$d^{(1)T}Ad^{(2)}=0$, 称$d^{(1)},d^{(2)}$ 关于矩阵$\boldsymbol{A}$共轭。
\end{definition}

\subsection{下降算法模型}

考虑规范形式的规划问题 $$
    (fS)\quad\begin{cases}
        min\quad  & f(\boldsymbol{x})   \\
        s.t.\quad & \boldsymbol{x}\in S
    \end{cases}
$$
希望能更接近目标函数的最优解,即对 $\forall k,f(\boldsymbol{x}^{(k+1)})<f(\boldsymbol{x}^{(k)})$,这样的算法称为下降算法。

首先要找到下降可行的方向,即能够接近最优解的方向。

设 $\bar{\boldsymbol{x}}\in S$,对方向 $\boldsymbol{d}$ ,如果 $\exists\delta>0$,使 \begin{equation}
    f(\boldsymbol{x}+\boldsymbol{\lambda d})<f(\bar{\boldsymbol{x}}),\forall\lambda\in(0,\delta)
    \label{con:3.1}
\end{equation}
那么称 $\boldsymbol{d}$ 为 $f(\boldsymbol{x})$ 在点 $\bar{\boldsymbol{x}}$ 的下降方向;就是说在某个点上,沿着这个方向移动可以朝着更优的解靠近。

然而,对于问题 $(fS)$ ,只找到下降方向,有时不能产生新的迭代点,还要求 $\boldsymbol{d}$ 为可行方向(换句话说,沿着可行方向前进,可以保持在可行解的范围内),即 $\exists\delta>0$ ,使
\begin{equation}
    \boldsymbol{x}+\boldsymbol{\lambda d}\in S,\forall\lambda\in(0,\delta)
    \label{con:3.2}
\end{equation}

如果方向 $\boldsymbol{D}$ 同时满足式(\ref{con:3.1})和(\ref{con:3.2}),那么称 $\boldsymbol{d}$ 为问题 $(fS)$ 的下降可行方向。

\begin{figure}[ht]
    \centering
    \begin{tikzpicture}
        \draw (0,0) circle (1cm);
        \filldraw[black] (0.5,0.5) circle (1pt) node[anchor=west] {$\bar{\boldsymbol{x}}$};
        \draw[->] (0.5,0.5) -- (-0.4,-0.5) node[anchor=west] {$\boldsymbol{d}$};
        \node[anchor=south] at (0,1.1) {$S$};
    \end{tikzpicture}
    \caption{下降可行方向}
    \label{fig:Feasible Descent Direction}
\end{figure}

下降算法模型如图\ref{fig:DescentOptimizationFlow}所示,接下来先要利用一维搜索找到步长因子 $\lambda_k$

\begin{figure}[ht]
    \centering
    % \documentclass{article}
% \usepackage{tikz}
% \usetikzlibrary{shapes.geometric, arrows}
% \usepackage{amsmath}

% \thispagestyle{empty}
% 流程图定义基本形状
\begin{tikzpicture}[node distance=3cm]
    \tikzstyle{startstop} = [rectangle, rounded corners, minimum width = 2cm, minimum height=1cm,text centered, draw = black, align=center]
    %\tikzstyle{io} = [trapezium, trapezium left angle=70, trapezium right angle=110, minimum width=2cm, minimum height=1cm, text centered, draw=black]
    \tikzstyle{process} = [rectangle, minimum width=2cm, minimum height=1cm, text centered, draw=black, align=center]
    \tikzstyle{decision} = [diamond, aspect = 3, text centered, draw=black,align=center]
    \tikzstyle{point}=[coordinate,]
    % 箭头形式
    \tikzstyle{arrow} = [->,>=stealth]

    % \begin{document}

    \node (start) [startstop] {选初始点 $\boldsymbol{x}^{(1)}\in S$,\\ 确定停机条件 $k=1$ };
    \node (pro1) [process, below of=start] {对 $\boldsymbol{x}^{(k)}$ 点选择\\
        $\boldsymbol{d}^{(k)}$ 为下降可行方向};
    \node (pro2) [process, below of=pro1] {线性搜索求步长因子 $\lambda_k$\\ $\boldsymbol{x}^{(k+1)}=\boldsymbol{x}^k+\lambda_k\boldsymbol{d}^{(k)}$\\ 使 $\boldsymbol{x}^{(k+1)}\in S$,且\\ $f(\boldsymbol{x}^{(k+1)})<f(\boldsymbol{x^{(k)}})$ };
    \node (dec1) [decision, below of=pro2, align=center] {$\boldsymbol{x}^{(k+1)}$\\ 是否满足停机条件?};
    \node[point, below of = start, yshift=1.5cm](point1){};
    \node (pro3) [process, left of=point1, xshift=-2cm] {$k=k+1$};
    \node (stop) [startstop, right of=dec1, node distance=5cm] {停机};


    \draw [arrow] (start) -- (pro1);
    \draw [arrow] (pro1) -- (pro2);
    \draw [arrow] (pro2) -- (dec1);
    \draw [arrow] (dec1) -| node[anchor=north] {N} (pro3);
    \draw [arrow] (pro3) -- (point1);
    \draw [arrow] (dec1) -- (stop);

\end{tikzpicture}
% \end{document}
    \caption{下降算法模型}
    \label{fig:DescentOptimizationFlow}
\end{figure}

\section{一维搜索}

一元函数求最小值和线性搜索均为一维搜索,用于求步长因子 $\lambda_k$ 即:
\begin{align*}
     & \min\quad f(\boldsymbol{x}^{(k)}+\lambda\boldsymbol{d}^{(k)})=\varphi (\lambda) \\
     & s.t.\quad \lambda\in S
\end{align*}
这里 $S$ 有三种情况:$(-\infty,+\infty),(0,+\infty),[a,b]$

\subsection{缩小区间的精确一维搜索}

考虑问题 $P$
\begin{align*}
     & \min\quad \varphi(\lambda)                 \\
     & s.t.\quad \lambda\in[\alpha,\beta]         \\
     & \varphi: \mathbf{R} \rightarrow \mathbf{R}
\end{align*}

\subsubsection{不确定区间和单峰函数}

不确定区间:表示 $[\alpha,\beta]$ 含最优解,即 $\varphi(\lambda)$ 的最小值,但是不知道具体位置

\begin{definition}
    设函数 $\varphi:\mathbf{R}\rightarrow\mathbf{R}, [\alpha,\beta]$为不确定区间,$\lambda^*$为 $\varphi$ 在 $[\alpha,\beta]$ 上的最小值点,如果 $\forall\lambda_1,\lambda_2\in[\alpha,\beta], \lambda_1<\lambda_2$满足:
    \begin{enumerate}
        \item 当 $\lambda_2\le\lambda^*$ 时,$\varphi(\lambda_1)>\varphi(\lambda_2)$
        \item 当 $\lambda_1\ge\lambda^*$ 时,$\varphi(\lambda_1)<\varphi(\lambda_2)$
    \end{enumerate}
    则称 $\varphi(\lambda)$ 在 $[\alpha,\beta]$ 上为是强单峰的。

    如果再加上条件,当 $\varphi(\lambda_1)\ne\varphi(\lambda^*), \varphi(\lambda_2)\ne\varphi(\lambda^*)$ 时,上式成立,则称 $\varphi(\lambda)$ 在 $[\alpha,\beta]$ 上为是单峰的。
\end{definition}

\begin{figure}[ht]
    \centering
    \begin{tikzpicture}
    % 绘制轴
    \draw (-1,0) -- (5,0) node[below] {};

    % 绘制强单峰函数曲线
    \draw[blue,line width=1pt, domain=0:3.5] plot (\x, {0.5*(\x-2)^2+1}) node[right] {强单峰};

    % 绘制单峰函数曲线
    \draw[red,line width=1pt, domain=0:1] plot (\x, {0.5*(\x-1)^2+2});
    \draw[red,line width=1pt, domain=1:2] -- (1, 2) -- (2,2);
    \draw[red,line width=1pt, domain=2:3.5] plot (\x, {0.5*(\x-1)^2+1.5}) node[right] {弱单峰};

    % 绘制区间边界
    \draw[dashed,line width=1pt] (0,0) -- (0,4);
    \draw[dashed,line width=1pt] (3.5,0) -- (3.5,4);

    % 添加标签
    \node[below] at (0,0) {$\alpha$};
    \node[below] at (3.5,0) {$\beta$};

\end{tikzpicture}
    \caption{强单峰与单峰}
    \label{fig:stronglyUnimodalAndUnimodal}
\end{figure}

单峰函数的特点是在最优解 $\lambda^*$ 的某一侧,距离 $\lambda^*$ 远的点的函数值必定不小于距离 $\lambda^*$ 近的点的函数值。根据该特点有下面的定理:

\begin{theorem}
    设 $\varphi(\lambda):\mathbf{R}\rightarrow\mathbf{R}$,在区间 $[\alpha,\beta]$ 上单峰。$\lambda, \mu\in [\alpha,\beta]$,且 $\lambda<\mu$,则
    \begin{enumerate}
        \item 若 $\varphi(\lambda)>\varphi(\mu)$,则 $\varphi(\xi)\ge\varphi(\mu),\forall \xi\in[\alpha,\lambda]$
        \item 若 $\varphi(\lambda)<\varphi(\mu)$,则 $\varphi(\xi)\ge\varphi(\mu),\forall \xi\in[\mu,\beta]$
        \item 若 $\varphi(\lambda)=\varphi(\mu)$,则 $\varphi(\xi)\ge\varphi(\mu),\forall \xi\in[\alpha,\lambda]\cup[\mu,\beta]$
    \end{enumerate}
\end{theorem}

$\xi$ 所在区间去掉后不会失去最优解,因此可以用缩小区间的方法求解最优解。也就是说,只需要在 $[\alpha,\beta]$ 上取两个点 $\lambda,\mu$ ,比较函数值 $\varphi(\lambda),\varphi(\mu)$ ,可以去掉$\xi$ 所在区间。

\subsubsection{进退法求出初始不确定区间}

实际计算时,往往不知道最优解的位置,因此需要先找到一个初始的不确定区间,基本思想是找三个点,使两端点的函数值比中间点的函数值大。

先任取点 $\lambda_0$ ,选取步长 $\delta >0$ 取 $\lambda_1=\lambda_0+\delta$ ,那么有两种情况
\begin{enumerate}
    \item 若 $\varphi(\lambda_0)<\varphi(\lambda_1)$,向左找,取 $\lambda_2=\lambda_0-2\delta$ ,再考察 $\varphi(\lambda_2)$:
          \begin{enumerate}
              \item 若 $\varphi(\lambda_0)\le\varphi(\lambda_2)$,则停,得到不确定区间 $[\lambda_2,\lambda_1]$
              \item 否则,取 $\lambda_1=\lambda_0,\lambda_0=\lambda_2,\delta=2\delta$ ,重复该过程。
          \end{enumerate}
    \item 若 $\varphi(\lambda_0)\ge\varphi(\lambda)$ ,向右找,取 $\lambda_2=\lambda_0+2\delta$ ,再考察 $\varphi(\lambda_2)$:
          \begin{enumerate}
              \item 若 $\varphi(\lambda_1)\le\varphi(\lambda_2)$,则停,得到不确定区间 $[\lambda_0,\lambda_2]$
              \item 否则,取 $\lambda_0=\lambda_1,\lambda_1=\lambda_2,\delta=2\delta$ ,重复该过程。
          \end{enumerate}
\end{enumerate}

如图\ref{fig:case of the retreat method 1}与\ref{fig:case of the retreat method 2}所示

\begin{figure}[ht]
    \centering
    \begin{minipage}{0.45\textwidth}
        \centering
        % 第一幅图的代码
        \begin{tikzpicture}
            \draw[line width=1pt] (-1,0) -- (5,0);

            \draw[line width=1pt] (0,0.1) -- (0,0) node[anchor=north] {$\lambda_2$};
            \draw[line width=1pt] (4,0.1) -- (4,0) node[anchor=north] {$\lambda_1$};
            \draw[line width=1pt] (3,0.1) -- (3,0) node[anchor=north] {$\lambda_0$};
            \draw[line width=1pt, dashed] (4,0) -- (4,3.5) node[anchor=south] {$\varphi(\lambda_1)$};
            \draw[line width=1pt, dashed] (3,0) -- (3,3) node[anchor=south] {$\varphi(\lambda_0)$};

            \draw[line width=1pt, decorate,decoration={brace,amplitude=5pt}] (0,0.1) -- (3,0.1) node[anchor=south,midway,yshift=0.3cm] {$2\delta$};
            \draw[line width=1pt, decorate,decoration={brace,amplitude=5pt}] (3,0.1) -- (4,0.1) node[anchor=south,midway,yshift=0.3cm] {$\delta$};
        \end{tikzpicture}
        \caption{进退法情况1}
        \label{fig:case of the retreat method 1}
    \end{minipage}
    % \hfill
    \begin{minipage}{0.45\textwidth}
        \centering
        % 第二幅图的代码
        \begin{tikzpicture}%[scale=0.9]
            \draw[line width=1pt] (-1,0) -- (5,0);

            \draw[line width=1pt] (0,0.1) -- (0,0) node[anchor=north] {$\lambda_0$};
            \draw[line width=1pt] (1,0.1) -- (1,0) node[anchor=north] {$\lambda_1$};
            \draw[line width=1pt] (4,0.1) -- (4,0) node[anchor=north] {$\lambda_2$};
            \draw[line width=1pt, dashed] (0,0) -- (0,3) node[anchor=south] {$\varphi(\lambda_0)$};
            \draw[line width=1pt, dashed] (1,0) -- (1,3.5) node[anchor=south] {$\varphi(\lambda_1)$};

            \draw[line width=1pt, decorate,decoration={brace,amplitude=5pt}] (0,0.1) -- (1,0.1) node[anchor=south,midway,yshift=0.3cm] {$\delta$};
            \draw[line width=1pt, decorate,decoration={brace,amplitude=5pt}] (1,0.1) -- (4,0.1) node[anchor=south,midway,yshift=0.3cm] {$2\delta$};
        \end{tikzpicture}
        \caption{进退法情况2}
        \label{fig:case of the retreat method 2}
    \end{minipage}
    % \label{fig:Two cases of the retreat method}
\end{figure}

需要注意的是,步长 $\delta$ 的选择要适度,太大时含有多个单峰区间,太小时迭代次数过多;另外,当 $\varphi(\lambda)$ 单调时,进退法无结果,因此需要加入迭代次数的限制。

\subsubsection{中点法}

若 $\varphi(\lambda):\mathbf{R}\rightarrow\mathbf{R}$ 可微,且 $\varphi(\lambda)$ 在 $[\alpha,\beta]$ 上单峰,那么可以用中点法求解最优解。

设 $\varphi(\lambda)$ 在单峰区间 $[\alpha,\beta]$ 可微,且不存在拐点(即 $\varphi^{\prime}(\lambda)$ 及 $\varphi^{\prime\prime}(\lambda)$ 同时为零的点),取 $\lambda=\frac{\alpha+\beta}{2}$ ,有中点法一般步骤:
\begin{enumerate}
    \item 若 $\varphi^{\prime}(\lambda)=0$ ,则 $\lambda$ 为最优解
    \item 若 $\varphi^{\prime}(\lambda)>0$ ,则 $\lambda$ 在上升段,$\lambda^*<\lambda$ ,去掉 $(\lambda,\beta]$ ,新区间为 $[\alpha,\lambda]$
    \item 若 $\varphi^{\prime}(\lambda)<0$ ,则 $\lambda$ 在下降段,$\lambda^*>\lambda$ ,去掉 $[\alpha,\lambda)$ ,新区间为 $[\lambda,\beta ]$
\end{enumerate}

利用中点法进行 $k$ 次迭代后,存在精度区间 $[\alpha^*,\beta^*]<\varepsilon$ ,也就是说有
\begin{align*}
    \frac{\beta-\alpha}{2^k}            & <\varepsilon                                      \\
    \frac{\beta-\alpha}{\varepsilon}    & <2^k                                              \\
    \ln\frac{\beta-\alpha}{\varepsilon} & <k\ln2                                            \\
    k                                   & >\frac{\ln\frac{\beta-\alpha}{\varepsilon}}{\ln2} \\
\end{align*}

需要注意的是,中点法只能用于单峰函数,就是说对函数的可微性有要求。

\subsubsection{黄金分割法}

黄金分割法是一种比较经典的缩小区间的方法,要求考虑条件:
\begin{enumerate}
    \item 对称性 $\lambda-\alpha=\beta-\mu$
    \item 缩减比不变 $\gamma=\frac{\hbox{保留区间的长度}}{\hbox{原区间的长度}}$
\end{enumerate}

\begin{figure}[ht]
    \centering
    \begin{tikzpicture}[line width=1pt]
        \draw (0,0) -- (10,0);

        \draw (0,0.1) -- (0,0) node[anchor=north] {$\alpha$};
        \draw (10,0.1) -- (10,0) node[anchor=north] {$\beta$};
        \draw (4,0.1) -- (4,0) node[anchor=north] {$\lambda$};
        \draw (6,0.1) -- (6,0) node[anchor=north] {$\mu$};

        \fill[pattern=north west lines] (6,0) rectangle (10,0.2);
    \end{tikzpicture}
    \caption{黄金分割法示意图}
    \label{fig:Golden section method}
\end{figure}

不妨设迭代选择的新的两个点为 $\lambda,\mu$ ,通过比较函数值去掉 $[\mu,\beta]$ (或 $[\alpha,\mu]$ )段,如图\ref{fig:Golden section method} ,那么有

\begin{gather*}
    \frac{\mu-\alpha}{\beta-\alpha}=\frac{\lambda-\alpha}{\mu-\alpha}=\gamma       \\
    \mu=\alpha+\gamma(\beta-\alpha),\lambda=\alpha+(1-\gamma)(\beta-\alpha) \\
    \gamma^2+\gamma-1 =0                                       \\
    \gamma=\frac{\sqrt{5}-1}{2}\approx0.618
\end{gather*}

下图为黄金分割法的流程图

\begin{figure}[ht]
    \centering
    
\begin{tikzpicture}[node distance=3cm,line width=1pt]
    \tikzstyle{startstop} = [rectangle, rounded corners, minimum width = 2cm, minimum height=1cm,text centered, draw = black, align=center]
    \tikzstyle{process} = [rectangle, minimum width=3cm, minimum height=1cm, text centered, draw=black, align=center]
    \tikzstyle{decision} = [diamond, aspect = 3, text centered, draw=black, align=center]
    \tikzstyle{point}=[coordinate]
    % 箭头形式
    \tikzstyle{arrow} = [->,>=stealth]

    % Nodes
    \node (start) [startstop] {初始区间 $[\alpha_1,\beta_1]$\\ $\gamma=\frac{\sqrt{5}-1}{2},\varepsilon,k=1$ };
    \node (pro1) [process, below of=start] {$\lambda_k=\alpha_k+(1-\gamma)(\beta_k-\alpha_k)$\\ $\mu_k=\alpha_k+\gamma(\beta_k-\alpha_k)$\\ 计算$\varphi(\lambda_k),\varphi(\mu_k)$};
    \node (dec1) [decision, below of=pro1] {$\beta_k-\alpha_k<\varepsilon ?$ };
    \node (dec2) [decision, below of=dec1] {$\varphi(\lambda_k)-\varphi(\mu_k) ?$ };
    \node (pro2) [process, below of=dec2] {$\alpha_{k+1}=\alpha_k,\beta_{k+1}=\mu_k,\mu_{k+1}=\lambda_k$\\ $\lambda_{k+1}=\alpha_{k+1}+(1-\gamma)(\beta_{k+1}-\alpha_{k+1})$\\ $\varphi(\mu_{k+1})=\varphi(\lambda_k)$,计算 $\varphi(\lambda_{k+1})$ };

    \node (stop) [startstop, right of=dec1, xshift=3cm] {停止 $\lambda^*=\frac{\beta_k+\alpha_k}{2}$ };
    \node (pro3) [process, right of=dec2, xshift=3cm] {$\alpha_{k+1}=\lambda_k,\beta_{k+1}=\beta_k,\lambda_{k+1}=\mu_k$\\ $\mu_{k+1}=\alpha_{k+1}+\gamma(\beta_{k+1}-\alpha_{k+1})$\\ $\varphi(\lambda_{k+1})=\varphi(\mu_k)$,计算 $\varphi(\mu_{k+1})$};
    \node (pro4) [process, left of=dec2, xshift=-3cm] {$\alpha_{k+1}=\lambda_k$\\ $\beta_{k+1}=\mu_k$\\ $k=k+1$ };

    \node (pro5) [process, right of=pro1, xshift=6cm, yshift=-1.5cm] {$k=k+1$ };
    \node (point1) [point, below of=start, yshift=1.5cm] {};
    \node (point2) [point, below of=pro1, yshift=1.5cm] {};


    % Arrows
    \draw [arrow] (start) -- (pro1);
    \draw [arrow] (pro1) -- (dec1);
    \draw [arrow] (dec1) -- node[right]{N}(dec2);
    \draw [arrow] (dec2) -- node[right]{$<0$} (pro2);
    \draw [arrow] (dec1) -- node[above]{Y} (stop);
    \draw [arrow] (dec2) -- node[above]{$>0$} (pro3);
    \draw [arrow] (dec2) -- node[above]{$=0$} (pro4);
    \draw [arrow] (pro4) |- (point1);
    \draw [arrow] (pro5) -- (point2);
    \draw [arrow] (pro3) -| (pro5);
    \draw [arrow] (pro2) -| (pro5);

\end{tikzpicture}


    \caption{黄金分割法流程}
    \label{fig:flow of Golden section method}
\end{figure}

\subsection{牛顿法和插值法}

\subsubsection{牛顿法}

牛顿法的基本思想是通过迭代逼近一个函数的最小值点。具体来说,该方法使用函数的二阶导数(也称为Hessian矩阵)来构建一个二次逼近,然后通过解这个二次逼近的极小值问题来更新当前的估计值。

主要用于已经���道 $\lambda_k$ 求 $\lambda^*$ 。

设函数 $\varphi(\lambda)$ 二次可微,考虑 $\varphi(\lambda)$ 在 $\lambda_k$ 处的泰勒展开
$$
    \varphi(\lambda)=\varphi(\lambda_k)+\varphi^{\prime}(\lambda_k)(\lambda-\lambda_k)+\frac{1}{2}\varphi^{\prime\prime}(\lambda_k)(\lambda-\lambda_k)^2+o(\lambda-\lambda_k)^2
$$
舍去高阶项,得到二次式
\begin{equation}
    q(\lambda)=\varphi(\lambda_k)+\varphi^{\prime}(\lambda_k)(\lambda-\lambda_k)+\frac{1}{2}\varphi^{\prime\prime}(\lambda_k)(\lambda-\lambda_k)^2
    \label{equ:Quadratic form of Newton's method}
\end{equation}

取 $q(\lambda)$ 作为 $\varphi(\lambda)$ 在 $\lambda_k$ 邻域的近视,当 $\varphi^{\prime\prime}(\lambda_k)>0$ 时,$q(\lambda)$ 的驻点即为极小(即为凸函数),作为新的迭代点。对式 \ref{equ:Quadratic form of Newton's method} 求导,并令其为零
$$
    q^{\prime}(\lambda)=\varphi^{\prime}(\lambda_k)+\varphi^{\prime\prime}(\lambda_k)(\lambda-\lambda_k)=0
$$
当 $\varphi^{\prime\prime}(\lambda_k)>0$ 时,解之得到牛顿迭代公式$\star$
\begin{equation}
    \lambda_{k+1}=\lambda_k-\frac{\varphi^{\prime}(\lambda_k)}{\varphi^{\prime\prime}(\lambda_k)}
    \label{equ:Newton's method}
\end{equation}

牛顿法的主要特点是二阶收敛,但是局部收敛。而且它的收敛速度很快。

\begin{figure}
    \centering
    \tikzstyle{startstop} = [rectangle, rounded corners, minimum width = 2cm, minimum height=1cm,text centered, draw = black, align=center]
\tikzstyle{process} = [rectangle, minimum width=3cm, minimum height=1cm, text centered, draw=black, align=center]
\tikzstyle{decision} = [diamond, aspect = 3, text centered, draw=black, align=center]
\tikzstyle{point}=[coordinate]
% 箭头形式
\tikzstyle{arrow} = [->,>=stealth]

\begin{tikzpicture}[node distance=2cm,line width=1pt]
    \node (start) [startstop] {初始 $\lambda,\varepsilon_1,\varepsilon_2,k=1$ };
    \node (dec1) [decision, below of=start] {$|\varphi^{\prime}(\lambda_k)|<\varepsilon_1 ?$ };
    \node (stop1) [startstop, right of=dec1, xshift=3cm] {停止,解$\lambda_k$ };
    \node (dec2) [decision, below of=dec1] {$\varphi^{\prime\prime}(\lambda_k)>0 ?$ };
    \node (stop2) [startstop, left of=dec2, xshift=-3cm] {停止};
    \node (pro1) [process, below of=dec2] {$\lambda_{k+1}=\lambda_k-\frac{\varphi^{\prime}(\lambda_k)}{\varphi^{\prime\prime}(\lambda_k)}$\\ $\varphi(\lambda_{k+1})$ };
    \node (dec3) [decision, right of=pro1, xshift=3cm] {$|\lambda_{k+1}-\lambda_k|<\varepsilon_2 ?$ };

    \node (pro2) [process, right of=dec2, xshift=6cm] {$k=k+1$ };
    \node (point) [point, below of=start, yshift=1.1cm] {};

    % arrows
    \draw [arrow] (start) -- (dec1);
    \draw [arrow] (dec1) -- node[above]{Y} (stop1);
    \draw [arrow] (dec1) -- node[right]{N} (dec2);
    \draw [arrow] (dec2) -- node[above]{N} (stop2);
    \draw [arrow] (dec2) -- node[right]{Y} (pro1);
    \draw [arrow] (pro1) -- (dec3);
    \draw [arrow] (dec3) -- node[right]{Y} (stop1);
    \draw [arrow] (dec3) -| node[right]{N} (pro2);
    \draw [arrow] (pro2) |- (point);

\end{tikzpicture}
    \caption{牛顿法流程}
    \label{fig:flow of Newton's method}
\end{figure}

\subsubsection{插值法}

插值法的基本思想是利用插值函数逼近所需求解的目标函数,然后求出插值函数的极小值点,作为新的迭代点。
常见的有三点二次插值、二点二次插值和三次插值多项式

(感觉不是很重要,具体见课本)

设 $\varphi(\lambda):\mathbf{R}\rightarrow \mathbf{R}$
\begin{enumerate}
    \item 三点二次插值。取 $\lambda_0,\lambda_1,\lambda_2$ ,得到各函数值 $\varphi_0=\varphi(\lambda_0), \varphi_1=\varphi(\lambda_1),\varphi_2=\varphi(\lambda_2)$ ,构造二次插值多项式:$\psi (\lambda)=a\lambda^2+b\lambda+c$ 有
          $$\begin{cases}
                  a\lambda_0^2+b\lambda_0+c=\varphi_0 \\
                  a\lambda_1^2+b\lambda_1+c=\varphi_1 \\
                  a\lambda_2^2+b\lambda_2+c=\varphi_2
              \end{cases}$$
          解得
          $$\begin{cases}
                  a=-\frac{(\lambda_0-\lambda_1)\varphi_2+(\lambda_1-\lambda-2)\varphi_0+(\lambda_2-\lambda_0)\varphi_1}{(\lambda_0-\lambda_1)(\lambda_1-\lambda_2)(\lambda_2-\lambda_0)}            \\
                  b=\frac{(\lambda_0^2-\lambda_1^2)\varphi_2+(\lambda_1^2-\lambda_2^2)\varphi_0+(\lambda_2^2-\lambda_0^2)\varphi_1}{(\lambda_0-\lambda_1)(\lambda_1-\lambda_2)(\lambda_2-\lambda_0)} \\
              \end{cases}$$
          有极小值点为 $$\bar{\lambda}=-\frac{b}{2a}$$
    \item 二点二次插值。
    \item 三次插值。
\end{enumerate}

\subsection{不精确的一维搜索}
% TODO: fix 一维搜索的不精确性 ref:https://www.bilibili.com/video/BV1M24y1v7Qc?t=5246.0&p=6

在实际搜索中,往往不需要特别精确,只要满足一定的条件即可,这样可以减少计算量。

\subsubsection{Goldstein 法}

设 $(fS)$ ,在 $\boldsymbol{x}^{(k)}$ 取方向 $\boldsymbol{d}^{(k)}$ ,有 $f^T(\boldsymbol{x}^{(k)})\boldsymbol{d}^{(k)}<0$ ,求解下面的问题
\begin{align*}
    (L-R)\begin{cases}
             \min\quad & f(\boldsymbol{x}^{(k)}+\lambda\boldsymbol{d}^{(k)}) \\
             s.t.\quad & \lambda\ge 0
         \end{cases}
\end{align*}

那么有 $\boldsymbol{x}^{(k+1)}=\boldsymbol{x}^{(k)}\lambda_k\boldsymbol{d}^{(k)}$,这里令 $\boldsymbol{s}^{(k)}=\boldsymbol{x}^{(k+1)}-\boldsymbol{x}=\lambda_k\boldsymbol{d}^{(k)}$。

Goldstein法遵循下面的规则:
\begin{equation}
    f(\boldsymbol{x}^{(k+1)})-f(\boldsymbol{x}^{(k)})\le\rho\nabla f^T(\boldsymbol{x}^{(k)})\boldsymbol{d}^{(k)}\boldsymbol{s}^{(k)}
    \label{equ:Goldstein rule 1}
\end{equation}
\begin{equation}
    f(\boldsymbol{x}^{(k+1)})-f(\boldsymbol{x}^{(k)})\ge(1-\rho)\nabla f^T(\boldsymbol{x}^{(k)})\boldsymbol{d}^{(k)}\boldsymbol{s}^{(k)}
    \label{equ:Goldstein rule 2}
\end{equation}
其中,$\rho\in(0,1/2)$,实际经验中常取 $\rho=0.1$ 或更小。

这里式\ref{equ:Goldstein rule 1}表示 $\lambda_k$ 取值应使 $f(\boldsymbol{x}^{(k+1)})$ 的值在直线 $y=f(\boldsymbol{x}^{(k)})+\rho\nabla f^T(\boldsymbol{x}^{(k)})\boldsymbol{d}^{(d)}\lambda$下方;式 \ref{equ:Goldstein rule 2} 表示 $\lambda_k$ 取值应使 $f(\boldsymbol{x}^{(k+1)})$ 的值在直线 $y=f(\boldsymbol{x}^{(k)})+(1-\rho)\nabla f^T(\boldsymbol{x}^{(k)})\boldsymbol{d}^{(d)}\lambda$上方。

几何意义如图\ref{fig:Axis of Goldstein method}所示,其中 $\nabla f^T(\boldsymbol{x}^{(k)})\boldsymbol{s}^{(k)}=\nabla f^T(\boldsymbol{x}^{(k)})\boldsymbol{d}^{(k)}\lambda_k$。

\begin{figure}[ht]
    \centering
    \begin{tikzpicture}
    % 坐标轴
    \draw[->] (0,0) -- (15,0) node[right] {$\lambda$};
    \draw[->] (0,0) -- (0,6) node[above] {$f$};
    \node[below] at (0,0) {O};

    % 函数图像
    \draw[thick] plot[smooth] coordinates {(0,5) (3,1.5) (9,4) (11.5,2) (14.5,3)};

    % 绘制直线
    \draw[line width=1pt] (0,5) -- (9,-0.5) node[below] {$y=f(\boldsymbol{x}^{(k)})+(1-\rho)\nabla f^T(\boldsymbol{x}^{(k)})\boldsymbol{d}^{(k)}\lambda$};
    \draw[line width=1pt, domain=0:15] (0,5)  -- (15,2);

    \node[above] at (2.5,5.2) {$y=f(\boldsymbol{x}^{(k)})+\rho\nabla f^T(\boldsymbol{x}^{(k)})\boldsymbol{d}^{(k)}\lambda$};
    \node[above] at (9,4.5) {$y=f(\boldsymbol{x}^{(k+1)}+\lambda\boldsymbol{d}^{(k)})$};

    \draw (4.83,0) -- (4.83,2.05);
    \draw (7.58,0) -- (7.58,3.5);
    \draw (10.54,0) -- (10.54,2.9);
    \draw (13.1,0) -- (13.1,2.4);

    \fill[pattern=north west lines] (4.83,0) rectangle (7.58,0.2);
    \node[anchor=north] at (6.205, -0.1) {可接受范围};
    \fill[pattern=north west lines] (10.54,0) rectangle (13.1,0.2);
    \node[anchor=north] at (11.82, -0.1) {可接受范围};

\end{tikzpicture}
    \caption{Goldstein法几何意义}
    \label{fig:Axis of Goldstein method}
\end{figure}

\begin{figure}[ht]
    \centering
    \tikzstyle{startstop} = [rectangle, rounded corners, minimum width = 2cm, minimum height=1cm,text centered, draw = black, align=center]
\tikzstyle{process} = [rectangle, minimum width=3cm, minimum height=1cm, text centered, draw=black, align=center]
\tikzstyle{decision} = [diamond, aspect = 3, text centered, draw=black, align=center]
\tikzstyle{point}=[coordinate]
% 箭头形式
\tikzstyle{arrow} = [->,>=stealth]

\begin{tikzpicture}[node distance=2cm,line width=1pt]
    \node (start) [startstop] {给$\rho\in(0,\frac{1}{2},\alpha>1,\beta\in(0,1))$\\初始$\lambda,\varphi_1=f(\boldsymbol{x}^{(k)}),\varphi^{\prime}_1=\nabla f^T(\boldsymbol{x}^{(k)})\boldsymbol{d}^{(k)}$};
    \node (pro1) [process, below of=start] {$\varphi_2=f(\boldsymbol{x}^{(k)}+\lambda\boldsymbol{d}^{(k)})$};
    \node (dec1) [decision, below of=pro1] {$\varphi_2\le\varphi_1+\rho\varphi_1^{\prime}\lambda$?};
    \node (dec2) [decision, below of=dec1] {$\varphi_2\ge\varphi_1+(1-\rho)\varphi_1^{\prime}\lambda$?};
    \node (pro2) [process, right of=dec1, xshift=3cm] {$\lambda=\beta\lambda$};
    \node (pro3) [process, right of=dec2, xshift=3cm] {$\lambda=\alpha\lambda$};
    \node (stop) [startstop, left of=dec2, xshift=-3cm] {停,解$\lambda_k$};
    \node (point1) [point, right of=pro1, xshift=5cm, yshift=1cm] {};
    \node (point2) [point, below of=start, yshift=1cm] {};

    % arrows
    \draw [arrow] (start) -- (pro1);
    \draw [arrow] (pro1) -- (dec1);
    \draw [arrow] (dec1) -- node[right]{Y} (dec2);
    \draw [arrow] (dec1) -- node[above]{N} (pro2);
    \draw [arrow] (dec2) -- node[above]{N} (pro3);
    \draw [arrow] (dec2) -- node[above]{Y} (stop);
    \draw [-] (pro2) -| (point1);
    \draw [-] (pro3) -| (point1);
    \draw [arrow] (point1) -- (point2);

\end{tikzpicture}
    \caption{Goldstein法流程}
    \label{fig:flow of Goldstein method}
\end{figure}

\subsubsection{Wolfe-Powell法}

考虑从 $\boldsymbol{x}^{(k)}$ ,沿方向 $\boldsymbol{d}^{(k)}$ 搜索新的迭代点 $\boldsymbol{x}^{(k)}+\lambda_k\boldsymbol{d}^{(k)}$

有规则 Wolfe-Powell :
\begin{equation}
    f(\boldsymbol{x}^{(k+1)})\le f(\boldsymbol{x}^{(k)})+\rho\nabla f^T(\boldsymbol{x}^{(k)})\boldsymbol{s}^{(k)}
    \label{equ:Wolfe-Powell rule 1}
\end{equation}
\begin{equation}
    \nabla f(\boldsymbol{x}^{(k+1)})\boldsymbol{s}^{(k)}\ge\sigma\nabla f^T(\boldsymbol{x}^{(k)})\boldsymbol{s}^{(k)}
    \label{equ:Wolfe-Powell rule 2}
\end{equation}

\begin{figure}[ht]
    \centering
    
\begin{tikzpicture}
    % 坐标轴
    \draw[->] (0,0) -- (15,0) node[right] {$\lambda$};
    \draw[->] (0,0) -- (0,6) node[above] {$f$};
    \node[below] at (0,0) {O};

    % 函数图像
    \draw[thick] plot[smooth] coordinates {(0,5) (3,1.5) (9,4) (11.7,2) (14.5,3)};

    % 绘制斜率
    \draw (1,2.6) -- (4,0.8);
    \node[left] at (2.5,1.65) {斜率为$\beta$};
    \draw (9.95,3) -- (12.95,1.2);
    \node[left] at (11.4,2.12) {斜率为$\beta$};

    % 绘制直线
    % \draw[line width=1pt] (0,5) -- (9,-0.5) node[below] {$y=f(\boldsymbol{x}^{(k)})+(1-\rho)\nabla f^T(\boldsymbol{x}^{(k)})\boldsymbol{d}^{(k)}\lambda$};
    \draw[line width=1pt, domain=0:15] (0,5)  -- (15,2);
    \node[above] at (4.3,4.5) {$y=f(\boldsymbol{x}^{(k)})+\rho\nabla f^T(\boldsymbol{x}^{(k)})\boldsymbol{d}^{(k)}\lambda$};

    \node[above] at (.7,5) {$f(\boldsymbol{x}^{(k)})$};
    \node[above] at (9,4.1) {$f(\boldsymbol{x}^{(k+1)}+\lambda\boldsymbol{d}^{(k)})$};

    \draw (2.55,0) -- (2.55,1.65);
    \draw (7.58,0) -- (7.58,3.5);
    \draw (11.45,0) -- (11.45,2.12);
    \draw (13.2,0) -- (13.2,2.38);

    \fill[pattern=north west lines] (2.55,0) rectangle (7.58,0.2);
    \node[anchor=north] at (5.065, -0.1) {可接受范围};
    \fill[pattern=north west lines] (11.45,0) rectangle (13.2,0.2);
    \node[anchor=north] at (12.325, -0.1) {可接受范围};

    \node[above] at (7.55,5.7) {斜率$\beta=\rho\nabla f^T(\boldsymbol{x}^{(k+1)})\boldsymbol{d}^{(k)}$};

\end{tikzpicture}


    \caption{Wolfe-Powell法几何意义}
    \label{fig:Axis of Wolfe-Powell method}
\end{figure}

\chapter{无约束最优化方法}

无约束最优化问题的,是指问题 $(fS)$ 中 $S=\mathbf{R}^n$ 的情况。为了简便,记无约束最优化问题为
$$
    (f)\ \min f(\boldsymbol{x})
$$
其中,$f:\mathbf{R}\rightarrow\mathbf{R}$

\section{最优性条件}

\section{最速下降法}

\section{牛顿法及其修正}

\section{共轭梯度法}

\end{document}