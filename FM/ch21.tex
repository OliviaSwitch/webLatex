\chapter{重积分}

\section{二重积分}

\subsection{二重积分的概念}

\begin{definition}
    设 $f(x,y)$ 在有界闭区域 $D$ 上有界,将 $D$ 划分为 $n$ 个小区域 $D_{ij}$,在每个小区域 $D_{ij}$ 取一点 $(\xi_{ij},\eta_{ij})$,作积分和
    \[
        \sum_{i=1}^{m} \sum_{j=1}^{n} f(\xi_{ij},\eta_{ij}) \Delta \delta_{ij}
    \]
    如果当小区域的直径趋于零时,这个积分和的极限存在,且与划分方法和点的选取无关,那么称此极限为 $f(x,y)$ 在区域 $D$ 上的二重积分,记作
    \[
        \iint_{D} f(x,y) \, \mathrm{d}\delta
    \]
\end{definition}

\subsection{累次积分}

\begin{theorem}
    设 $f(x,y)$ 在有界闭区域 $D$ 上有界,且 $D$ 的边界为 $a \leq x \leq b, \varphi_{1}(x) \leq y \leq \varphi_{2}(x)$,则
    \[
        \iint_{D} f(x,y) \, \mathrm{d}\delta = \int_{a}^{b} \left( \int_{\varphi_{1}(x)}^{\varphi_{2}(x)} f(x,y) \, \mathrm{d}y \right) \, \mathrm{d}x
    \]
\end{theorem}

\subsection{二重积分的性质}

\begin{enumerate}
    \item 线性性质:设 $f(x,y)$ 和 $g(x,y)$ 在区域 $D$ 上有界,$k_{1},k_{2}$ 为常数,则
          \[
              \iint_{D} (\alpha f(x,y) + \beta g(x,y)) \, \mathrm{d}\delta = \alpha \iint_{D} f(x,y) \, \mathrm{d}\delta + \beta \iint_{D} g(x,y) \, \mathrm{d}\delta
          \]
    \item 区域可加性:设 $D$ 可表示为两个无交区域 $D_{1},D_{2}$ 的并,$f(x,y)$ 在 $D$ 上有界,则
          \[
              \iint_{D} f(x,y) \, \mathrm{d}\delta = \iint_{D_{1}} f(x,y) \, \mathrm{d}\delta + \iint_{D_{2}} f(x,y) \, \mathrm{d}\delta
          \]
    \item 保号性:设 $f(x,y)$ 在区域 $D$ 上有界,且 $f(x,y) \geq 0$,则
          \[
              \iint_{D} f(x,y) \, \mathrm{d}\delta = 0 \iff f(x,y) = 0
          \]
          \[
              f(x,y)\equiv1,\ \iint_{D} 1 \, \mathrm{d}\delta = \delta
          \]
    \item 绝对值不等式:设 $f(x,y)$ 在区域 $D$ 上有界,则
          \[
              \left| \iint_{D} f(x,y) \, \mathrm{d}\delta \right| \leq \iint_{D} |f(x,y)| \, \mathrm{d}\delta
          \]
          \[
              f(x,y)\le g(x,y)\Rightarrow \iint_{D}f(x,y) \, \mathrm{d}\delta\le\iint_{D}g(x,y) \, \mathrm{d}\delta
          \]
    \item 设 $M,m$ 分别为 $f(x,y)$ 在区域 $D$ 上的最大值和最小值,则
          \[
              m\delta\le\iint_{D}f(x,y) \, \mathrm{d}\delta\le M\delta
          \]

\end{enumerate}

\subsection{二重积分的计算}

\begin{enumerate}
    \item 若积分区域 $D$ 为矩形区域,且 $f(x,y)$ 在 $D$ 上连续,则
          \[
              \iint_{D} f(x,y) \, \mathrm{d}\delta = \int_{a}^{b} \left( \int_{c}^{d} f(x,y) \, \mathrm{d}y \right) \, \mathrm{d}x
          \]
          若 $f(x,y) = f_1(x)f_2(y)$
          \[\begin{aligned}
                  \iint_{D} f(x,y) \, \mathrm{d}\delta & = \int_{a}^{b} \, \mathrm{d}x \int_{c}^{d} f_1(x)f_2(y) \, \mathrm{d}y = \int_{a}^{b} f_1(x) \, \mathrm{d}x \int_{c}^{d} f_2(y) \, \mathrm{d}y \\
                                                       & = \left( \int_{a}^{b} f_1(x) \, \mathrm{d}x \right) \left( \int_{c}^{d} f_2(y) \, \mathrm{d}y \right)
              \end{aligned}\]
    \item 若积分区域 $D$ 为三角形区域,且 $f(x,y)$ 在 $D$ 上连续,则
          \[
              \iint_{D} f(x,y) \, \mathrm{d}\delta = \int_{a}^{b} \left( \int_{\varphi_{1}(x)}^{\varphi_{2}(x)} f(x,y) \, \mathrm{d}y \right) \, \mathrm{d}x
          \]
\end{enumerate}

\subsection{二重积分的变量代换}

\begin{enumerate}
    \item 设 $x=x(u,v),y=y(u,v)$ 为区域 $D$ 到区域 $D'$ 的一一映射,且满足
          \[
              \begin{cases}
                  x=x(u,v) \\
                  y=y(u,v)
              \end{cases}
          \]
          为 $D$ 到 $D'$ 的可逆变换,且 $x(u,v),y(u,v)$ 具有一阶连续偏导数,则
          \[
              \iint_{D} f(x,y) \, \mathrm{d}\delta = \iint_{D'} f(x(u,v),y(u,v)) \left| J \right| \, \mathrm{d}\delta'
          \]
          其中 $J = \left| \frac{\partial(x,y)}{\partial(u,v)} \right| = \left| \frac{\partial x}{\partial u} \frac{\partial y}{\partial v} - \frac{\partial x}{\partial v} \frac{\partial y}{\partial u} \right|$
    \item 设 $x=x(u),y=y(v)$ 为区域 $D$ 到区域 $D'$ 的一一映射,且满足
          \[
              \begin{cases}
                  x=x(u) \\
                  y=y(v)
              \end{cases}
          \]
          为 $D$ 到 $D'$ 的可逆变换,且 $x(u),y(v)$ 具有一阶连续偏导数,则
          \[
              \iint_{D} f(x,y) \, \mathrm{d}\delta = \iint_{D'} f(x(u),y(v)) \left| J \right| \, \mathrm{d}\delta'
          \]
          其中 $J = \left| \frac{\mathrm{d}x}{\mathrm{d}u} \frac{\mathrm{d}y}{\mathrm{d}v} \right|$,$\mathrm{d}\delta' = \left| \frac{\mathrm{d}x}{\mathrm{d}u} \frac{\mathrm{d}y}{\mathrm{d}v} \right| \, \mathrm{d}u \, \mathrm{d}v$
\end{enumerate}

\subsubsection{极坐标变换}

设 $x=r\cos\theta,y=r\sin\theta$,则
\[
    \begin{aligned}
        \iint_{D} f(x,y) \, \mathrm{d}\delta & = \iint_{D'} f(r\cos\theta,r\sin\theta) \left| J \right| \, \mathrm{d}\delta' \\
                                             & = \iint_{D'} f(r\cos\theta,r\sin\theta) r \, \mathrm{d}r \, \mathrm{d}\theta
    \end{aligned}
\]

\section{格林公式}

\begin{theorem}
    设 $D$ 是平面区域,$P(x,y),Q(x,y)$ 在 $D$ 上有连续偏导数,则
    \[
        \iint_{D} \left( \frac{\partial Q}{\partial x} - \frac{\partial P}{\partial y} \right) \, \mathrm{d}x \mathrm{d}y = \oint_{L} P \, \mathrm{d}x + Q \, \mathrm{d}y
    \]
\end{theorem}

这里 $L$ 是区域 $D$ 的边界,按照逆时针方向取正向。

\begin{quote}[口诀]
    交换相减反求偏导,交叉相乘积分加。
\end{quote}

面积公式:
\[
    \iint_{D} 1 \, \mathrm{d}\delta = \frac{1}{2} \oint_{L} x \, \mathrm{d}y - y \, \mathrm{d}x
\]