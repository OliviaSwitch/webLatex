\chapter{多元函数的极限与连续}

\section{基本概念}

平面:$\mathbf{R}^2=\mathbf{R}\times\mathbf{R}=\{(x,y)|x,y\in\mathbf{R}\}$

平面点集:$\{(x,y)|(x,y)\hbox{满足条件}P\}$

邻域:$U(P_0,\delta)=\{P||PP_0|<\delta\}$

内点:$P_0$是集合$D$的内点,如果存在$\delta>0$,使得$U(P_0,\delta)\subset D$

外点:$P_0$是集合$D$的外点,如果存在$\delta>0$,使得$U(P_0,\delta)\cap D=\emptyset$

(边)界点:$P_0$是集合$D$的边界点,如果对任意$\delta>0$,$U(P_0,\delta)$内既有$D$内的点,也有$D$外的点

聚点:对任意$\delta>0$,$U(P_0,\delta)$内有$D$内的点

开集:集合$D$中的每一点都是$D$的内点,如$(a,b)$

闭集:集合$D$中的每一个边界点都是$D$的点,如$[a,b]$

开域:联通的开集

闭域:联通的闭集

有界集:集合$D$内的点都在某一邻域内

无界集:集合$D$内的点没有界限约束

联通集:集合$D$内的任意两点都可以用$D$内的折线连接

\section{二元函数的极限}

称$f$在$D$上当$P\to P_0$时以$A$为极限, 记
$$\lim_{P\to P_0}f(P)=A$$

当$P,P_0$分别用坐标$(x, y),(x_0, y_0)$表示时, 上式也常写作
$$\lim_{(x,y)\to(x_0,y_0)}f(x,y)=A$$

多元函数的逼近可以沿着任何一条路径进行, 但是极限只有一个, 与逼近的路径无关。如果极限不相等, 则称多元函数在该点无极限。