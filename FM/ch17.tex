\chapter{多元函数微分学}

\section{可微性}

\subsection{偏导数}

\begin{definition}
    设函数$z=f(x,y)$在点$(x_0,y_0)$的某邻域内有定义,当$x$在$x_0$处有增量$\Delta x$,$y$在$y_0$处有增量\\$\Delta y$时,相应的函数有增量$\Delta z=f(x_0+\Delta x,y_0+\Delta y)-f(x_0,y_0)$,如果极限
        \[
            \lim_{\Delta x\to 0}\frac{\Delta z}{\Delta x}=\lim_{\Delta x\to 0}\frac{f(x_0+\Delta x,y_0+\Delta y)-f(x_0,y_0)}{\Delta x}
        \]
        存在,则称此极限为函数$z=f(x,y)$在点$(x_0,y_0)$处对$x$的偏导数,记作
        \[
            \left.\frac{\partial z}{\partial x}\right|_{(x_0,y_0)}\quad\text{或}\quad f'_x(x_0,y_0)\quad\text{或}\quad z'_x
        \]
        同理可得函数$z=f(x,y)$在点$(x_0,y_0)$处对$y$的偏导数。
\end{definition}

怎么求:
\begin{itemize}
    \item 对$x$的偏导数:将$y$看作常数,对$x$求导;
    \item 对$y$的偏导数:将$x$看作常数,对$y$求导。
\end{itemize}

\subsubsection{关于连续性}

\begin{enumerate}
    \item 对于一元函数,可导必定连续
    \item 对于多元函数,偏导数存在不一定连续
\end{enumerate}

\subsection{全微分}

\begin{definition}
    设函数$z=f(x,y)$在点$(x_0,y_0)$的某邻域内有定义,且在该点有偏导数,则称函数\\$z=f(x,y)$在点$(x_0,y_0)$处可微分,如果存在常数$A$和$B$,使得全增量
        \[
            \Delta z=A\Delta x+B\Delta y+o(\rho)
        \]
        其中$\rho=\sqrt{(\Delta x)^2+(\Delta y)^2}$,则称$A\Delta x+B\Delta y$为函数$z=f(x,y)$在点$P_0=(x_0,y_0)$处的全微分,记作
        \[
            \left.\mathrm{d}z\right|_{P_0}=df(x_0,y_0)=A\Delta x+B\Delta y
        \]
        当$\Delta x$和$\Delta y$趋于零时,全微分$\mathrm{d}z$可作为全增量$\Delta z$的近似值,于是有近似公式
    \[
        f(x,y)\approx f(x_0,y_0)+A(x-x_0)+B(y-y_0)
    \]
\end{definition}

\subsubsection{可微性条件}

% \begin{theorem}
%     函数$z=f(x,y)$在点$(x_0,y_0)$处可微分的充分必要条件是:函数$z=f(x,y)$在点$(x_0,y_0)$处的偏导数存在且连续。
% \end{theorem}

\begin{theorem}
    若二元函数 $f$ 在其定义域内一点 $(x_0,y_0)$ 处可微,则 $f$ 在该点关于每个自变量的偏导数都存在。
    此时,全微分可写成
    \[
        \mathrm{d}f(x,y)=f_x(x,y)\mathrm{d}x+f_y(x,y)\mathrm{d}y
    \]
\end{theorem}

\begin{theorem}[可微的充分条件]
    若函数 $z=f(x,y)$ 在点 $(x_0,y_0)$ 处的偏导数 $f_x(x_0,y_0)$ 和 $f_y(x_0,y_0)$ 存在且连续,则 $f$ 在该点可微。
\end{theorem}

另外,连续是可微的一个必要条件。

\subsection{曲面的切平面与法线}

\begin{definition}
    设曲面 $z=f(x,y)$ 在点 $(x_0,y_0,z_0)$ 处可微,且 $f_x(x_0,y_0)\neq 0$,则曲面在该点的切平面方程为
    \[
        z-z_0=f_x(x_0,y_0)(x-x_0)+f_y(x_0,y_0)(y-y_0)
    \]
\end{definition}

同理,有曲面 $F(x,y,z)=0$ 在点 $(x_0,y_0,z_0)$ 处可微,则曲面在该店的切平面方程为
\[
    F_x(x_0,y_0,z_0)(x-x_0)+F_y(x_0,y_0,z_0)(y-y_0)+F_z(x_0,y_0,z_0)(z-z_0)=0
\]

\begin{definition}
    设曲面 $z=f(x,y)$ 在点 $(x_0,y_0,z_0)$ 处可微,且 $f_x(x_0,y_0)\neq 0$,则曲面在该点的法线方程为
    \[
        \frac{x-x_0}{f_x(x_0,y_0)}=\frac{y-y_0}{f_y(x_0,y_0)}=\frac{z-z_0}{-1}
    \]
\end{definition}

同理,有曲面 $F(x,y,z)=0$ 在点 $(x_0,y_0,z_0)$ 处可微,则曲面在该店的法线方程为
\[
    \frac{x-x_0}{F_x(x_0,y_0,z_0)}=\frac{y-y_0}{F_y(x_0,y_0,z_0)}=\frac{z-z_0}{F_z(x_0,y_0,z_0)}
\]


