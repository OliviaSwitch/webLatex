\chapter{多元函数微分学}

\section{偏导数}

\begin{definition}
    设函数$z=f(x,y)$在点$(x_0,y_0)$的某邻域内有定义,当$x$在$x_0$处有增量$\Delta x$,$y$在$y_0$处有增量$\Delta y$时,相应的函数有增量$\Delta z=f(x_0+\Delta x,y_0+\Delta y)-f(x_0,y_0)$,如果极限
    \[
        \lim_{\Delta x\to 0}\frac{\Delta z}{\Delta x}=\lim_{\Delta x\to 0}\frac{f(x_0+\Delta x,y_0+\Delta y)-f(x_0,y_0)}{\Delta x}
    \]
    存在,则称此极限为函数$z=f(x,y)$在点$(x_0,y_0)$处对$x$的偏导数,记作
    \[
        \left.\frac{\partial z}{\partial x}\right|_{(x_0,y_0)}\quad\text{或}\quad f'_x(x_0,y_0)\quad\text{或}\quad z'_x
    \]
    同理可得函数$z=f(x,y)$在点$(x_0,y_0)$处对$y$的偏导数。
\end{definition}

怎么求:
\begin{itemize}
    \item 对$x$的偏导数:将$y$看作常数,对$x$求导;
    \item 对$y$的偏导数:将$x$看作常数,对$y$求导。
\end{itemize}

\subsection{关于连续性}

\begin{enumerate}
    \item 对于一元函数,可导必定连续
    \item 对于多元函数,偏导数存在不一定连续
\end{enumerate}