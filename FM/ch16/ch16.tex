\chapter{多元函数的极限与连续}

\section{基本概念}

平面:$\mathbf{R}^2=\mathbf{R}\times\mathbf{R}=\{(x,y)|x,y\in\mathbf{R}\}$

平面点集:$\{(x,y)|(x,y)\hbox{满足条件}P\}$

邻域:$U(P_0,\delta)=\{P||PP_0|<\delta\}$

内点:$P_0$是集合$D$的内点,如果存在$\delta>0$,使得$U(P_0,\delta)\subset D$

外点:$P_0$是集合$D$的外点,如果存在$\delta>0$,使得$U(P_0,\delta)\cap D=\emptyset$

(边)界点:$P_0$是集合$D$的边界点,如果对任意$\delta>0$,$U(P_0,\delta)$内既有$D$内的点,也有$D$外的点

聚点:对任意$\delta>0$,$U(P_0,\delta)$内有$D$内的点

开集:集合$D$中的每一点都是$D$的内点,如$(a,b)$

闭集:集合$D$中的每一个边界点都是$D$的点,如$[a,b]$

开域:联通的开集

闭域:联通的闭集

有界集:集合$D$内的点都在某一邻域内

无界集:集合$D$内的点没有界限约束

联通集:集合$D$内的任意两点都可以用$D$内的折线连接

\section{二元函数的极限}

称$f$在$D$上当$P\to P_0$时以$A$为极限, 记
$$\lim_{P\to P_0}f(P)=A$$

当$P,P_0$分别用坐标$(x, y),(x_0, y_0)$表示时, 上式也常写作
$$\lim_{(x,y)\to(x_0,y_0)}f(x,y)=A$$

多元函数的逼近可以沿着任何一条路径进行, 但是极限只有一个, 与逼近的路径无关。如果极限不相等, 则称多元函数在该点无极限。

\subsection{重极限与累次极限}

在上面讨论的$\lim_{(x,y)\to(x_0,y_0)}f(x,y)=A$中, 自变量 $(x, y)$是以任何方式趋于 $(x_0, y_0)$的, 这种极限也称为重极限。

而$x$与$y$依一定的先后顺序, 相继趋于 $x_0$ 与 $y_0$ 时 $f$ 的极限, 这种极限称为累次极限。
若对每一个$y\in Y(y\not y_0)$,存在极限$\lim_{x\to x_0}f(x,y)$,它一般与$y$有关,记作
$$\varphi(y)=\lim_{x\to x_0}f(x,y)$$
如果进一步还存在极限
$$L=\lim_{y\to y_0}\varphi(y)$$
则称此$L$为$f(x,y)$先对$x(x\to x_0)$后对$y(y\to y_0)$的累次极限,记作
$$L=\lim_{y\to y_0}\lim_{x\to x_0}f(x,y)$$

{\theorem{如果$f(x,y)$的重极限$\lim\limits_{(x,y)\to(x_0,y_0)}f(x,y)$与累次极限$\lim\limits_{y\to y_0}\lim\limits_{x\to x_0}f(x,y)$都存在,则两者必定相等。}}

\subsubsection{$\varepsilon-\delta$定义}

对于任何正数$\varepsilon$,都能够找到一个正数$\delta$,当x满足${0<\mid x-a\mid <\delta }$时,对于满足上式的x都有\\ ${0<\mid f(x)-b\mid <\varepsilon }$。

\section{二元函数的连续性}

和一元函数相似,二元函数的连续性也有以下三种定义:
$$\lim_{ \substack{x\to x_0 \\ y\to y_0} }f(x,y)=f(x_0,y_0)$$% 双行下标 \substack{ \\ } 
\begin{enumerate}
    \item 有定义
    \item 有极限
    \item 极限等于函数值
\end{enumerate}
几何意义:不断开的曲面。

\subsection{复合函数的连续性}

设函数$z=f(x,y)$在点$(x_0,y_0)$的某邻域内有定义,函数$u=g(x,y)$在点$(x_0,y_0)$的某邻域内有定义,且$f(x,y)$在点$(x_0,y_0)$连续,$g(x,y)$在点$(x_0,y_0)$连续,那么复合函数$u=g(f(x,y))$在点$(x_0,y_0)$连续。

“连续函数的连续函数是连续函数”。