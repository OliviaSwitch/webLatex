\chapter{隐函数定理及其应用}

\section{隐函数}

\begin{definition}
    设方程 $F(x,y)=0$ 在点 $(x_0,y_0)$ 的某一邻域内恒有解 $y=f(x)$,且 $f(x_0)=y_0$,若 $f(x)$ 在点 $x_0$ 处可微,则称 $y=f(x)$ 为方程 $F(x,y)=0$ 在点 $(x_0,y_0)$ 处的隐函数。
\end{definition}

\subsection{隐函数定理}

\begin{theorem}[隐函数存在唯一性定理]
    设函数 $F(x,y)$ 满足下列条件
    \begin{enumerate}
        \item $F(x_0,y_0)=0$;
        \item $F(x,y)$ 在点 $(x_0,y_0)$ 的某一邻域内有连续偏导数 $F_{y}(x, y)$;
        \item $F_{y}(x_0,y_0)\neq 0$。
    \end{enumerate}
    则在点 $(x_0,y_0)$ 的某一邻域内,方程 $F(x,y)=0$ 有且仅有一个连续可微的隐函数 $y=f(x)$,满足 $F(x,f(x))=0$,且 $y_0=y(x_0)$,并有
    \begin{equation}
        \frac{\mathrm{d}y}{\mathrm{d}x}=-\frac{F^{\prime}_x}{F^{\prime}_y}
    \end{equation}
\end{theorem}

\begin{theorem}[隐函数可微性定理]
    设函数 $F(x,y)$ 满足隐函数存在唯一性定理的条件,在 $D$ 内还存在连续的 $F_{x}(x, y)$ 则由方程 $F( x, y ) = 0$ 所确定的隐函数 $y = f (x)$ 在 $I$ 内有连续的导函数,且
    \begin{equation}
        f^{\prime}(x)=-\frac{F^{\prime}_x(x,y)}{F^{\prime}_y(x,y)}
    \end{equation}
\end{theorem}

% \subsection{隐函数求导}

% \begin{theorem}[解法一:形式计算法]
%     设方程 $F(x,y)=0$ 在点 $(x_0,y_0)$ 的某一邻域内恒有解 $y=f(x)$,且 $f(x_0)=y_0$,若 $f(x)$ 在点 $x_0$ 处可微,则称 $y=f(x)$ 为方程 $F(x,y)=0$ 在点 $(x_0,y_0)$ 处的隐函数。设 $F(x,y)$ 在点 $(x_0,y_0)$ 的某一邻域内有连续偏导数 $F_{x}(x, y)$ 和 $F_{y}(x, y)$,且 $F_{y}(x_0,y_0)\neq 0$。则在点 $(x_0,y_0)$ 的某一邻域内,方程 $F(x,y)=0$ 有且仅有一个连续可微的隐函数 $y=f(x)$,满足 $F(x,f(x))=0$,且 $y_0=y(x_0)$,并有
%     \begin{equation}
%         \frac{\mathrm{d}y}{\mathrm{d}x}=-\frac{F^{\prime}_x}{F^{\prime}_y}
%     \end{equation}
% \end{theorem}

\section{隐函数组}

\begin{equation}
    \begin{cases}
        F(x,y,u,v)=0 \\
        G(x,y,u,v)=0
    \end{cases}
\end{equation}

\begin{theorem}[雅可比行列式]
    设函数 $F(x,y,u,v)$ 和 $G(x,y,u,v)$ 在点 $(x_0,y_0,u_0,v_0)$ 的某一邻域内有连续偏导数 $F_{x},F_{y},F_{u},F_{v},G_{x},G_{y},G_{u},G_{v}$,且
    \begin{equation}
        J= \frac{\partial(F,G)}{\partial(u,v)}=\begin{vmatrix}
            F_{u} & F_{v} \\
            G_{u} & G_{v}
        \end{vmatrix}\neq 0
    \end{equation}
    则在点 $(x_0,y_0,u_0,v_0)$ 的某一邻域内,方程组 $F(x,y,u,v)=0$ 和 $G(x,y,u,v)=0$ 有且仅有一个连续可微的隐函数组 $u=f(x,y)$ 和 $v=g(x,y)$,满足 $F(x,y,f(x,y),g(x,y))=0$ 和 $G(x,y,f(x,y),g(x,y))=0$,且 $u_0=u(x_0,y_0)$ 和 $v_0=v(x_0,y_0)$,并有
    \begin{equation}
        \begin{cases}
            \frac{\partial u}{\partial x}=\frac{1}{J}\frac{\partial(F,G)}{\partial(x,v)}=-\frac{\begin{vmatrix} F_{x} & F_{v} \\ G_{x} & G_{v} \end{vmatrix}}{\begin{vmatrix} F_{u} & F_{v} \\ G_{u} & G_{v} \end{vmatrix}} , \frac{\partial v}{\partial x}=\frac{1}{J}\frac{\partial(F,G)}{\partial(u,x)}=-\frac{\begin{vmatrix} F_{u} & F_{v} \\ G_{u} & G_{v} \end{vmatrix}}{\begin{vmatrix} F_{x} & F_{v} \\ G_{x} & G_{v} \end{vmatrix}} \\
            \frac{\partial u}{\partial y}=\frac{1}{J}\frac{\partial(F,G)}{\partial(y,v)}=-\frac{\begin{vmatrix} F_{u} & F_{x} \\ G_{u} & G_{x} \end{vmatrix}}{\begin{vmatrix} F_{u} & F_{v} \\ G_{u} & G_{v} \end{vmatrix}} , \frac{\partial v}{\partial y}=\frac{1}{J}\frac{\partial(F,G)}{\partial(u,y)}=-\frac{\begin{vmatrix} F_{u} & F_{v} \\ G_{u} & G_{v} \end{vmatrix}}{\begin{vmatrix} F_{u} & F_{x} \\ G_{u} & G_{x} \end{vmatrix}}
        \end{cases}
    \end{equation}
\end{theorem}