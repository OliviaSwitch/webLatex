\chapter{曲线积分}

\section{第一型曲线积分}

第一型曲线积分是对弧长的积分,它是曲线积分的最简单形式。

计算步骤:

\begin{enumerate}
    \item 画出所积曲线的示意图,并转化为定积分的形式:$\int_{L} f(x,y) \, \mathrm{d}s$
    \item 确定积分区间 $[x_1,x_2]$ 或 $[y_1,y_2]$ 或 $[t_1,t_2]$
    \item 计算 $\mathrm{d}s$ :
          \[
              \mathrm{d}s = \sqrt{1+(\frac{\mathrm{d}y}{\mathrm{d}x})^2} \, \mathrm{d}x = \sqrt{1+(\frac{\mathrm{d}x}{\mathrm{d}y})^2} \, \mathrm{d}y = \sqrt{(\frac{\mathrm{d}x}{\mathrm{d}t})^2+(\frac{\mathrm{d}y}{\mathrm{d}t})^2} \, \mathrm{d}t
          \]
    \item 将 $\mathrm{d}s$ 代入积分式中,计算积分
\end{enumerate}

% \subsection{极坐标下的曲线积分}

% 圆的参数方程:
% \[
%     \begin{cases}
%         x = r \cos \theta \\
%         y = r \sin \theta
%     \end{cases}
% \]

\section{第二型曲线积分}

第二型曲线积分是对向量场的积分,它是曲线积分的一般形式。

计算步骤:

\begin{enumerate}
    \item 画出所积曲线的示意图,并转化为定积分的形式:
          \[
              \int_{L} P(x,y) \, \mathrm{d}x + \int{L} Q(x,y) \, \mathrm{d}y = \int_{L} P(x,y) \, \mathrm{d}x + Q(x,y) \, \mathrm{d}y = \int_{L} \vec{F}(x,y) \cdot \, \mathrm{d}\vec{r}
          \]
    \item 确定积分区间 $[x_1,x_2]$ 或 $[y_1,y_2]$ 或 $[t_1,t_2]$(注意有方向)
    \item 计算 $\mathrm{d}x$ 和 $\mathrm{d}y$ :
          \[
              \begin{cases}
                  \mathrm{d}x = \frac{\mathrm{d}x}{\mathrm{d}t} \, \mathrm{d}t \\
                  \mathrm{d}y = \frac{\mathrm{d}y}{\mathrm{d}t} \, \mathrm{d}t
              \end{cases}
          \]
    \item 将 $\mathrm{d}x$ 和 $\mathrm{d}y$ 代入积分式中,计算积分
\end{enumerate}

性质:

\begin{enumerate}
    \item 线积分与路径无关:
          \[
              \int_{L} \vec{F}(x,y) \cdot \, \mathrm{d}\vec{r} = \int_{L_1} \vec{F}(x,y) \cdot \, \mathrm{d}\vec{r} = \int_{L_2} \vec{F}(x,y) \cdot \, \mathrm{d}\vec{r}
          \]
    \item 线积分与参数化无关
          \[
              \int_{L} (\alpha \vec{F}_1 (x,y) + \beta \vec{F}_2 (x,y)) \cdot \, \mathrm{d}\vec{r} = \alpha \int_{L} \vec{F}_1 (x,y) \cdot \, \mathrm{d}\vec{r} + \beta \int_{L} \vec{F}_2 (x,y) \cdot \, \mathrm{d}\vec{r}
          \]
    \item 线积分与方向有关
          \[
              -\int_{L} \vec{F}(x,y) \cdot \, \mathrm{d}\vec{r} = \int_{-L} \vec{F}(x,y) \cdot \, \mathrm{d}\vec{r}
          \]
\end{enumerate}