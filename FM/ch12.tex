\chapter{数项级数}

\section{级数的收敛性}

\subsection{级数的概念}

\begin{definition}
    设 $\{a_{n}\}$ 是一个数列,称 $\sum\limits_{n=1}^{\infty} a_{n}$ 为数项级数,记作 $\sum a_{n}$,其中 $a_{n}$ 称为级数的通项。
\end{definition}

\subsection{级数的收敛性}

\begin{definition}
    设 $\sum a_{n}$ 是一个数项级数,如果数列 $\{S_{n}\}$ 收敛,其中 $S_{n} = \sum\limits_{i=1}^{n} a_{i}$,则称级数 $\sum a_{n}$ 收敛,否则称级数 $\sum a_{n}$ 发散。
\end{definition}

\begin{theorem}[级数收敛的柯西准则]
    设 $\sum a_{n}$ 是一个数项级数,级数 $\sum a_{n}$ 收敛的充分必要条件是:对于任意 $\varepsilon > 0$,存在正整数 $N$,使得对于任意 $m > N$ 及任意的正整数 $p$,有
    \begin{equation*}
        \left| a_{m+1} + a_{m+2} + \cdots + a_{m+p} \right| < \varepsilon
    \end{equation*}
\end{theorem}

\begin{theorem}[级数收敛的必要条件]
    设 $\sum a_{n}$ 是一个数项级数,如果 $\sum a_{n}$ 收敛,则 $\lim\limits_{n \to \infty} a_{n} = 0$。
\end{theorem}

\begin{theorem}
    设 $\sum a_{n}$ 和 $\sum b_{n}$ 是两个数项级数,则对任意常数 $c,d$,级数 $\sum (ca_{n} + db_{n})$ 也收敛,并且有
    \begin{equation*}
        \sum (ca_{n} + db_{n}) = c \sum a_{n} + d \sum b_{n}
    \end{equation*}
\end{theorem}

\begin{theorem}
    去掉、增加或改变级数的有限项并不改变级数的敛散性。
\end{theorem}

\begin{theorem}
    在收敛级数的项中任意加括号, 既不改变级数的收敛性,也不改变它的和。
\end{theorem}

\subsubsection{常见的级数}

\begin{enumerate}
    \item 等比级数:$\sum\limits_{n=0}^{\infty} a q^{n} = \frac{a}{1-q}$,其中 $|q| < 1$ 时收敛。
    \item 调和级数:$\sum\limits_{n=1}^{\infty} \frac{1}{n}$ 发散。
    \item $p$ 级数:$\sum\limits_{n=1}^{\infty} \frac{1}{n^{p}}$ 当 $p > 1$ 时收敛,$p \leq 1$ 时发散。
\end{enumerate}

\section{正项级数}

\subsection{正项级数收敛性的一般判别原则}

\begin{theorem}[比较原则]
    设 $\sum a_{n}$ 和 $\sum b_{n}$ 是两个正项级数,如果存在正整数 $N$,使得对于任意 $n > N$,有 $a_{n} \leq b_{n}$,则有
    \begin{enumerate}
        \item 若 $\sum b_{n}$ 收敛,则 $\sum a_{n}$ 也收敛;
        \item 若 $\sum a_{n}$ 发散,则 $\sum b_{n}$ 也发散。
    \end{enumerate}
\end{theorem}

在实际使用上,比较原则的极限形式通常更方便

\begin{theorem}[比较原则的极限形式]
    设 $\sum a_{n}$ 和 $\sum b_{n}$ 是两个正项级数,如果 $\lim\limits_{n \to \infty} \frac{a_{n}}{b_{n}} = A$,则有
    \begin{enumerate}
        \item 若 $0 < A < +\infty$,则 $\sum a_{n}$ 与 $\sum b_{n}$ 同时收敛或同时发散;
        \item 若 $A = 0$ 且 $\sum b_{n}$ 收敛,则 $\sum a_{n}$ 也收敛;
        \item 若 $A = +\infty$ 且 $\sum b_{n}$ 发散,则 $\sum a_{n}$ 也发散。
    \end{enumerate}
\end{theorem}

\subsection{比式判别法和根式判别法}

\begin{theorem}[比式判别法]
    设 $\sum a_{n}$ 是一个正项级数,如果且存在某正整数 $N$,及常数 $q(0<q<1)$,使得对于任意 $n > N$,有
    \begin{enumerate}
        \item $\frac{a_{n+1}}{a_{n}} \leq q$,则 $\sum a_{n}$ 收敛;
        \item $\frac{a_{n+1}}{a_{n}} \geq 1$,则 $\sum a_{n}$ 发散。
    \end{enumerate}
\end{theorem}

\begin{theorem}[比式判别法的极限形式]
    设 $\sum a_{n}$ 是一个正项级数,如果 $\lim\limits_{n \to \infty} \frac{a_{n+1}}{a_{n}} = q$,则有
    \begin{enumerate}
        \item 若 $q < 1$,则 $\sum a_{n}$ 收敛;
        \item 若 $q > 1$ 或 $q = +\infty$,则 $\sum a_{n}$ 发散。
    \end{enumerate}
\end{theorem}

\begin{theorem}[根式判别法]
    设 $\sum a_{n}$ 是一个正项级数,且存在某正数 $N$ 及常数 $l$,使得对于任意 $n > N$,有
    \begin{enumerate}
        \item $\sqrt[n]{a_{n}} \leq l < 1$,则 $\sum a_{n}$ 收敛;
        \item $\sqrt[n]{a_{n}} \geq 1$,则 $\sum a_{n}$ 发散。
    \end{enumerate}
\end{theorem}

\begin{theorem}[根式判别法的极限形式]
    设 $\sum a_{n}$ 是一个正项级数,如果 $\lim\limits_{n \to \infty} \sqrt[n]{a_{n}} = l$,则有
    \begin{enumerate}
        \item 若 $l < 1$,则 $\sum a_{n}$ 收敛;
        \item 若 $l > 1$,则 $\sum a_{n}$ 发散。
    \end{enumerate}
\end{theorem}

\subsection{积分判别法}

\begin{theorem}[积分判别法]
    设 $f$ 为 $[1,+\infty)$ 上的连续正函数,且单调递减,那么正项级数 $\sum f(n)$ 与反常积分 $\int_{1}^{+\infty} f(x) \, \mathrm{d}x$ 同时收敛或同时发散。
\end{theorem}

\section{一般项级数}

\subsection{交错级数}

\begin{definition}
    若级数的各项符号正负相间,即 $\sum (-1)^{n} a_{n}$,则称该级数为交错级数。
\end{definition}

\begin{theorem}[莱布尼茨判别法]
    若交错级数满足
    \begin{enumerate}
        \item $a_{n} \geq 0$;
        \item $\lim\limits_{n \to \infty} a_{n} = 0$ 单调递减;
    \end{enumerate}
    则交错级数 $\sum (-1)^{n} a_{n}$ 收敛。
\end{theorem}

\subsection{绝对收敛级数及其性质}

\begin{definition}
    若级数 $\sum |a_{n}|$ 收敛,则称级数 $\sum a_{n}$ 绝对收敛。
\end{definition}

\begin{theorem}
    绝对收敛级数必收敛。
\end{theorem}

\begin{theorem}[条件收敛]
    若级数 $\sum a_{n}$ 收敛,但 $\sum |a_{n}|$ 发散,则称级数 $\sum a_{n}$ 为条件收敛级数。
\end{theorem}

\begin{theorem}[级数的重排]
    设 $\sum a_{n}$ 是一个绝对收敛级数,对于任意一一对应的正整数 $n_{1},n_{2},\cdots,n_{k},\cdots$,级数 $\sum a_{n_{k}}$ 也收敛,并且有
    \begin{equation*}
        \sum a_{n} = \sum a_{n_{k}}
    \end{equation*}
\end{theorem}

\begin{theorem}
    绝对收敛级数的任意重排仍然绝对收敛,且其和不变。
\end{theorem}

\begin{theorem}[级数的乘积]
    若 $\sum a_{n}$ 为收敛级数,a为常数,则 $\sum u a_{n} = u \sum a_{n}$。
\end{theorem}

% \begin{theorem}[柯西定理]
%     设 $\sum a_{n}$ 为绝对收敛级数,则对于任意 $\varepsilon > 0$,存在正整数 $N$,使得对于任意 $m > N$ 及任意的正整数 $p$,有
%     \begin{equation*}
%         \left| a_{m+1} + a_{m+2} + \cdots + a_{m+p} \right| < \varepsilon
%     \end{equation*}
% \end{theorem}

\subsection{阿贝尔判别法和狄利克雷判别法}

\begin{theorem}[阿贝尔判别法]
    设 $\sum a_{n}$ 为一个级数,如果
    \begin{enumerate}
        \item $\sum a_{n}$ 的部分和数列 $\{S_{n}\}$ 有界;
        \item $\sum b_{n}$ 单调趋于零。
    \end{enumerate}
    则 $\sum a_{n} b_{n}$ 收敛。
\end{theorem}

\begin{theorem}[狄利克雷判别法]
    设 $\sum a_{n}$ 为一个级数,如果
    \begin{enumerate}
        \item $\sum a_{n}$ 的部分和数列 $\{S_{n}\}$ 有界;
        \item $\sum b_{n}$ 单调趋于零且单调递减。
    \end{enumerate}
    则 $\sum a_{n} b_{n}$ 收敛。
\end{theorem}

\section{幂级数}

\subsection{幂级数的概念}

\begin{definition}
    设 $\{a_{n}\}$ 是一个数列,称级数 $\sum a_{n} x^{n}$ 为幂级数。
\end{definition}

\begin{theorem}[阿贝尔定理]
    若幂级数 $\sum a_{n} x^{n}$ 在 $x = x_{0}$ 处收敛,那么当 $|x|<|x_0|$ 时该幂级数一定绝对收敛;反之当 $x=x_0$ 发散,那么当 $|x|<|x_0|$ 时该幂级数一定发散。
\end{theorem}

\begin{definition}[收敛半径]
    设幂级数 $\sum a_{n} x^{n}$ 的收敛半径为 $R$,则当 $x \in (-R,R)$ 时,该幂级数收敛。
\end{definition}

收敛半径的计算:

设幂级数 $\sum a_{n} x^{n}$ 的收敛半径为 $R$,则

\[\rho = \lim\limits_{n \to \infty} \left| \frac{a_{n+1}}{a_{n}} \right| \]
\[R = \frac{1}{\rho}\]

\subsection{函数展开成幂级数}

\begin{theorem}
    设函数 $f(x)$ 在 $x_{0}$ 处有 $n$ 阶导数,且 $f^{(n)}(x)$ 在 $x_{0}$ 处连续,那么 $f(x)$ 在 $x_{0}$ 处展开成幂级数的表达式为
    \[f(x) = f(x_{0}) + f'(x_{0})(x-x_{0}) + \cdots + \frac{f^{(n)}(x_{0})}{n!}(x-x_{0})^{n} + o((x-x_{0})^{n})\]
\end{theorem}

\subsubsection{利用泰勒公式直接展开}

由泰勒级数理论可知, 函数 $f(x)$展开成幂级数的步骤如下 :
\begin{enumerate}
    \item 求函数及其各阶导数在 $x = 0$ 处的值 ;
    \item 写出 $x_0=0$ 麦克劳林级数 $f(x)=\sum\limits_{n=0}^{\infty}\frac{f^{(n)}(0)}{n!}(0)x^n$ , 并求出其收敛半径 $R$ ;
    \item 判别在收敛区间 $(-R,R)$ 内 $\lim_{n \to \infty} R_n(x)$ 是否为$0$
\end{enumerate}

常见的泰勒级数
\begin{itemize}
    \item $e^{x} = \sum\limits_{n=0}^{\infty} \frac{x^{n}}{n!}$;
    \item $\sin x = \sum\limits_{n=0}^{\infty} (-1)^{n} \frac{x^{2n+1}}{(2n+1)!}$;
    \item $\cos x = \sum\limits_{n=0}^{\infty} (-1)^{n} \frac{x^{2n}}{(2n)!}$;
    \item $\ln(1+x) = \sum\limits_{n=1}^{\infty} (-1)^{n-1} \frac{x^{n}}{n}$;
    \item $\arctan x = \sum\limits_{n=0}^{\infty} (-1)^{n} \frac{x^{2n+1}}{2n+1}$;
    \item $\frac{1}{1-x} = \sum\limits_{n=0}^{\infty} x^{n},x\in(-1,1)$;
    \item $\frac{1}{1+x} = \sum\limits_{n=0}^{\infty} (-1)^{n} x^{n},x\in(-1,1)$;
\end{itemize}

\subsubsection{利用已知幂级数展开}

\subsection{求幂级数的和函数}

\begin{theorem}[性质]
    \begin{enumerate}
        \item 逐项可积性:设幂级数 $\sum a_{n} x^{n}$ 的和函数 $S(t)$ 在收敛域上可积,
              \[\int_{0}^{x}S(t)dt=\int_{0}^x(\sum a_n t^n)dt =\sum \frac{a_n}{n+1}x^{n+1}\]
        \item 逐项微分性:设幂级数 $\sum a_{n} x^{n}$ 的和函数 $S(t)$ 在收敛域上可导,
              \[S'(x)=\sum na_n x^{n-1}\]
    \end{enumerate}
\end{theorem}